\chapter{Информационно-управляющая система}
Информационно-управляющая система представляет собой прикладное программное обеспечение, выполненное на базе платформы <<Windows Presentation Foundation>> (WPF), являющейся частью фреймворка <<.NET>>.

Выбор среды разработки обусловлен тем, что пользовательское приложение должно объединять в себе отображение функционального пользовательского интерфейса с возможностью проведения ресурсоёмких операций, таких как вычисление траекторий движения, генерация и компиляция программ работы, за приемлимое время.


\section{Функциональная схема системы}
На рисунке~\ref{fig:SystemSchema} изображена функциональная схема разработанной информационно-управляющей системы.

\begin{figure}[H]
    \centering
    \vspace{14pt}
    \includegraphics[width=\linewidth]{Figures/Software/SystemSchema}
    \caption{Функциональная схема разработанной информационно-управляющей системы}
    \label{fig:SystemSchema}
\end{figure}

Как видно из рисунка~\ref{fig:SystemSchema}, система разделена на подсистемы, каждая из которых отвечает за выполнение конкретных задач.

\subsection{Подсистема отображения}
Подсистема отображения отвечает за предоставление оператору полнофункционального пользовательского интерфейса.

Основным экраном пользовательского интерфейса является экран настройки параметров работы, приведённый на рисунке~\ref{fig:UserInterface}.

\begin{figure}[H]
    \centering
    \vspace{14pt}
    \includegraphics[width=\linewidth]{Figures/Software/UserInterface}
    \caption{Экран настройки параметров работы}
    \label{fig:UserInterface}
\end{figure}

Работая с представленным на рисунке~\ref{fig:UserInterface} интерфейсом, оператор имеет возможность редактировать параметры изделия, ограничивать область поверхности наплавки, задавать диаметр входного отверстия и расстояние от входного отверстия до поверхности наплавки.
После чего, оператор производит корректировку траектории движения рабочего инструмента, задаёт скорость движения, определяет частоту и амплитуду соответствующих выбранной конфигурации колебаний.
При изменении оператором параметров траектории движения, её вид обновляется в пользовательском интерфейсе в режиме реального времени.

По завершении редактирования, все параметры работы передаются в подсистему конфигурации, где значения параметров проходят проверку на корректность и сохраняются.

\subsection{Подсистема конфигурации}
Подсистема конфигурации отвечает за сохранение новых и управление существующими конфигурациями программ работы робота.

Каждая конфигурация представляет собой набор значений всех параметров, полностью описывающий изделие и рабочую программу робота.
При сохранении конфигурации, подсистема также отвечает за проверку значений параметров на корректность.
Проверка значений осуществляется по предустановленным правилам.
Например, такие значения, как скорости движения рабочего инструмента и частота колебаний должны задаваться в виде положительных чисел с определёнными верхними и нижними ограничениями.

Все конфигурации хранятся в базе конфигураций и загружаются по запросу от оператора, для чего подсистема конфигурации предоставляет соответствующий интерфейс для подсистемы отображения.

Конфигурации сохраняются в текстовом формате JSON .
Этот формат отличается повсеместной поддержкой и наличием готового инструментария для работы во всех языках программирования, а также простотой отображения структурированной информации в текстовом виде, благодаря чему, её можно просматривать и редактировать при помощи любого текстового редактора.
Пример текстового отображения структурированной информации в формате JSON приведён в листинге~\ref{lst:JSON}.

\begin{lstlisting}[language={json}, caption={Текстовое отображение информации в формате JSON}, label={lst:JSON}]
{
	"Name": "TestDetail",
	"Detail": {
		"CenterPosition": {
			"X": 0.0,
			"Y": 0.0,
			"Z": 0.0
		},
		"DiameterMinimum": 50.0,
		"DiameterMaximum": 100.0
	},
	"WeldingConfiguration": {
		"AmplitudeHeight": 10.0,
		"AmplitudeWidth": 10.0,
		"Speed": 15.0,
		"Frequency": 40.0,
		"Overlap": 0.0,
		"NumberOfDiameters": 3,
		"NumberOfPoints": 1000,
		"StartAngle": 0.0
	}
}
\end{lstlisting}

\subsection{Подсистема создания программ}
Программы работы для контроллеров FANUC имеют два основных расширения.
Первоначальные файлы программ имеют расширение <<.LS>>, что является текстовым представлением программы в кодировке ASCII в виде листинга, благодаря чему, с ними удобно работать как с обычными текстовыми файлами.
Файлы с расширение <<.TP>> -- это скомпилированные файлы программы в виде байт-кода.

Новые версии контроллеров FANUC поставляются с предустановленным пакетом <<ASCII Upload>>, назначением которого является компиляция текстовых файлов программ с расширением <<.LS>> при их загрузке на контроллер.
Контроллеры более старых серий, каким и является используемый в работе FANUC R-J3iB, этого пакета не имеют, так что поддерживают загрузку исключительно скомпилированных файлов с расширением <<.TP>>.

Подсистема создания программ отвечает за преобразование конфигурации, заданной оператором, в программу работы для контроллера робота.

Первоначально, программы создаются в виде текстовых <<.LS>> файлов.
Так как в работе используется контроллер без предустановленного пакета расширения <<ASCII Upload>>, то подсистема создания программ также отвечает за компиляцию текстовых <<.LS>> файлов в готовые для отправки на контроллер робота байтовые <<.TP>> файлы.

\subsection{Подсистема коммуникации}
Подсистема коммуникации отвечает за установку и поддержание соединения с контроллером робота.
Она предоставляет интерфейсы для других подсистем, позволяющие получать данные о состоянии робота и его текущем положении, считывать и записывать значения внутренних переменных контроллера робота.

Также подсистема коммуникации отвечает за загрузку скомпилированных программ работы робота по протоколу FTP .
Во время передачи программ, контроллер робота выступает в роли сервера, а подсистема коммуникации в роли клиента.

\subsection{Подсистема управления}
Для функционирования разработанной системы, на контроллере робота резервируются внутренние регистры, выполняющие роль переменных.
На контроллерах FANUC используются два типа регистров: численные регистры <<R>> для хранения целочисленных и дробных значений и позиционные регистры <<PR>> для хранения позиций.

В процессе работы, значения регистров могут быть записаны и считаны как внутри программы работы, так и при помощи комманд разработанной системы управления, за что отвечает соответствующий интерфейс подсистемы коммуникации.

Таким образом, при помощи считывания и записи значений регистров осуществляется коммуникация между системой управления и контроллером робота.

Перечень задействованных регистров с их назначением приведён в таблице~\ref{tab:VariablesMap}.

\begin{longtable}[H]{|p{0.06\linewidth}|p{0.20\linewidth}|p{0.60\linewidth}|}
    \caption{Список используемых для коммуникации переменных}
    \label{tab:VariablesMap}
    \\ \hline
    № & Наименование & Описание \\ \hline

    R1 & ProgramNumber & Номера программы работы робота.
    Задаётся управляющей системой. \\ \hline

    R2 & MovePermission & Разрешение на выполнение программ, связанных с движением робота.
    Устанавливается управляющей программой. \newline

    0 -- движение запрещено. \newline
    1 -- движение разрешено. \\ \hline

    R3 & WeldPermission & Разрешение на использование сварочного аппарата.
    Устанавливается управляющей программой. \newline

    0 -- наплавка запрещена. \newline
    1 -- наплавка разрешена. \\ \hline

    R4 & Status & Статус контроллера робота.
    Устанавливается контроллером робота.
    Может быть изменён управляющей системой как способ квитирования возникшей ошибки. \newline

    0 -- Ready, готовность к выполнению программы. \newline
    1 -- Busy, робот в процессе выполнения программы, перемещения. \newline
    2 -- Welding, робот в процессе наплавочных работ. \newline
    3 -- Error, контроллер робота в состоянии ошибки. \\ \hline

    PR1 & HomePosition & Начальное положение робота, в которое он возвращается по окончании выполнения операций.
    Задаётся единожды на этапе пусконаладочных работ. \\ \hline

    PR2, PR3, \ldots, PR* & TouchPosition & Произвольное количество позиционных регистров, предназначенных для хранения положений, определяемых при поиске положения и ориентации изделия путём ощупывания электродом. \\ \hline
\end{longtable}

Подсистема управления отвечает за считывание и запись значений перечисленных регистров при помощи соответствующего интерфейса обмена данными, предоставляемого подсистемой коммуникации.

Также подсистема управления инициирует начало процессов генерации управляющих программ и их загрузки на контроллер робота.

\subsection{Подсистема мониторинга}
Подсистема мониторинга в режиме реального времени отслеживает изменения положения робота и его состояний при помощи соответствующего интерфейса, предоставляемого подсистемой коммуникации.

Все изменения передаются в подсистему отображения для информирования оператора.


\section{Главная программа работы контроллера}
Для функционирования комплекса необходимо, чтобы на контроллере робота была выбрана и запущена главная управляющая программа <<Main>>.
Задачей программы <<Main>> является ожидание изменения значения регистра R1 (ProgramNumber).
Таким образом, управляющей системой осуществляется выбор необходимой для запуска программы работы робота.

Блок-схема функционирования программы <<Main>> изображена на рисунке~\ref{fig:MainProgramSchema}.

\begin{figure}[H]
    \centering
    \vspace{14pt}
    \includegraphics[width=\linewidth]{Figures/Software/MainProgramSchema}
    \caption{Блок-схема функционирования программы <<Main>>}
    \label{fig:MainProgramSchema}
\end{figure}

При изменении регистра R1, программа работы <<Main>> запускает соответствующую значению регистра программу, если такая программа загружена на контроллер.

Если же значение регистра R1 указывает на несуществующую программу, контроллер устанавливает регистр R4 (Status) в значение Error, сигнализирующее о том, что произошла ошибка.
Выполнение программы <<Main>> приостанавливается до тех пор, пока оператор не изменит статус регистра R4 при помощи пользовательского интерфейса управляющей системы.
Такой подход к квитированию позволяет исключить возможность того, что оператор не будет уведомлен о возникновении ошибки.

После завершения выполнения запущенной программы работы, или после квитирования ошибки, если программа работы отсутствует на контроллере, программа <<Main>> возвращается в своё первоначальное состояние и продолжает ожидать изменение значения регистра R1.


\section{Описание функционирования системы}
При запуске информационно-управляющей системы происходит автоматический процесс инициализации.

Сперва, происходит установка соединения между управляющей системой и контроллером робота при помощи подсистемы коммуникации.
Если установить соединение не удаётся, подсистема коммуникации продолжает попытки подключения в фоновом режиме.

При успешном подключении системы к контроллеру, подсистема мониторинга начинает непрерывное отслеживание положения робота и его состояний.

Также происходит инициализация подсистемой управления зарезервированных для работы регистров, таких, как разрешение на движение или на выполнение наплавочных работ.
Инициализация регистров управления заключается в занулении их значений.
Перечень всех регистров приведён в таблице~\ref{tab:VariablesMap}.

После чего происходит проверка значения позиционного регистра PR1 (HomePosition).
Описываемое им начальное положение должно быть отлично от нуля.
Если начальное положение не задано, оператору необходимо его указать, так как начальное положение является безопасной точкой отвода рабочего инструмента по окончании выполнения работ.

Последним шагом инициализации является проверка наличия на контроллере робота главной программы работы <<Main>> и её запуск.
Если программа <<Main>> отсутствует, то подсистема создания программ создаёт файл программы, компилирует его и загружает скомпилированную программу <<Main>> в виде байт-кода на контроллер робота.

На этом этапе комплекс полностью готов к выполнению работ.

Оператор устанавливает и закрепляет деталь на рабочей поверхности, котороя представляет собой сварочный стол.
При помощи пользовательского интерфейса, оператор выбирает соответствующую детали конфигурацию наплавки, проверяет параметры работы и траекторию, после чего инициирует запуск.

В соответствии с выбранной оператором конфигурацией, подсистема создания программ генерирует и компилирует программы работы, необходимые для определения положения и ориентации изделия в пространстве, происходит их загрузка на контроллер робота.

Подсистема управления поочерёдно изменяет значение регистра контроллера R1 (ProgramNumber).
Тем самым, подсистема управления указывает исполняемой на контроллере главной программе <<Main>> на то, какая программа работа должна быть запущена.
Поочерёдно запускаются программы определения положения и ориентации изделия.

Определение положения и ориентации осуществляется при помощи нахождения точек на поверхности детали путём касания детали сварочным электродом.
Координаты точек касания сохраняются контроллером в позиционных регистрах PR2-PR*.
По завершении выполнения программ, подсистема управления считывает сохранённые в позиционных регистрах PR2-PR* значения и расчитывает положение и ориентацию детали.

На основании этих расчётов, подсистема создания программ генерирует программы работы, описывающие траекторию наплавки с учётом найденных положения и ориентации детали.
Программы работы компилируются и загружаются на контроллер робота.

Подсистема управления при помощи изменения значения регистра R1, производит запуск программ работы наплавки.
На текущем этапе, задачей оператора комплекса является контроль качества наплавленного слоя между выполняемыми программами работы и реагирование на нештатные ситуации.
