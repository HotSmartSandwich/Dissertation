\chapter{Информационно-управляющая система}
Информационно-управляющая система представляет собой прикладное программное обеспечение, выполненное на базе платформы <<Windows Presentation Foundation>> (WPF), являющейся частью фреймворка <<.NET>>.

Выбор среды разработки обусловлен тем, что пользовательское приложение должно объединять в себе отображение функционального пользовательского интерфейса с возможностью проведения ресурсоёмких операций, таких как вычисление траекторий движения, генерация и компиляция программ работы, за приемлимое время.

На рисунке~\ref{fig:SystemSchema} изображена функциональная схема разработанной информационно-управляющей системы.

\begin{figure}[H]
    \centering
    \vspace{14pt}
    \includegraphics[width=\linewidth]{Figures/Software/SystemSchema}
    \caption{Функциональная схема разработанной информационно-управляющей системы}
    \label{fig:SystemSchema}
\end{figure}

Как видно из рисунка~\ref{fig:SystemSchema}, система разделена на подсистемы, каждая из которых отвечает за выполнение конкретных задач.


\section{Подсистема отображения}
Подсистема отображения отвечает за предоставление оператору полнофункционального пользовательского интерфейса.

Основным экраном пользовательского интерфейса является экран настройки параметров работы, приведённый на рисунке~\ref{fig:UserInterface}.

\begin{figure}[H]
    \centering
    \vspace{14pt}
    \includegraphics[width=\linewidth]{Figures/Software/UserInterface}
    \caption{Экран настройки параметров работы}
    \label{fig:UserInterface}
\end{figure}

Работая с представленным на рисунке~\ref{fig:UserInterface} интерфейсом, оператор имеет возможность редактировать параметры изделия, ограничивать область поверхности наплавки, задавать диаметр входного отверстия и расстояние от входного отверстия до поверхности наплавки.
После чего, оператор производит корректировку траектории движения рабочего инструмента, задаёт скорость движения, определяет частоту и амплитуду соответствующих выбранной конфигурации колебаний.
При изменении оператором параметров траектории движения, её вид обновляется в пользовательском интерфейсе в режиме реального времени.

По завершении редактирования, все параметры работы передаются в подсистему конфигурации, где значения параметров проходят проверку на корректность и сохраняются.


\section{Подсистема конфигурации}
Подсистема конфигурации отвечает за сохранение новых и управление существующими конфигурациями программ работы робота.

Каждая конфигурация представляет собой набор значений всех параметров, полностью описывающий изделие и рабочую программу робота.
При сохранении конфигурации, подсистема также отвечает за проверку значений параметров на корректность.
Проверка значений осуществляется по предустановленным правилам.
Например, такие значения, как скорости движения рабочего инструмента и частота колебаний должны задаваться в виде положительных чисел с определёнными верхними и нижними ограничениями.

Все конфигурации хранятся в базе конфигураций и загружаются по запросу от оператора, для чего подсистема конфигурации предоставляет соответствующий интерфейс для подсистемы отображения.

Конфигурации сохраняются в текстовом формате JSON .
Этот формат отличается повсеместной поддержкой и наличием готового инструментария для работы во всех языках программирования, а также простотой отображения структурированной информации в текстовом виде, благодаря чему, её можно просматривать и редактировать при помощи любого текстового редактора.
Пример текстового отображения структурированной информации в формате JSON приведён в листинге~\ref{lst:JSON}.

\begin{lstlisting}[caption={Текстовое отображение информации в формате JSON}, label={lst:JSON}]
{
	"Detail": {
		"CenterPosition": {
			"X": 0.0,
			"Y": 0.0,
			"Z": 0.0
		},
		"DiameterMinimum": 200.0,
		"DiameterMaximum": 400.0
	},
	"WeldingConfiguration": {
		"AmplitudeHeight": 30.0,
		"AmplitudeWidth": 30.0,
		"Speed": 100.0,
		"Frequency": 100.0,
		"Overlap": 0.0,
		"NumberOfDiameters": 3,
		"NumberOfPoints": 1000,
		"StartAngle": 0.0
	}
}
\end{lstlisting}


\section{Постпроцессор}
Постпроцессор отвечает за перевод траектории движения, заданной оператором, в программу работы для контроллера робота, структура которой была описана в подразделе~\ref{sec:ProgramStructure}.
Благодаря тому, что программы работы для контроллера робота в формате <<.LS>> имееют кодировку ASCII, с ними удобно работать как с обычными текстовыми файлами.

Также постпроцессор отвечает за компиляцию созданных текстовых <<.LS>> файлов в бинарные <<.TP>> файлы.
Как было описана в подразделе~\ref{sec:ProgramStructure}, это необходимо для загрузки программ работы на контроллеры, не имеющие предустановленного пакета расширения <<ASCII Upload>>.
%На первом этапе, в новый <<.LS>> файл записываются координаты всех точек в соответствии с подразделом~\ref{subsec:RobotConfiguration}.


\section{Подсистема коммуникации}
Подсистема коммуникации отвечает за установку соединения с контроллером робота.
Она предоставляет интерфейсы для других подсистем, позволяющие получать данные о состоянии робота и его текущем положении, считывать и записывать значения внутренних переменных контроллера робота.

Также подсистема коммуникации отвечает за загрузку скомпилированных программ работы робота по протоколу FTP .
Во время передачи программ, контроллер робота выступает в роли сервера, а подсистема коммуникации в роли клиента.


\section{Подсистема управления}
Для функционирования разработанной системы, на контроллере робота резервируются внутренние переменные.
При помощи считывания и записи значений этих переменных, осуществляется коммуникация между контроллером робота и системой.
Отдельные переменные отводятся под разрешение на выполнение работы, разрешение на зажигание сварочной дуги, записи текущего состояния робота и так далее.

За считывание и запись этих переменных отвечает подсистема управления, использующая соответствующий интерфейс обмена данными, предоставляемый подсистемой коммуникации.

Также подсистема управления инициирует начало процесса генерации управляющих программ и их загрузку на контроллер робота.


\section{Подсистема мониторинга}
Подсистема мониторинга в режиме реального времени отслеживает изменения положения робота и его состояний при помощи соответствующего интерфейса, предоставляемого подсистемой коммуникации.

Все изменения передаются в подсистему отображения для информирования оператора.
