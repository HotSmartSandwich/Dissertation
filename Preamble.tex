% Настройки стиля ГОСТ 7-32
% Для начала определяем, хотим мы или нет, чтобы рисунки и таблицы нумеровались в пределах раздела, или нам нужна сквозная нумерация.
\EqInChapter % формулы будут нумероваться в пределах раздела
\TableInChapter % таблицы будут нумероваться в пределах раздела
\PicInChapter % рисунки будут нумероваться в пределах раздела

\setcounter{page}{1} % нумерация страниц начиная с n-ой

\usepackage{color}

\usepackage{cite}

% Математический шрифт в том числе для стрелочек векторов
\usepackage{libertine}
\usepackage[varvw]{newtxmath}
%\usepackage{amsmath}
%\usepackage{newtxmath}

% Добавляем гипертекстовое оглавление в PDF
%\usepackage[
%    bookmarks=true, colorlinks=true, unicode=true,
%    urlcolor=black,linkcolor=black, anchorcolor=black,
%    citecolor=black, menucolor=black, filecolor=black,
%]{hyperref}

% Изменение начертания шрифта --- после чего выглядит таймсоподобно.
% apt-get install scalable-cyrfonts-tex

% Настройка шрифта
%\usepackage{pscyr}
\renewcommand{\rmdefault}{ftm} % Times New Roman

% Чтобы было красиво
\usepackage{microtype}

% Отключение переносов и текст по ширине
\usepackage{ragged2e}
\tolerance=1000
\hyphenpenalty=10000
\emergencystretch=3em

\usepackage{float}

% non-italic в формулах
\usepackage{mathtools}

% вставка титульного листа
\usepackage{pdfpages}

\usepackage{graphicx}   % Пакет для включения рисунков
\graphicspath{{./Figures/}}


% Поля страницы
%\usepackage{geometry}
\geometry{left=3cm}
\geometry{right=1.5cm}
\geometry{top=1.5cm}
\geometry{bottom=2.4cm}

% Произвольная нумерация списков и многоуровневые нумерованные списки
\usepackage{enumerate}
\renewcommand{\labelenumi}{\arabic{enumi}.}
\renewcommand{\labelenumii}{\arabic{enumi}.\arabic{enumii}.}

% ячейки в несколько строчек
\usepackage{multirow}

% itemize внутри tabular
\usepackage{paralist, array}

% Перенос на новую страницу многострочных формул
\allowdisplaybreaks