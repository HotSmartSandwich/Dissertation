\chapter{Введение}
Износ рабочих поверхностей подвижных деталей в составе машин и устройств происходит почти во всех областях промышленности.
Наиболее распространёнными видами износа являются абразивный износ и контактная усталость.

% Скорее всего, эти два абзаца совершенно ненужны
Абразивный износ заключается в разрушении поверхностей металла твердыми зернами абразива при пластическом деформировании и микрорезании трущихся поверхностей.
Этот вид износа, в первую очередь, характерен для горно-добывающей и строительной промышленности.

Контактная усталость – это износ, при котором трение качения и высокое давление выступают в качестве механического воздействия одной металлической поверхности на другую.
Данный вид износа ярко выражен, например, на уплотнительных поверхностях трубопроводной арматуры и колёсах железнодорожного транспорта.
%

На арматуростроительных предприятиях, таких как ПАО «Аскольд» (г. Арсеньев), АО «Армалит» (г. Санкт-Петербург), ОАО «Завод «Буревестник» (г. Гатчина), а также других профильных предприятиях, для увеличения срока службы трубопроводной арматуры выполняются технологические процессы наплавки уплотнительных поверхностей износостойкими материалами.
