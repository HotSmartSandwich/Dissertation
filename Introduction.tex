\chapter{Введение}
Износ рабочих поверхностей подвижных деталей происходит почти во всех областях промышленности~\cite{Oo_2018}.

На рисунке~\ref{fig:Introduction:StopValve} изображён запорный клапан в разрезе, предназначенный для перекрытия потока рабочей среды.

\begin{figure}[H]
    \centering
    \vspace{14pt}
    \includegraphics[width=\linewidth]{Figures/Introduction/StopValve}
    \caption{Запорный клапан в разрезе}
    \label{fig:Introduction:StopValve}
\end{figure}

Выделенные на рисунке~\ref{fig:Introduction:StopValve} области называются уплотнительными поверхностями и, как правило, подвержены повышенному износу по причине возникновения трения <<металл по металлу>>.

На арматуростроительных предприятиях, таких как ПАО <<Аскольд>> (г. Арсеньев), АО <<Армалит>> (г. Санкт-Петербург), ОАО <<Завод <<Буревестник>>>> (г. Гатчина), а также других профильных предприятиях, для увеличения срока службы трубопроводной арматуры выполняются технологические процессы наплавки уплотнительных поверхностей износостойкими материалами.

Под наплавкой подразумеваются операции нанесения уплотнительных металлических поверхностей в установленных областях изготавливаемых деталей, осуществляемых посредством сварочного оборудования.

На ПАО <<Аскольд>> наплавка таких поверхностей осуществляется по технологии ручной дуговой наплавки неплавящимся вольфрамовым электродом в среде защитного инертного газа.
Наплавка осуществляется в несколько слоёв.
Каждый слой, в свою очередь, состоит из нескольких концентрических окружностей, называемых валиками.
Отдельные валики наплавляются последовательно, с перерывами.
Это необходимо для того, чтобы температура изделия не превышала температуру, допустимую технологическим процессом.
После наплавки каждого валика проводится зачистка поверхности от шлака и брызг металла, а также осуществляется визуальный осмотр качества наплавки.
На рисунке~\ref{fig:Introduction:WeldingDetail} приведён пример изделия в процессе выполнения наплавки.

\begin{figure}[H]
    \centering
    \vspace{14pt}
    \includegraphics[width=\linewidth]{Figures/Introduction/WeldingDetail}
    \caption{Изделие во время выполнения наплавки}
    \label{fig:Introduction:WeldingDetail}
\end{figure}

На рисунке~\ref{fig:Introduction:WeldingDetail} слева изображена деталь сразу после наплавки второго валика до зачистки поверхности.
Видно, что наплавленный слой подвержен включениям шлака тёмного цвета.
Справа изображена эта же деталь после наплавки третьего валика и зачистки поверхности металлической щёткой: поверхность чистая, имеет металлический блеск.
В таком виде детали направляются на дальнейшую механическую обработку.

Для рассматриваемого производственного направления характерен высокий процент неизбежно-технологического брака, закладываемого в стоимость продукции.
Основной причиной брака является человеческий фактор: операции наплавки крайне требовательны к строгому соблюдению технологического процесса.

Операции наплавки производятся согласно технологическому процессу, в котором описываются как все необходимые для изделия подготовительные процедуры (требования к поверхности, температурный режим), так и основные параметры работы непосредственно при наплавке.
Значения этих параметров приведены в таблице~\ref{tab:Introduction:WeldingParameters}.

\begin{table}[H]
    \footnotesize
    \caption{Режимы работы для операций наплавки}
    \label{tab:Introduction:WeldingParameters}
    \begin{tabular}{|p{0.09\linewidth}|p{0.1\linewidth}|p{0.11\linewidth}|p{0.1\linewidth}|p{0.13\linewidth}|p{0.09\linewidth}|p{0.07\linewidth}|p{0.1\linewidth}|}
        \hline
        Условный диаметр арматуры, мм & Сварочный ток, А & Напряжение на дуге, В & Амплитуда колебаний, мм & Ширина наплавляемых валиков, мм & Скорость наплавки, м/ч & Расход аргона, л/мин & Частота колебаний, кол/мин \\ \hline
        50--65                        & 140--160         & 26--28                & 5--10                   & 8--14                           & 10--12                 & 18--25               & 30--50                     \\ \hline
        80                            & 140--160         & 26--28                & 6--12                   & 10--16                          & 10--12                 & 18--25               & 20--40                     \\ \hline
        100--150                      & 150--170         & 27--29                & 8--14                   & 12--18                          & 9--11                  & 18--25               & 20--40                     \\ \hline
        200+                          & 160--180         & 28--30                & 10--16                  & 14--20                          & 8--10                  & 18--25               & 20--30                     \\ \hline
    \end{tabular}
\end{table}

Как видно из таблицы~\ref{tab:Introduction:WeldingParameters}, в технологическом процессе указываются достаточно грубые диапазоны значения параметров.
Конкретные значения из этих диапазонов выбираются непосредственно сварщиком.
И если такие параметры, как сварочные ток и напряжение, расход защитного газа можно выставить и поддерживать достаточно точно, то такие параметры, как скорость перемещения горелки, частота и амплитуда колебаний при выполнении наплавки вручную возможно соблюдать лишь приблизительно.
Работа каждого сварщика индивидуальна и может отличаться от изделия к изделию.

Кроме того, к наплавке износостойкими материалами допускаются только высококвалифицированные электрогазосварщики 5-го и 6-го разрядов, что приводит к потребности в узкоспециализированных и высокооплачиваемых кадрах.

Цель работы заключается в разработке информационно-управляющей системы для роботизированного комплекса наплавки износостойких поверхностей.

Роботизированный комплекс с информационно-управляющей системой должен удовлетворять следующим требованиям:

\begin{itemize}
    \item обеспечивать выполнение технологических процессов наплавки в полностью автоматическом режиме;
    \item участие оператора должно сводиться к предварительной подготовке роботизированного комплекса к работе, установке и креплению изделия на рабочей поверхности, выбор технологического процесса при помощи панели оператора и запуск наплавки в автоматическом режиме с панели оператора;
    \item обеспечивать заданные конструкторской и нормативно-технической документацией физико-механические и химические свойства наплавляемых поверхностей, а также их качество, отсутствие дефектов или превышений ими допустимых норм;
    \item использовать наплавку аргонодуговым методом в качестве основной технологии;
    \item автоматически определять положение установленного изделия в пространстве, тем самым компенсируя неточности в его установке оператором;
    \item отображать основные показатели текущего состояния системы и процесса;
    \item хранить основные технологические характеристики процесса наплавки как отдельные программы работы с возможностью их редактирования и повторного использования.
\end{itemize}

В соответствии с поставленной целью, были сформулированы следующие задачи:

\begin{itemize}
    \item определить состав аппаратной части роботизированного комплекса;
    \item разработать метод определения положения и ориентации установленного изделия в пространстве;
    \item определить способ генерации программ работы для робота;
    \item разработать методы коммуникации информационно-управляющей системы с контроллером робота;
    \item разработать пользовательский интерфейс для оператора роботизированного комплекса.
\end{itemize}
