\chapter{Обзор существующих решений} \label{ch:SolutionsOverview}
Идея автоматизировать родственные процессы сварки и наплавки не нова.
Ручные сварка и наплавка -- достаточно сложные процессы, требующие от сварщика высокой квалификации и опыта.
Стоит принять во внимание, что любой ручной труд индивидуален, что приводит к непостоянству качества выпускаемой продукции.
В таких условиях невозможно определить степень влияния на появление брака конкретных параметров работы.
В то же время, практика работ~\cite{Al_Sarraf_2016,Dai_2011,Gwan_Hyung_Kim,Liang_2011,Lin_2018,Xu_2008} показала, что постепенное малое отклонение, фиксирование результатов и корректировка таких параметров могли бы существенно снизить процент брака.


\section{Установки для наплавки}
Одним из наиболее популярных способов автоматизации процессов наплавки является использование специализированных наплавочных установок.
Эксплуатация таких установок описана в работах~\cite{Baskoro_2016,Deyong_You_2014,Jafari_2010,Qi_2019,Su_2010,Yi_Jinggang_2010}.
Как правило, такие установки состоят из закреплённой на трёхосевой каретке сварочной горелки и двухосевого позиционера.
Каретка регулирует расстояние рабочего инструмента до поверхности наплавки и осуществляет его перемещение в горизонтальной плоскости.
Позиционер отвечает за вращение изделия, за счёт чего формируется наплавляемый слой.
Пример описываемой установки приведён на рисунке~\ref{fig:Overview:WeldingMachine}.

\begin{figure}[H]
    \centering
    \vspace{14pt}
    \includegraphics[height=10cm]{Figures/SolutionsOverview/WeldingMachine}
    \caption{Установка для наплавки}
    \label{fig:Overview:WeldingMachine}
\end{figure}

Главным преимуществом таких установок является сравнительно невысокая стоимость при высокой жёсткости конструкции, которая, в свою очередь, позволяет достичь достаточной для рассматриваемого производственного направления повторяемости процесса.
Однако, такие установки имеют ряд значительных недостатков.

Несмотря на кажущуюся простоту, разработка программы работы для каждого нового изделия является крайне сложным и продолжительным процессом, заключающемся в эмпирическом подборе множества параметров.
Отладочные работы должны производиться оператором-сварщиком высокой квалификации, который во время пробных наплавок должен непрерывно производить визуальный контроль сварочной ванны, подбирать положение и высоту горелки, корректировать значения тока и напряжения.
Для того, чтобы разработать программу работы для нового изделия, даже опытному оператору-сварщику необходимо произвести не менее десяти пробных наплавок.

Другим важным ограничением является проприетарность программного обеспечения.
Система управления, разрабатываемая поставщиком оборудования, как правило, выполняется на базе программируемого логического контроллера и не подразумевает изменения или дальнейшей доработки.

Сами же установки имеют ряд ограничений эксплуатации.
Например, наплавка труднодоступных поверхностей внутри корпусов может обеспечиваться только при помощи наклона позиционера, что ведёт к нарушению технологического процесса, по которому наплавка должно производиться в нижнем, то есть горизонтальном, положении.


\section{Роботизированные комплексы}
Универсальным решением для автоматизации самых разных производственных процессов является организация роботизированного комплекса на основе шестиосевого робота-манипулятора.
Многозвенная конструкция не имеет такой жёсткости, как рассмотренные установки, но для процессов сварки и наплавки она не требуется.
Зато ставшая стандартной среди промышленных роботов-манипуляторов кинематическая схема <<Puma>> обеспечивает высокую досягаемость рабочего инструмента робота с сохранением возможности гибко изменять его ориентацию в пространстве, что было продемонстрировано в работах~\cite{Fan_Tien_Cheng_1997,Gupta_1990,Mei_2019}.
Для выполнения наплавки на изделия из широкого номенклатурного ряда, большая часть которых имеет труднодоступные области и индивидуальный технологический процесс, использование шестиосевого робота-манипулятора видится оптимальным с точки зрения удобства эксплуатации решением.

Единственным серьёзным недостатком такого решения является высокая стоимость современной роботизированной ячейки.
В неё входит не только сам робот, но и целый ряд сопутствующего оборудования, такого, как позиционер, сварочный аппарат, устройство подачи присадочного материала, дополнительные системы визуального и температурного контроля, а также программное обеспечение.


\section{Выводы}
Достаточно высокая повторяемость выполнения процессов сварки и наплавки гарантирует идентичность качества, что, в свою очередь, позволит при небольших отклонениях тех или иных параметров определить их влияние на качество конечной продукции.
Таким образом, автоматизация сварочных процессов ведёт к увеличению качества выпускаемой продукции и снижению процента брака.

Несмотря на то, что специализированные установки значительно дешевле, они не способны покрыть весь спектр задач наплавки износостойкими материалами.
Они хорошо подходят для типовых задач, например, для сварки труб.
Но применение их в таком технологически сложном процессе, как наплавка, когда различные изделия из широкого списка номенклатуры поступают преимущественно малыми партиями, видится малоперспективным из-за закрытого программного обеспечения и сложного процесса формирования программы работы для каждого нового изделия.

Роботизированный комплекс для наплавки на базе шестиосевого робота-манипулятора позволяет охватить весь спектр задач, связанных с наплавкой.
Возможно как совершенствование комплекса посредством внедрения дополнительного оборудования (различные измерительные системы, дополнительные системы технического зрения и т.д.), так и дальнейшее применение роботов для других задач, не связанных с технологическим процессом наплавки или сварки.
