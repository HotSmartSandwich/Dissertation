\chapter{Обзор существующих решений} \label{ch:SolutionsOverview}


\section{Способы корректировки ручной работы}
Ручные сварка и наплавка -- это достаточно сложные операции, требующие от сварщика высокой квалификации и опыта~\cite{Hattori_1965,Seregina_2018}.
Это косвено подтверждается большим количество научных работ, так или иначе связанных с усовершенствованием этих процессов.

Например, в работе~\cite{Aiteanu} описывается разработанная сварочная маска, предназначенная для повышения удобства ручных операций сварки и наплавки за счёт улучшения обзора рабочей области и контроля качества в режиме реального времени при помощи технологий дополненной реальности.
Несмотря на технологичность такого решения, оно малоприменимо в условиях реального производства: опытный сварщик способен без труда контролировать положение и скорость перемещения сварочной горелки в допустимых диапазонах.
Потому такое решение оказывается совершенно неоправданным из-за своей дороговизны и малой практической пользы.

Более применимым на практике видится подход, описанный в работе~\cite{Muller_2018}.
В ней технологический процесс аргонодуговой сварки неплавящимся электродом представлен в виде объекта управления.
Цель работы заключается в корректировке операций сварки, выполняемых вручную, при помощи механизированной регулировки длины дуги.

Похожий подход к улучшению качества процесса за счёт корректировки его параметров был описан в работах~\cite{Yang_2020,Dai_2011,Xu_2008}.
Значимым отличием является то, что регулирование в этих работах осуществляется не за счёт перемещения сварочной горелки и изменения длины дуги, а за счёт изменения параметров работы сварочного аппарата.

Как описанная в работе~\cite{Muller_2018} стабилизирующая система, так и описанные в работах~\cite{Yang_2020,Dai_2011,Xu_2008} методы регулировки параметров, действительно позволяют добиться улучшения качества процесса, но все эти разработки не исключают из сварочных процессов влияние человеческого фактора, который и является основной причиной брака.


\section{Автоматизирование процессов наплавки}

\subsection{Установки для наплавки}
Одним из наиболее популярных способов автоматизации процессов наплавки является использование специализированных наплавочных установок.
Эксплуатация таких установок описана в работах~\cite{Baskoro_2016,Deyong_You_2014,Jafari_2010,Qi_2019,Su_2010,Yi_Jinggang_2010}.
Как правило, такие установки состоят из закреплённой на трёхосевой каретке сварочной горелки и двухосевого позиционера.
Каретка регулирует расстояние рабочего инструмента до поверхности наплавки и осуществляет его перемещение в горизонтальной плоскости.
Позиционер отвечает за вращение изделия, за счёт чего формируется наплавляемый слой.
Пример описываемой установки приведён на рисунке~\ref{fig:Overview:WeldingMachine}.

\begin{figure}[H]
    \centering
    \vspace{14pt}
    \includegraphics[height=15cm]{Figures/SolutionsOverview/WeldingMachine}
    \caption{Установка для наплавки}
    \label{fig:Overview:WeldingMachine}
\end{figure}

Главным преимуществом таких установок является сравнительно невысокая стоимость при высокой жёсткости конструкции, которая, в свою очередь, позволяет достичь достаточную для рассматриваемого производственного направления повторяемость процесса.
Однако, такие установки имеют ряд значительных недостатков.

На рисунке~\ref{fig:Overview:ProgramEditor} изображено окно редактирования программы работы для одной из таких установок.

\begin{figure}[H]
    \centering
    \vspace{14pt}
    \includegraphics[width=\linewidth]{Figures/SolutionsOverview/ProgramEditor}
    \caption{Интерфейс редактирования программы работы}
    \label{fig:Overview:ProgramEditor}
\end{figure}

Несмотря на кажущуюся простоту, разработка программы работы для каждого нового изделия является крайне сложным и продолжительным процессом, заключающемся в эмпирическом подборе скорости наплавки и смещения горелки для каждого из проходов.

Отладочные работы должны производиться оператором-сварщиком высокой квалификации, который во время пробных наплавок должен непрерывно производить визуальный контроль сварочной ванны, подбирать положение и высоту горелки, корректировать значения тока и напряжения.
Для того, чтобы разработать программу работы для нового изделия, даже опытному оператору-сварщику необходимо произвести не менее десяти пробных наплавок.

Другим важным ограничением является проприетарность программного обеспечения.
Система управления, разрабатываемая поставщиком оборудования, как правило, выполняется на базе программируемого логического контроллера и не подразумевает изменения или дальнейшей доработки.
На рисунке~\ref{fig:Overview:ProgramInterface} представлен основной экран пользовательского интерфейса.

\begin{figure}[H]
    \centering
    \vspace{14pt}
    \includegraphics[width=\linewidth]{Figures/SolutionsOverview/ProgramInterface}
    \caption{Основное окно интерфейса}
    \label{fig:Overview:ProgramInterface}
\end{figure}

Главным недостатком изображённого на рисунке~\ref{fig:Overview:ProgramInterface} пользовательского интерфейса является малая информативность, которую, как было сказано ранее, невозможно доработать или, например, скорректировать под изменившиеся задачи.

Сами же установки имеют ряд ограничений эксплуатации.
Например, наплавка труднодоступных поверхностей внутри корпусов может обеспечиваться только при помощи наклона позиционера, что ведёт к нарушению технологического процесса, по которому наплавка должно производиться в нижнем, то есть горизонтальном, положении.

\subsection{Роботизированные комплексы}
Универсальным решением для автоматизации самых разных производственных процессов является организация роботизированного комплекса на основе шестиосевого робота-манипулятора.
Многозвенная конструкция не имеет такой жёсткости, как рассмотренные установки, но для процессов сварки и наплавки она не требуется.
Зато ставшая стандартной среди промышленных роботов-манипуляторов кинематическая схема <<Puma>> обеспечивает высокую досягаемость рабочего инструмента робота с сохранением возможности гибко изменять его ориентацию в пространстве, что было продемонстрировано в работах~\cite{Fan_Tien_Cheng_1997,Gupta_1990,Mei_2019}.
Для выполнения наплавки на изделия из широкого номенклатурного ряда, большая часть которых имеет труднодоступные области и индивидуальный технологический процесс, использование шестиосевого робота-манипулятора видится оптимальным с точки зрения удобства эксплуатации решением.

Единственным серьёзным недостатком такого решения является высокая стоимость современной роботизированной ячейки.
В неё входит не только сам робот, но и целый ряд сопутствующего оборудования, такого, как позиционер, сварочный аппарат, устройство подачи присадочного материала, дополнительные системы визуального и температурного контроля, а также программное обеспечение.


\section{Требования к аппаратно-программному комплексу}
Одним из главных недостатков любого ручного труда является его индивидуальность, что приводит к непостоянству качества выпускаемой продукции.
Практика работ~\cite{Al_Sarraf_2016,Dai_2011,Gwan_Hyung_Kim,Liang_2011,Lin_2018,Xu_2008} показала, что постепенное малое отклонение параметров работы позволяет определить их степень влияния на появление брака.
Поэтому, основным требованием к комплексу для наплавки износостойких поверхностей является обеспечение достаточно высокой повторяемости процесса для того, чтобы при помощи малых отклонений тех или иных параметров работы иметь возможность оценить их влияние на качество конечной продукции.

Задачей аппаратно-программного комплекса является выполнение технологических процессов наплавки в полностью автоматическом режиме с заданной скоростью и точностью.

Комплекс должен обеспечивать заданные конструкторской и нормативно-технической документацией физико-механические и химические свойства наплавляемых поверхностей, а также их качество, отсутствие дефектов или превышений ими допустимых норм.
При выполнении операций наплавки не допускается повреждение и появление дефектов основного тела
изделия или заготовки.


\section{Выводы}
Достаточно высокая повторяемость выполнения процессов сварки и наплавки гарантирует идентичность качества, что, в свою очередь, позволит при небольших отклонениях тех или иных параметров определить их влияние на качество конечной продукции.
Таким образом, автоматизация сварочных процессов ведёт к увеличению качества выпускаемой продукции и снижению процента брака.

Несмотря на то, что специализированные установки значительно дешевле, они не способны покрыть весь спектр задач наплавки износостойкими материалами.
Они хорошо подходят для типовых задач, например, для сварки труб.
Но применение их в таком технологически сложном процессе, как наплавка, когда различные изделия из широкого списка номенклатуры поступают преимущественно малыми партиями, видится малоперспективным из-за закрытого программного обеспечения и сложного процесса формирования программы работы для каждого нового изделия.

Роботизированный комплекс для наплавки на базе шестиосевого робота-манипулятора позволяет охватить весь спектр задач, связанных с наплавкой.
Возможно как совершенствование комплекса посредством внедрения дополнительного оборудования (различные измерительные системы, дополнительные системы технического зрения и т.д.), так и дальнейшее применение роботов для других задач, не связанных с технологическим процессом наплавки или сварки.
