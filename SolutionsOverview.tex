\chapter{Обзор существующих решений}
Ручные сварка и наплавка -- это достаточно сложные операции, требующие от сварщика высокой квалификации и опыта~\cite{Hattori_1965, Seregina_2018}.
Это косвено подтверждается большим количество научных работ, так или иначе связанных с облегчением труда сварщиков и усовершенствованием технологического процесса.


\section{Способы корректирования ручной работы}
В работе~\cite{Aiteanu} описывается разработанная сварочная маска, предназначенная для повышения удобства ручных операций сварки и наплавки за счёт улучшения обзора рабочей области и контроля качества в режиме реального времени при помощи технологий дополненной реальности.
Несмотря на технологичность такого решения, оно малоприменимо в условиях реального производства: опытный сварщик способен без труда поддерживать правильное положение сварочной горелки.
Потому такое решение оказывается совершенно неоправданным из-за своей дороговизны и малой практической пользы.

Более применимым на практике видится подход, описанный в работе~\cite{Muller_2018}.
В ней технологический процесс аргонодуговой сварки неплавящимся электродом представлен в виде объекта управления.
Цель работы заключается в корректировке операций сварки, выполняемых вручную, при помощи механизированной регулировки длины дуги.

Похожий подход к улучшению качества процесса за счёт корректировки его параметров был описан в работах~\cite{Yang_2020, Dai_2011, Xu_2008}.
Значимым отличием является то, что регулирование в этих работах осуществляется не за счёт перемещения сварочной горелки и изменения длины дуги, а за счёт изменения параметров работы сварочного аппарата.

Как описанная в работе~\cite{Muller_2018} стабилизирующая система, так и описанные в работах~\cite{Yang_2020, Dai_2011, Xu_2008} методы регулировки параметров, действительно позволяют добиться улучшения качества процесса, но все эти разработки не исключают из сварочных операций влияние человеческого фактора, который и является основной причиной брака.
Для этого необходимо полностью автоматизировать процесс.


\section{Автоматизация процессов наплавки}

\subsection{Установки для наплавки} \label{subsec:WeldingMachines}
Одним из наиболее популярных способов автоматизации процессов наплавки является использование специализированных наплавочных установок.
Эксплуатация таких установок описана в работах~\cite{Baskoro_2016, Deyong_You_2014, Jafari_2010, Su_2010, Yi_Jinggang_2010}.
Как правило, установки для наплавки состоят из закреплённой на трёхосевой каретке сварочной горелки и двухосевого позиционера.
В работе~\cite{Qi_2019} рассматривается сварочная машина для сварки швов большого диаметра с аналогичным способом крепления рабочего инструмента.

Каретка регулирует расстояние рабочего инструмента до поверхности наплавки и осуществляет его перемещение в горизонтальной плоскости.
Позиционер отвечает за вращение изделия, за счёт чего формируется наплавляемый слой.
Пример описываемой установки приведён на рисунке~\ref{fig:Overview:WeldingMachine}.

\begin{figure}[H]
    \centering
    \vspace{14pt}
    \includegraphics[height=15cm]{Figures/SolutionsOverview/WeldingMachine}
    \caption{Установка для наплавки}
    \label{fig:Overview:WeldingMachine}
\end{figure}

Главным преимуществом таких установок является сравнительно невысокая стоимость при высокой жёсткости конструкции, которая, в свою очередь, позволяет достичь достаточную для рассматриваемого производственного направления повторяемость процесса.
Однако, такие установки имеют ряд значительных недостатков.

На рисунке~\ref{fig:Overview:ProgramEditor} изображено окно редактирования программы работы для одной из таких установок.
Несмотря на кажущуюся простоту, разработка программы работы для каждого нового изделия является крайне сложным и продолжительным процессом, заключающемся в эмпирическом подборе скорости наплавки и смещения горелки для каждого из проходов.

\begin{figure}[H]
    \centering
    \vspace{14pt}
    \includegraphics[width=\linewidth]{Figures/SolutionsOverview/ProgramEditor}
    \caption{Интерфейс редактирования программы работы}
    \label{fig:Overview:ProgramEditor}
\end{figure}

Отладочные работы должны производиться оператором-сварщиком высокой квалификации, который во время пробных наплавок должен непрерывно производить визуальный контроль сварочной ванны, подбирать положение и высоту горелки, корректировать значения тока и напряжения.
Чтобы разработать программу работы для нового изделия даже опытному оператору-сварщику необходимо произвести не менее десяти пробных наплавок.

Другим важным ограничением является проприетарность программного обеспечения.
Система управления, разрабатываемая поставщиком оборудования, как правило, выполняется на базе программируемого логического контроллера и не подразумевает изменения или дальнейшей доработки.
На рисунке~\ref{fig:Overview:ProgramInterface} представлен основной экран пользовательского интерфейса.

\begin{figure}[H]
    \centering
    \vspace{14pt}
    \includegraphics[width=\linewidth]{Figures/SolutionsOverview/ProgramInterface}
    \caption{Основное окно интерфейса}
    \label{fig:Overview:ProgramInterface}
\end{figure}

Главным недостатком изображённого на рисунке~\ref{fig:Overview:ProgramInterface} пользовательского интерфейса является малая информативность, которую, как было сказано ранее, невозможно доработать или, например, скорректировать под изменившиеся задачи.

Сами же установки имеют ряд ограничений эксплуатации.
Например, наплавка труднодоступных поверхностей внутри корпусов может обеспечиваться только при помощи наклона позиционера, что ведёт к нарушению технологического процесса, по которому наплавка должно производиться в нижнем, то есть горизонтальном, положении.
К тому же, многие установки не реализует функцию поперечных колебаний горелки, что необходимо для качественного формирования наплавочного слоя.

\subsection{Роботизированные комплексы}
Универсальным решением для автоматизации самых разных производственных процессов является организация роботизированного комплекса на основе шестиосевого робота-манипулятора.
Многозвенная конструкция не имеет такой жёсткости, как рассмотренные установки, но для выполнения наплавочных операций она не требуется.
Зато ставшая стандартной среди промышленных роботов-манипуляторов кинематическая схема <<Puma>> обеспечивает высокую досягаемость рабочего инструмента робота с сохранением возможности гибко изменять его ориентацию в пространстве, что было продемонстрировано в работах~\cite{Fan_Tien_Cheng_1997, Gupta_1990, Mei_2019}.
Для выполнения наплавки на изделия из широкого номенклатурного ряда, большая часть которых имеет труднодоступные области и индивидуальный технологический процесс, использование шестиосевого робота-манипулятора видится оптимальным с точки зрения удобства эксплуатации решением.

Единственным серьёзным недостатком такого решения является высокая стоимость современной роботизированной ячейки.
В неё входит не только сам робот, но и целый ряд сопутствующего оборудования, такого, как позиционер, сварочный аппарат, устройство подачи присадочного материала, дополнительные системы визуального и температурного контроля, а также программное обеспечение.


\section{Методы определение положения и ориентации изделия}
Одной из целей автоматизации производственных процессов является сведение к минимуму влияния человеческого фактора.
Однако, полностью нивелировать его практически невозможно.
Неточности возникают с момента монтажа, когда может быть допущено недостаточное выравнивание оборудования относительно уровня пола или могут присутствовать неровности самого фундамента.
Также не всегда возможно разработать такую оснастку, в которой изделие при закреплении каждый раз будет позиционироваться одинаково.

\subsection{Системы технического зрения}
Одним из способов определения положения и ориентации изделия является использование систем технического зрения.

Как правило, системы технического зрения для роботов -- это готовые интегрированные решения, предлагаемые производителями в расширенной комплектации оборудования.
С одной стороны, это гарантирует полную совместимость системы технического зрения с оборудованием комплекса.
С другой, это накладывает ограничения на применение системы технического зрения для решения других задач, что только усугубляется необходимостью в использовании проприетарного программного обеспечения.

Использование систем технического зрения также связано с необходимостью соблюдения некоторых специфичных условий для окружающей среды, таких, как правильное позиционирование камеры относительно объекта, хорошая освещённость.
При том, что стоимость систем технического зрения, способных обеспечить высокую точность определения положения и ориентации изделия, достаточно высока.

\subsection{Методы на основе касания}
При производстве свариваемых заготовок, как правило, большого размера, допускаются различные отклонения и допуски.
Для того, чтобы компенсировать низкую точность обработки изделий перед сваркой, разрабатывают алгоритмы определения положения изделия на основании касаний его рабочим инструментом.

Подобный метод использовался в работе~\cite{Fei_Gao_2015} и был направлен на определение смещений свариваемых конструкций.
Такой способ выгодно отличается от систем технического зрения тем, что он не требует дополнительного дорогостоящего оборудования при обеспечении высокой точности.

Как правило, суть таких алгоритмов заключается в медленном перемещении рабочей точки инструмента внутри области, в которой положение изделия является допустимым.
При обнаружении касания, координаты рабочей точки фиксируются и используются для дальнейшего пересчёта и определения положения изделия.
При выходе рабочей точки за пределы допустимой области, условия поиска изменяются, либо операция поиска прекращается из-за необходимости вмешательства оператора.

Именно такой подход был выбран для реализации алгоритмов определения положения и ориентации изделия в абсолютной системе координат робота.


\section{Выводы}
Одним из главных недостатков любого ручного труда является его индивидуальность, что приводит к непостоянству качества выпускаемой продукции.
Поэтому, для исключения из технологически сложного процесса наплавки человеческого фактора, снижения процента брака, уменьшения себестоимости изделий и увеличения производительности, на ПАО <<Аскольд>> было решено автоматизировать процессы наплавки.

Практика работ~\cite{Al_Sarraf_2016, Dai_2011, Gwan_Hyung_Kim, Liang_2011, Lin_2018, Xu_2008} показала, что малое отклонение сварочных параметров позволяет определить их влияние на качество сварного соединения.
Так как сварка и наплавка -- это родственные процессы, то подобным образом возможно определить степень влияния тех или иных параметров наплавки на появление брака готовой продукции.

Поэтому, основным требованием к аппаратной части комплекса для наплавки износостойких поверхностей является обеспечение достаточно высокой точности и повторяемости процесса.
Это необходимо для достижения идентичности качества при выполнении наплавки на одних и тех же режимах работы и для обеспечения возможности определения влияния малых отклонений тех или иных параметров работы на качество готовой продукции.

Как показала практика эксплуатации установок для наплавки, описанных в подразделе~\ref{subsec:WeldingMachines}, автоматизация наплавочных процессов даже на специализированном оборудовании является сложной и трудоёмкой задачей, потому как все основные параметры наплавки задаются технологическим процессом в виде диапазонов, что было продемонстрировано в таблице~\ref{tab:Introduction:WeldingParameters}.

Несмотря на то, что специализированные установки значительно дешевле, они не способны покрыть весь спектр задач наплавки износостойкими материалами.
Они хорошо подходят для типовых задач, например, для сварки труб.
Но применение их в таком технологически сложном процессе, как наплавка, когда различные изделия из широкого списка номенклатуры поступают преимущественно малыми партиями и зачастую имеют индивидуальный технологический процесс, видится малоперспективным из-за закрытого программного обеспечения и сложного процесса формирования программы работы для каждого нового изделия.

Роботизированный комплекс для наплавки на базе шестиосевого робота-манипулятора позволяет охватить весь спектр задач, связанных с наплавкой.
Возможно как совершенствование комплекса посредством внедрения дополнительного оборудования (различные измерительные системы, дополнительные системы технического зрения и т.д.), так и дальнейшее применение роботов для других задач, не связанных с технологическим процессом наплавки или сварки.

В качестве аппаратной составляющей комплекса, в работе используется предоставленный предприятием шестиосевой робот манипулятор FANUC M-16iB/20 с контроллером FANUC R-J3iB и сварочный полуавтомат PANASONIC YD-500KR.
