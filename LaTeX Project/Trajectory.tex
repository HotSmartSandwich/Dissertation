\section{Генерация траекторий}

\subsection{Параметры траектории движения} \label{subsec:TrajectoryParameters}
В технологических процессах, связанных с операциями сварки и наплавки, как правило, определяются лишь основные сварочные параметры: вид сварки, присадочный материал, режимы работы сварочного аппарата.
В случае наплавки также указывается частота и амплитуда колебаний.
Но почти никогда не даётся рекомендаций касательно выбора типа сварочных колебаний.
Выбор конкретного паттерна колебаний чаще всего остаётся непосредственно за сварщиком, что приводит к ещё большему увеличению в разбросе качества конечной продукции.

Рассматриваемые в подразделе~\ref{subsec:WeldingMachines} установки для наплавки, как правило, реализуют только зигзагообразные колебания.
Решения для роботизированной сварки/наплавки, как, например, FANUC ArcTool и KUKA.ArcTech предлагают готовый инструментарий для задания сварочной траектории с колебаниями лишь из узкого набора предустановленных паттернов с ограниченной возможностью их параметрического редактирования.

Для того чтобы проследить влияние выбранного типа колебаний и их конкретных параметров на качество конечной продукции, в рамках этой работы был разработан программный модуль, позволяющий универсальным способом задавать траекторию движения сварочной горелки с независимым наложением на неё колебаний разных типов.

\subsection{Создание траектории движения}
Траектория движения сварочной горелки строится на основании двух составляющих: путь и накладываемые на него колебания.
При создании траектории движения, сначала выбирается тип пути.
После, задаётся тип колебаний и их основные параметры, прописанные в технологическом процессе для конкретного изделия.

Так как технологический процесс разработан для наплавки вручную, то все эти параметры указываются в виде допустимого диапазона с учётом возможных неточностей в работе сварщика.
В случае роботизированной наплавки, конкретные значения параметров выбираются из диапазона оператором исходя из внешнего вида предполагаемой траектории.

Как было описано в подразделе~\ref{subsec:ProgramStructure}, на контроллер робота программа движения передаётся в виде набора инструкций и используемых в программе положений.
Таким образом, траектория движения робота представляется в виде набора точек.
Количество точек напрямую зависит от параметров конкретной траектории и должно подбираться оператором исходя из её внешнего вида.

Для более простых траекторий с небольшим итоговым числом колебаний, количество точек может быть значительно меньше, чем для более сложных траекторий.
Недостаточное количество точек может привести к значительному снижению точности движения по траектории.
Чрезмерно завышенное число точек, в свою очередь, существенно сказывается на времени компиляции и загрузки программы работы.
К тому же, более старые версии контроллеров не всегда справляются с обработкой программ работы большого объёма.
Пример внешнего вида одной и той же траектории при разном количестве точек приведён на рисунке~\ref{fig:Trajectory:LowPointTrajectoryExample}.

\begin{figure}[H]
    \centering
    \vspace{14pt}
    \includegraphics[width=\linewidth]{Figures/Trajectory/LowPointTrajectoryExample}
    \caption{Траектория при разном количестве точек}
    \label{fig:Trajectory:LowPointTrajectoryExample}
\end{figure}

Для того, чтобы иметь возможность менять количество точек в траектории, сама траектория должна описываться в параметрическом виде, где координаты каждой точки зависят от некоторого параметра $t$.
Величина шага изменения этого параметра и будет влиять на количество точек.

Как было указано в подразделе~\ref{subsec:TrajectoryParameters}, траектория движения состоит из основного пути движения и наложенных на него колебаний.
И основной путь, и колебания представляют собой набор точек.

Для примера, рассмотрим траекторию, приведённую на рисунке~\ref{fig:Trajectory:LowPointTrajectoryExample}.
Это траектория движения по окружности с круговыми колебаниями, изображёнными на рисунке~\ref{fig:Trajectory:CircleOscillation}.

\begin{figure}[H]
    \centering
    \vspace{14pt}
    \includegraphics[width=\linewidth]{Figures/Trajectory/CircleOscillation}
    \caption{Пример круговых колебаний}
    \label{fig:Trajectory:CircleOscillation}
\end{figure}

Сперва, вычисляется массив из $n$ точек, из которых состоит основная траектория:

\begin{gather}
    \label{eq:Trajectory:CirclePath}
    \begin{cases}
        \begin{aligned}
            x_p = \dfrac{D}{2} \cdot \cos t \\
            \\
            y_p = \dfrac{D}{2} \cdot \sin t
        \end{aligned}
    \end{cases} \\
    \nonumber
    0 \leq t \leq 2 \pi,
\end{gather} \\
где $D$ -- диаметр окружности основной траектории.

Для корректной генерации колебаний, необходимо вычислить общее количество колебаний, которое зависит от частоты колебаний, протяжённости выбранного пути и скорости перемещения рабочего инструмента:

\begin{equation*}
    F_N = \dfrac{2 \pi \cdot D}{\upsilon} \cdot F,
\end{equation*} \\
где $\upsilon$ -- скорость движения по траектории, $F$ -- частота колебаний.

Вычисление точек колебаний осуществляется следующим образом:

\begin{gather}
    \begin{cases}
        \label{eq:Trajectory:CircleOscillation}
        \begin{aligned}
            x_o = \dfrac{A_w}{2} \cdot \sin t \\
            \\
            y_o = \dfrac{A_h}{2} \cdot \cos t
        \end{aligned}
    \end{cases} \\
    \nonumber
    0 \leq t \leq 2 \pi \cdot F_N,
\end{gather} \\
где $A_w$ -- амплитуда колебаний по ширине, $A_h$ -- амплитуда колебаний по высоте, а общее количество точек соотвествует количеству точек основной траектории $n$.

Таким образом, точки колебаний, полученные по формуле~\ref{eq:Trajectory:CircleOscillation}, описывают одинаковые окружности, которые, при наложении на основную траекторию, образуют колебания, изображённые на рисунке~\ref{fig:Trajectory:CircleOscillation}

Для каждой точки основной траектории рассчитывается угол направления колебания:

\begin{equation}
    \label{eq:Trajectory:Angle}
    a_i = \arctan \left( \dfrac{{y_p}_{i+1} - {y_p}_i}{{x_p}_{i+1} - {x_p}_{i}} \right),
\end{equation} \\
где $i \in [1, 2, ..., n - 1]$ -- номер точки траектории.

Для каждой из $n$ точек основного пути~\ref{eq:Trajectory:CirclePath} происходит сложение вектор-столбца, составленного из координат этой точки с повёрнутым на соответствующий номеру этой точки угол~\ref{eq:Trajectory:Angle} вектор-столбцом, образованным координатами соответствующей точки колебаний~\ref{eq:Trajectory:CircleOscillation}:

\begin{gather*}
    \begin{bmatrix}
        {x_t}_i \\
        {y_t}_i
    \end{bmatrix} =
    \begin{bmatrix}
        {x_p}_i \\
        {y_p}_i
    \end{bmatrix} +
    \begin{bmatrix}
        \cos a_i & - \sin a_i \\
        \sin a_i & \cos a_i
    \end{bmatrix}
    \begin{bmatrix}
        {x_o}_i \\
        {y_o}_i
    \end{bmatrix}.
\end{gather*} \\

Таким образом происходит наложение колебаний на основной путь движения, что позволяет получить итоговую траекторию движения.
Пример наложения круговых колебаний на траектории по прямой и по окружности изображён на рисунке~\ref{fig:Trajectory:CircleOscillationOnPath}

\begin{figure}[H]
    \centering
    \vspace{14pt}
    \includegraphics[width=\linewidth]{Figures/Trajectory/CircleOscillationOnPath}
    \caption{Пример наложения круговых колебаний на траектории движения по прямой и по окружности}
    \label{fig:Trajectory:CircleOscillationOnPath}
\end{figure}

Важной особенностью разработанного алгоритма является его универсальность для разных типов колебаний.
При добавлении колебаний нового типа, достаточно описать их в параметрической форме, подобной уравнению~\ref{eq:Trajectory:CircleOscillation}.

Например, параметрическая форма для колебаний восьмёркой, представленных на рисунке~\ref{fig:Trajectory:EightOscillation}, имеет следующий вид:

\begin{gather*}
    \begin{cases}
        \begin{aligned}
            &x_o = \dfrac{\dfrac{A_w}{2} \cdot \sin t \cos t}{1 + \sin^2 t} \\
            \\
            &y_o = \dfrac{\dfrac{A_h}{2} \cdot \cos t}{1 + \sin^2 t}
        \end{aligned}
    \end{cases} \\
    0 \leq t \leq 2 \pi \cdot F_N.
\end{gather*}

\begin{figure}[H]
    \centering
    \vspace{14pt}
    \includegraphics[width=\linewidth]{Figures/Trajectory/EightOscillation}
    \caption{Пример колебаний восьмёркой}
    \label{fig:Trajectory:EightOscillation}
\end{figure}

Наложение колебаний восьмёркой на траекторию движения по прямой и по окружности изображено на рисунке~\ref{fig:Trajectory:EightOscillationOnPath}

\begin{figure}[H]
    \centering
    \vspace{14pt}
    \includegraphics[width=\linewidth]{Figures/Trajectory/EightOscillationOnPath}
    \caption{Пример наложения колебаний восьмёркой на траектории движения по прямой и по окружности}
    \label{fig:Trajectory:EightOscillationOnPath}
\end{figure}

Одним из простейших видов колебаний являются колебания зигзагом, представленные на рисунках~\ref{fig:Trajectory:ZigzagOscillation} и~\ref{fig:Trajectory:ZigzagOscillationOnPath}.
Их параметрическая форма имеет следующий вид:

\begin{gather*}
    \begin{cases}
        \begin{aligned}
            &x_o = 0 \\
            &y_o = A_h \cdot \left( 0.5 - \left| t - 2 \floor*{\dfrac{t}{2}} - 1 \right| \right)
        \end{aligned}
    \end{cases} \\
    0 \leq t \leq 2 \cdot F_N.
\end{gather*}

\begin{figure}[H]
    \centering
    \vspace{14pt}
    \includegraphics[width=\linewidth]{Figures/Trajectory/ZigzagOscillation}
    \caption{Пример колебаний зигзагом}
    \label{fig:Trajectory:ZigzagOscillation}
\end{figure}

\begin{figure}[H]
    \centering
    \vspace{14pt}
    \includegraphics[width=\linewidth]{Figures/Trajectory/ZigzagOscillationOnPath}
    \caption{Пример наложения колебаний зигзагом на траектории движения по прямой и по окружности}
    \label{fig:Trajectory:ZigzagOscillationOnPath}
\end{figure}
