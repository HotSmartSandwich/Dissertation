\chapter{Заключение}
В процессе выполнения работы были решены различные задачи, необходимые для разработки информационно-управляющей системы роботизированного комплекса, назначением которого является наплавка износостойких поверхностей на изделия судовой арматуры.

Были рассмотренны различные подходы и способы автоматизации процесса наплавки, приведены их недостатки, обосновывающие использование в рассматриваемом производственном направлении шестиосевого робота-манипулятора.

Для взаимодействия оператора с роботизированным комплексом, было разработано приложение, предоставляющее весь необходимый для работы с комплексом функционал в виде графического пользовательского интерфейса.
Разработанное приложение предназначено не только для вывода графического интерфейса, но также для коммуникации с контроллером робота, считывания его состояний, создания, компиляции и загрузки программ работы.

Для минимизации влияния человеческого фактора и выполнения наплавочных работ в полностью автоматическом режиме, был разработан метод определения положения и ориентации детали.
Разработанный метод основан на определении точек на поверхности изделия при помощи касания сварочным электродом, а потому совмещает в себе отсутствие необходимости в приобретении дополнительного дорогостоящего оборудования и высокую точность.

В результате, были проведены опытные работы по определению положения и ориентации изделия, а также опытные наплавочные работы, которые подтвердили возможность и показали высокий потенциал автоматизации такого сложного производственного направления, как наплавка износостойких поверхностей.
