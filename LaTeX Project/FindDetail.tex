\section{Определение положения и ориентации детали}
Чтобы скомпенсировать неточности в позиционировании изделия, необходимо определить его положение и ориентацию в абсолютной системе координат, связанной с основанием робота.

\subsection{Определение касания}
Для проведения опытных работ использовался сварочный аппарат Panasonic YD-500KR, представленный на рисунке~\ref{fig:FindDetail:Panasonic_YD-500KR}.

\begin{figure}[H]
    \centering
    \vspace{14pt}
    \includegraphics[height=10cm]{Figures/FindDetail/Panasonic_YD-500KR}
    \caption{Сварочный аппарат Panasonic YD-500KR}
    \label{fig:FindDetail:Panasonic_YD-500KR}
\end{figure}

Сварочный аппарат используется совместно с устройством подачи проволоки, к которому подключается сварочная горелка.
Как видно на рисунке~\ref{fig:FindDetail:Panasonic_YD-500KR}, сварочная горелка имеет эргономичную форму и не подразумевает возможности крепления к роботу.
Потому, для её крепления к фланцу робота был разработан и изготовлен специальный крепёж.
Фотография закреплённой на роботе горелки приведена на рисунке~\ref{fig:FindDetail:WeldHolder}.

\begin{figure}[H]
    \centering
    \vspace{14pt}
    \includegraphics[height=12cm]{Figures/FindDetail/WeldHolder}
    \caption{Разработанный крепёж для сварочной горелки}
    \label{fig:FindDetail:WeldHolder}
\end{figure}

Используемый сварочный аппарат предназначен для полуавтоматической сварки.
То есть, присадочный материал в виде проволоки играет роль электрода и непрерывно подаётся с катушки в сварочную ванну.
В процессе сварки, сварочная дуга образуется между присадочной проволокой и изделием, которое подключено к выходу сварочного аппарата с полярностью, обратной полярности проволоки.
При прикосновении изделия электродом, цепь сварочного источника замыкается, соответственно, напряжение в цепи падает.
Зафиксировав падение напряжения, можно определить наличие касания.

Как правило, современные сварочные аппараты, даже не предназначенные для роботизированной сварки, имеют возможность снятия показаний тока и напряжения.
Однако, сварочные аппараты Panasonic YD-500KR, выпускаемые с 2000 года, такой опции не имеют.
Поэтому, для определения касания была разработана и реализована схема подключения, изображённая на рисунке~\ref{fig:FindDetail:ElectricalCircuit}.

\begin{figure}[H]
    \centering
    \vspace{14pt}
    \includegraphics[height=9cm]{Figures/FindDetail/ElectricalCircuit}
    \caption{Электрическая схема для определения касания}
    \label{fig:FindDetail:ElectricalCircuit}
\end{figure}

На схеме E11 и E12 -- источники постоянного напряжения сварочного аппарата для напряжения сварки и для питания внутренней логики соответственно.
E2 -- источник напряжения логики контроллера робота.
Контакт Kw условно обозначает наличие на электроде сварочного напряжения, а контакт Kt -- наличие контакта между электродом и изделием.
В начальном состоянии, когда оба контакта разомкнуты, светодиод оптопары U излучает свет под действием тока, протекающего от источника E12 через резистор R14.
Благодаря этому, фототранзистор оптопары находится в открытом положении, а напряжение между его контактами остаётся пренебрежительно малым.

При контакте электрода и изделия, то есть при замыкании контакта Kt, ток источника E12 протекает через этот контакт, минуя цепи с резисторами R13 и R14.
Светодиод оптопары перестаёт светить, фототранзистор запирается, а между его контактов фиксируется напряжение +24 В, что сигнализирует о наличии касания между электродом и изделием.

\subsection{Расчёт ориентации детали} \label{subsec:DetailOrientation}
Наплавка выполняется на уплотнительную поверхность, находящуюся внутри корпуса.
Для удобства дальнейшей работы, в центре поверхности наплавки расположим связанную с ней систему координат.
Пример изделия с выделенной штриховкой поверхностью наплавки приведён на рисунке~\ref{fig:FindDetail:ValveExample}.

\begin{figure}[H]
    \centering
    \vspace{14pt}
    \includegraphics[height=9cm]{Figures/FindDetail/DetailExample}
    \caption{Модель корпуса с выделенной поверхностью наплавки}
    \label{fig:FindDetail:ValveExample}
\end{figure}

Поверхность наплавки расположена параллельно плоскости верхнего фланца.
Поэтому, для определения ориентации связанной СК достаточно определить ориентацию плоскости фланца.

Для нахождения ориентации плоскости фланца, необходимо определить три точки на его поверхности.
Однако, для компенсации возможных неточностей в определении координат точек касания, предлагается использовать большее количество опорных точек.

Поиск точек на поверхности осуществляется при помощи движения рабочего инструмента робота вертикально сверху-вниз до фиксирования касания, либо до выхода из заранее определённой области поиска.

Найденные точки удобнее всего аппроксимировать плоскостью при помощи метода наименьших квадратов.

Рассмотрим уравнение плоскости:

\begin{equation}
    \label{eq:FindDetail:Plane}
    z = A x + B y + C,
\end{equation} \\
где $x$, $y$, $z$ -- координаты точек, принадлежащих плоскости, $A$ и $B$ -- коэффициенты наклона плоскости относительно осей $X$ и $Y$ соответственно, $C$ -- вертикальное смещение плоскости по оси $Z$.

Тогда функция для нахождения ошибки $i$-го измерения будет иметь следующий вид:

\begin{equation}
    \label{eq:FindDetail:Plane_MNK_e}
    e(x_i, y_i, z_i) = A x_i + B y_i + C - z_i.
\end{equation} \\

Задачей метода наименьших квадратов является минимизация суммы квадратов ошибок, то есть следующей функции:

\begin{equation}
    \label{eq:FindDetail:Plane_MNK_S}
    S(A, B, C) = \sum_{i=1}^{n} \left[ e(x_i, y_i, z_i) \right]^2.
\end{equation} \\

Функции~\ref{eq:FindDetail:Plane_MNK_S} достигает своего минимального значение в такой точке, в которой градиент этой функции равен нулю.

\begin{gather*}
    \nabla S(A, B, C) = 0, \\
    \begin{cases}
        \begin{aligned}
            \dfrac{\partial S(A, B, C)}{\partial A} = 2 \sum_{i=1}^{n} \left[
            \dfrac{\partial e(x_i, y_i, z_i)}{\partial A} \cdot e(x_i, y_i, z_i) \right] = 0 \\
            \dfrac{\partial S(A, B, C)}{\partial B} = 2 \sum_{i=1}^{n} \left[
            \dfrac{\partial e(x_i, y_i, z_i)}{\partial B} \cdot e(x_i, y_i, z_i) \right] = 0 \\
            \dfrac{\partial S(A, B, C)}{\partial C} = 2 \sum_{i=1}^{n} \left[
            \dfrac{\partial e(x_i, y_i, z_i)}{\partial C} \cdot e(x_i, y_i, z_i) \right] = 0
        \end{aligned}
    \end{cases} =
\end{gather*}
\begin{align}
    \nonumber
    = &\begin{cases}
           \begin{aligned}
               &\sum_{i=1}^{n} \left[ x_i \left( A x_i + B y_i - C - z_i \right) \right] = 0 \\
               &\sum_{i=1}^{n} \left[ y_i \left( A x_i + B y_i - C - z_i \right) \right] = 0 \\
               &\sum_{i=1}^{n} \left[ - \left( A x_i + B y_i - C - z_i \right) \right] = 0
           \end{aligned}
    \end{cases} = \\ \nonumber
    = &\begin{cases}
           \begin{aligned}
               &\sum_{i=1}^{n} \left[ A x_i^2 + B x_i y_i - C x_i - x_i z_i \right] = 0 \\
               &\sum_{i=1}^{n} \left[ A x_i y_i + B y_i^2 - C y_i - y_i z_i \right] = 0 \\
               &\sum_{i=1}^{n} \left[ A x_i + B y_i - C - z_i \right] = 0
           \end{aligned}
    \end{cases} = \\ \nonumber
    = &\begin{cases}
           \begin{aligned}
               &A \sum_{i=1}^{n} x_i^2
               + B \sum_{i=1}^{n} x_i y_i
               - C \sum_{i=1}^{n} x_i
               - \sum_{i=1}^{n} x_i z_i = 0 \\
               &A \sum_{i=1}^{n} x_i y_i
               + B \sum_{i=1}^{n} y_i^2
               - C \sum_{i=1}^{n} y_i
               - \sum_{i=1}^{n} y_i z_i = 0 \\
               &A \sum_{i=1}^{n} x_i
               + B \sum_{i=1}^{n} y_i
               - C \cdot n - \sum_{i=1}^{n} z_i = 0
           \end{aligned}
    \end{cases} = \\
    = &\begin{cases}
           \label{eq:FindDetail:Plane_MNK_Final_System}
           \begin{aligned}
               &A \sum_{i=1}^{n} x_i^2
               + B \sum_{i=1}^{n} x_i y_i
               - C \sum_{i=1}^{n} x_i
               = \sum_{i=1}^{n} x_i z_i \\
               &A \sum_{i=1}^{n} x_i y_i
               + B \sum_{i=1}^{n} y_i^2
               - C \sum_{i=1}^{n} y_i
               = \sum_{i=1}^{n} y_i z_i \\
               &A \sum_{i=1}^{n} x_i
               + B \sum_{i=1}^{n} y_i
               - C \cdot n = \sum_{i=1}^{n} z_i
           \end{aligned}
    \end{cases}
\end{align} \\

Система~\ref{eq:FindDetail:Plane_MNK_Final_System} имеет следующий вид:

\begin{equation}
    \label{eq:FindDetail:Plane_Cramer_System}
    \begin{cases}
        \begin{aligned}
            A a_{11} + B a_{12} + C a_{13} = b_1 \\
            A a_{21} + B a_{22} + C a_{23} = b_2 \\
            A a_{31} + B a_{32} + C a_{33} = b_3
        \end{aligned}
    \end{cases},
\end{equation} \\
где $A$, $B$ и $C$ -- искомые коэффициенты уравнения плоскости~\ref{eq:FindDetail:Plane},

\begin{alignat*}{3}
    a_{11} &= \sum_{i=1}^{n} x_i^2, & \qquad
    a_{12} &= \sum_{i=1}^{n} x_i y_i, & \qquad
    a_{13} &= - \sum_{i=1}^{n} x_i, \\
    a_{21} &= \sum_{i=1}^{n} x_i y_i, &
    a_{22} &= \sum_{i=1}^{n} y_i^2, &
    a_{23} &= - \sum_{i=1}^{n} y_i, \\
    a_{31} &= \sum_{i=1}^{n} x_i, &
    a_{32} &= \sum_{i=1}^{n} y_i, &
    a_{33} &= - n.
\end{alignat*} \\

Решим систему~\ref{eq:FindDetail:Plane_Cramer_System} методом Крамера:

\begin{alignat*}{3}
    A &= \dfrac{\Delta_A}{\Delta}, & \qquad
    B &= \dfrac{\Delta_B}{\Delta}, & \qquad
    C &= \dfrac{\Delta_C}{\Delta},
\end{alignat*} \\
где

\begin{alignat*}{2}
    \Delta &= \begin{vmatrix}
                  a_{11} & a_{12} & a_{13} \\
                  a_{21} & a_{22} & a_{23} \\
                  a_{31} & a_{32} & a_{33}
    \end{vmatrix}, & \qquad
    \Delta_A &= \begin{vmatrix}
                    b_1 & a_{12} & a_{13} \\
                    b_2 & a_{22} & a_{23} \\
                    b_3 & a_{32} & a_{33}
    \end{vmatrix}, \\
    \Delta_B &= \begin{vmatrix}
                    a_{11} & b_1 & a_{13} \\
                    a_{21} & b_2 & a_{23} \\
                    a_{31} & b_3 & a_{33}
    \end{vmatrix}, & \qquad
    \Delta_C &= \begin{vmatrix}
                    a_{11} & a_{12} & b_1 \\
                    a_{21} & a_{22} & b_2 \\
                    a_{31} & a_{32} & b_3
    \end{vmatrix}.
\end{alignat*} \\

Найдя коэффициенты уравнения плоскости~\ref{eq:FindDetail:Plane}, рассчитаем углы наклона плоскости фланца и, соответственно, углы наклона связанной с поверхностью наплавки системой координат относительно осей $X$ и $Y$ абсолютной системы координат:

\begin{align*}
    \alpha &= \arctan(A), \\
    \beta &= \arctan(B).
\end{align*}

\subsection{Расчёт положения детали} \label{subsec:DetailPosition}
Расчёт положения детали основывается на определении точек, лежащих на внутренней грани входного отверстия корпуса.
Поиск точек происходит в повёрнутой системе координат, параметры которой были определены при поиске ориентации корпуса.
Таким образом, все найденные точки будут лежать в плоскости $XY$.

Так как входное отверстие имеет форму окружности, то для расчёта координат его центра достаточно трёх точек.
Но, как и при определении ориентации детали, для компенсации возможных неточностей в определении координат точек касания, предлагается использовать большее количество опорных точек.

Поиск точек осуществляется движением от произвольной точки внутри отверстия до соприкосновения с изделием или до выхода из области поиска.
Найденное произвольное количество точек аппроксимируется методом наименьших квадратов.

Уравнение окружности имеет следующий вид:

\begin{equation*}
    \left( x - x_0 \right)^2 + \left( y - y_0 \right)^2 = r^2,
\end{equation*} \\
где $x$, $y$ -- координаты точек, лежащих на окружности, $x_0$ и $y_0$ -- координаты центра окружности, $r$ - радиус окружности.

Функция ошибки $i$-го измерения будет иметь следующий вид:

\begin{equation}
    \begin{aligned}
        \label{eq:FindDetail:Circle_MNK_e}
        e(x_i, y_i) &= \left( x_i - x_0 \right)^2 + \left( y_i - y_0 \right)^2 - r^2 = \\
        &= x_i^2 + y_i^2 - 2 (x_i x_0 + y_i y_0) + x_0^2 + y_0^2 - r^2.
    \end{aligned}
\end{equation} \\

Тогда функция для минимизации квадрата ошибок будет иметь следующий вид:

\begin{equation*}
    S(x_0, y_0, r) = \sum_{i=1}^{n} \left[ e(x_i, y_i) \right]^2.
\end{equation*} \\

Задача сводится к поиску таких значений $x_0$ и $y_0$, при которой функция $S(x_0, y_0, r)$ достигает своего минимума, то есть её частные производные по $x_0$ и $y_0$ равны нулю:

\begin{align}
    \nonumber
    &\begin{cases}
         \begin{aligned}
             \dfrac{\partial S(x_0, y_0, r)}{\partial x_0} = 2 \sum_{i=1}^{n} \left[
             \dfrac{\partial e(x_i, y_i)}{\partial x_0} \cdot e(x_i, y_i) \right] = 0 \\
             \dfrac{\partial S(x_0, y_0, r)}{\partial y_0} = 2 \sum_{i=1}^{n} \left[
             \dfrac{\partial e(x_i, y_i)}{\partial y_0} \cdot e(x_i, y_i)\right] = 0 \\
             \dfrac{\partial S(x_0, y_0, r)}{\partial r} = 2 \sum_{i=1}^{n} \left[
             \dfrac{\partial e(x_i, y_i)}{\partial y_0} \cdot e(x_i, y_i)\right] = 0
         \end{aligned}
    \end{cases} = \\ \nonumber
    = &\begin{cases}
           \begin{aligned}
               &2 \sum_{i=1}^{n} \left[ (2 x_0 - 2 x_i) \cdot e(x_i, y_i) \right] = 0 \\
               &2 \sum_{i=1}^{n} \left[ (2 y_0 - 2 y_i) \cdot e(x_i, y_i) \right] = 0 \\
               &2 \sum_{i=1}^{n} \left[ 2 r \cdot e(x_i, y_i) \right] = 0
           \end{aligned}
    \end{cases} = \\ \nonumber
    = &\begin{cases}
           \begin{aligned}
               &4 \sum_{i=1}^{n} \left[ x_0 \cdot e(x_i, y_i) - x_i \cdot e(x_i, y_i) \right] = 0 \\
               &4 \sum_{i=1}^{n} \left[ y_0 \cdot e(x_i, y_i) - x_i \cdot e(x_i, y_i) \right] = 0 \\
               &4 n \cdot r \sum_{i=1}^{n} e(x_i, y_i) = 0
           \end{aligned}
    \end{cases} = \\
    = &\begin{cases}
           \label{eq:FindDetail:Circle_MNK_system}
           \begin{aligned}
               &n \cdot x_0 \sum_{i=1}^{n} e(x_i, y_i)
               - \sum_{i=1}^{n} \left[ x_i \cdot e(x_i, y_i) \right] = 0 \\
               &n \cdot y_0 \sum_{i=1}^{n} e(x_i, y_i)
               - \sum_{i=1}^{n} \left[ y_i \cdot e(x_i, y_i) \right] = 0 \\
               &\sum_{i=1}^{n} e(x_i, y_i) = 0
           \end{aligned}
    \end{cases}
\end{align} \\

Так как минимальным значением функции $S(x_0, y_0, r)$ является ноль, то:

\begin{equation}
    \label{eq:FindDetail:Circle_MNK_e_is_zero}
    \sum_{i=1}^{n} e(x_i, y_i) = 0.
\end{equation} \\

Подставив~\ref{eq:FindDetail:Circle_MNK_e_is_zero} в первые два уравнения системы~\ref{eq:FindDetail:Circle_MNK_system}, получим:

\begin{equation*}
    \begin{cases}
        \begin{aligned}
            &\sum_{i=1}^{n} \left[ x_i \cdot e(x_i, y_i) \right] = 0 \\
            &\sum_{i=1}^{n} \left[ y_i \cdot e(x_i, y_i) \right] = 0 \\
            &\sum_{i=1}^{n} e(x_i, y_i) = 0
        \end{aligned}
    \end{cases}
\end{equation*} \\

Раскрыв $e(x_i, y_i)$ в соответствии с~\ref{eq:FindDetail:Circle_MNK_e} и преобразовав систему уравнений, получим:

\begin{align}
    \nonumber
    &\begin{cases}
         \begin{aligned}
             &\sum_{i=1}^{n} \left[ x_i \left(
             x_i^2 + y_i^2 - 2 (x_i x_0 + y_i y_0) + x_0^2 + y_0^2 - r^2 \right) \right] = 0 \\
             &\sum_{i=1}^{n} \left[ y_i \left(
             x_i^2 + y_i^2 - 2 (x_i x_0 + y_i y_0) + x_0^2 + y_0^2 - r^2 \right) \right] = 0 \\
             &\sum_{i=1}^{n} \left[
             x_i^2 + y_i^2 - 2 (x_i x_0 + y_i y_0) + x_0^2 + y_0^2 - r^2 \right] = 0
         \end{aligned}
    \end{cases} = \\ \nonumber
    = &\begin{cases}
           \begin{aligned}
               &\sum_{i=1}^{n} \left[
               x_i^3 + x_i y_i^2 - 2 x_i (x_i x_0 + y_i y_0)
               + x_i \left( x_0^2 + y_0^2 - r^2 \right) \right] = 0 \\
               &\sum_{i=1}^{n} \left[
               x_i^2 y_i + y_i^3 - 2 y_i (x_i x_0 + y_i y_0)
               + y_i \left( x_0^2 + y_0^2 - r^2 \right) \right] = 0 \\
               &\sum_{i=1}^{n} x_i^2 + \sum_{i=1}^{n} y_i^2
               - 2 \left( x_0 \sum_{i=1}^{n} x_i + y_0 \sum_{i=1}^{n} y_i \right)
               + n \left( x_0^2 + y_0^2 - r^2 \right) = 0
           \end{aligned}
    \end{cases} = \\
    = &\begin{cases}
           \label{eq:FindDetail:Circle_MNK_full_system}
           \begin{aligned}
               &\sum_{i=1}^{n} x_i^3 + \sum_{i=1}^{n} x_i y_i^2
               - 2 \sum_{i=1}^{n} x_i \left( x_0 \sum_{i=1}^{n} x_i + y_0 \sum_{i=1}^{n} y_i \right) + \\
               &\qquad + \sum_{i=1}^{n} x_i \left( x_0^2 + y_0^2 - r^2 \right) = 0 \\
               \\
               &\sum_{i=1}^{n} x_i^2 y_i + \sum_{i=1}^{n} y_i^3
               - 2 \sum_{i=1}^{n} y_i \left( x_0 \sum_{i=1}^{n} x_i + y_0 \sum_{i=1}^{n} y_i \right) + \\
               &\qquad + \sum_{i=1}^{n} y_i \left( x_0^2 + y_0^2 - r^2 \right) = 0 \\
               \\
               &x_0^2 + y_0^2 - r^2 = - \dfrac{1}{n} \left(
               \sum_{i=1}^{n} x_i^2 + \sum_{i=1}^{n} y_i^2
               - 2 \left( x_0 \sum_{i=1}^{n} x_i
               + y_0 \sum_{i=1}^{n} y_i \right)\right)
           \end{aligned}
    \end{cases}
\end{align} \\

Подставим значение выражения $x_0^2 + y_0^2 - r^2$ из третьего уравнения системы~\ref{eq:FindDetail:Circle_MNK_full_system} в её первые два уравнения:

\begin{align}
    \nonumber
    &\begin{cases}
         \begin{aligned}
             &\sum_{i=1}^{n} x_i^3 + \sum_{i=1}^{n} x_i y_i^2 - 2 \left(
             x_0 \sum_{i=1}^{n} x_i^2 + y_0 \sum_{i=1}^{n} x_i y_i \right) - \\
             &\qquad - \dfrac{1}{n} \sum_{i=1}^{n} x_i \left(
             \sum_{i=1}^{n} x_i^2 + \sum_{i=1}^{n} y_i^2
             - 2 \left( x_0 \sum_{i=1}^{n} x_i + y_0 \sum_{i=1}^{n} y_i \right)\right) = 0 \\
             \\
             &\sum_{i=1}^{n} x_i^2 y_i + \sum_{i=1}^{n} y_i^3 - 2 \left(
             x_0 \sum_{i=1}^{n} x_i y_i + y_0 \sum_{i=1}^{n} y_i^2 \right) - \\
             &\qquad - \dfrac{1}{n} \sum_{i=1}^{n} y_i \left(
             \sum_{i=1}^{n} x_i^2 + \sum_{i=1}^{n} y_i^2
             - 2 \left( x_0 \sum_{i=1}^{n} x_i + y_0 \sum_{i=1}^{n} y_i \right)\right) = 0
         \end{aligned}
    \end{cases} = \\ \nonumber
    = &\begin{cases}
           \begin{aligned}
               &2 x_0 \left( \dfrac{1}{n} \sum_{i=1}^{n} x_i \sum_{i=1}^{n} x_i - \sum_{i=1}^{n} x_i^2 \right)
               + 2 y_0 \left( \dfrac{1}{n} \sum_{i=1}^{n} x_i \sum_{i=1}^{n} y_i
               - \sum_{i=1}^{n} x_i y_i \right) + \\
               &\qquad + \sum_{i=1}^{n} x_i^3 + \sum_{i=1}^{n} x_i y_i^2
               - \dfrac{1}{n} \sum_{i=1}^{n} x_i
               \left( \sum_{i=1}^{n} x_i^2 + \sum_{i=1}^{n} y_i^2 \right) = 0 \\
               \\
               &2 x_0 \left( \dfrac{1}{n} \sum_{i=1}^{n} x_i \sum_{i=1}^{n} y_i
               - \sum_{i=1}^{n} x_i y_i \right)
               + 2 y_0 \left( \dfrac{1}{n} \sum_{i=1}^{n} x_i \sum_{i=1}^{n} x_i - \sum_{i=1}^{n} y_i^2 \right) + \\
               &\qquad + \sum_{i=1}^{n} x_i^2 y_i + \sum_{i=1}^{n} y_i^3
               - \dfrac{1}{n} \sum_{i=1}^{n} y_i
               \left(\sum_{i=1}^{n} x_i^2 + \sum_{i=1}^{n} y_i^2 \right) = 0
           \end{aligned}
    \end{cases} = \\
    = &\begin{cases}
           \label{eq:FindDetail:Circle_MNK_final_system}
           \begin{aligned}
               &2 x_0 \left( \sum_{i=1}^{n} x_i^2 - \dfrac{1}{n} \sum_{i=1}^{n} x_i \sum_{i=1}^{n} x_i \right)
               + 2 y_0 \left( \sum_{i=1}^{n} x_i y_i
               - \dfrac{1}{n} \sum_{i=1}^{n} x_i \sum_{i=1}^{n} y_i \right) = \\
               &\qquad = \sum_{i=1}^{n} x_i^3 + \sum_{i=1}^{n} x_i y_i^2
               - \dfrac{1}{n} \sum_{i=1}^{n} x_i \left( \sum_{i=1}^{n} x_i^2 + \sum_{i=1}^{n} y_i^2 \right) \\
               \\
               &2 x_0 \left( \sum_{i=1}^{n} x_i y_i
               - \dfrac{1}{n} \sum_{i=1}^{n} x_i \sum_{i=1}^{n} y_i \right)
               + 2 y_0 \left( \sum_{i=1}^{n} y_i^2 - \dfrac{1}{n} \sum_{i=1}^{n} x_i \sum_{i=1}^{n} x_i \right) = \\
               &\qquad = \sum_{i=1}^{n} x_i^2 y_i + \sum_{i=1}^{n} y_i^3
               - \dfrac{1}{n} \sum_{i=1}^{n} y_i \left( \sum_{i=1}^{n} x_i^2 + \sum_{i=1}^{n} y_i^2 \right)
           \end{aligned}
    \end{cases}
\end{align} \\

Нетрудно заметить, что система~\ref{eq:FindDetail:Circle_MNK_final_system} имеет следующий вид:

\begin{equation}
    \label{eq:FindDetail:SLAU_system}
    \begin{cases}
        \begin{aligned}
            x_0 a_{11} + y_0 a_{12} = b_1
            \\
            x_0 a_{21} + y_0 a_{22} = b_2
        \end{aligned}
    \end{cases},
\end{equation} \\
где

\begin{alignat*}{2}
    a_{11} &= 2 \left( \sum_{i=1}^{n} x_i^2 - \dfrac{1}{n} \sum_{i=1}^{n} x_i \sum_{i=1}^{n} x_i \right), & \qquad
    a_{12} &= 2 \left( \sum_{i=1}^{n} x_i y_i - \dfrac{1}{n} \sum_{i=1}^{n} x_i \sum_{i=1}^{n} y_i \right), \\
    a_{21} &= 2 \left( \sum_{i=1}^{n} x_i y_i - \dfrac{1}{n} \sum_{i=1}^{n} x_i \sum_{i=1}^{n} y_i \right), &
    a_{22} &= 2 \left( \sum_{i=1}^{n} y_i^2 - \dfrac{1}{n} \sum_{i=1}^{n} x_i \sum_{i=1}^{n} x_i \right),
\end{alignat*} \\

\begin{align*}
    b_1 &= \sum_{i=1}^{n} x_i^3 + \sum_{i=1}^{n} x_i y_i^2
    - \dfrac{1}{n} \sum_{i=1}^{n} x_i \left(
    \sum_{i=1}^{n} x_i^2 + \sum_{i=1}^{n} y_i^2 \right), \\
    b_2 &= \sum_{i=1}^{n} x_i^2 y_i + \sum_{i=1}^{n} y_i^3
    - \dfrac{1}{n} \sum_{i=1}^{n} y_i \left(
    \sum_{i=1}^{n} x_i^2 + \sum_{i=1}^{n} y_i^2 \right).
\end{align*} \\

Система~\ref{eq:FindDetail:SLAU_system} является нормальной формой системы двух линейных алгебраических уравнений с двумя неизвестными и имеет единственное решение.
Найдём его метод Крамера:

\begin{equation*}
    x_0 = \dfrac{\Delta_{x_0}}{\Delta}, \qquad
    y_0 = \dfrac{\Delta_{y_0}}{\Delta},
\end{equation*} \\
где

\begin{align*}
    \Delta = \begin{vmatrix}
                 a_{11} & a_{12} \\
                 a_{21} & a_{22}
    \end{vmatrix} &= a_{11} a_{22} - a_{12} a_{21}, \\
    \Delta_{x_0} = \begin{vmatrix}
                       b_1 & a_{12} \\
                       b_2 & a_{22}
    \end{vmatrix} &= a_{22} b_1 - a_{12} b_2, \\
    \Delta_{y_0} = \begin{vmatrix}
                       a_{11} & b_1 \\
                       a_{21} & b_2
    \end{vmatrix} &= a_{11} b_2 - a_{21} b_1.
\end{align*} \\

Зная положение входного отверстия изделия, его ориентацию и высоту от фланца до поверхности наплавки, можно точно восстановить положение и ориентацию связанной с поверхностью наплавки системой координат, в которой будут проводиться все наплавочные работы.
