\chapter{Опытные работы}
Для проверки возможности и целесообразности выполнения роботизированной наплавки, был проведён ряд опытных работ.


\section{Ощупывание}
Одной из задач проведённых опытных работ было подтверждение или опровержение гипотезы о возможности достаточно точного определения положения и ориентации изделия при помощи нахождения точек на поверхности детали касанием электрода.
Работа проводилась по следующему сценарию:

\begin{enumerate}
    \item оператором пропускается небольшое количество сварочной проволоки;
    \item проволока обрезается таким образом, чтобы вылет проволоки составлял ~2-3 сантиметра;
    \item при помощи определения касания в определённом зафиксированном месте на сварочном столе, определяется длина вылета проволоки;
    \item определяются точки, лежащие на поверхности фланца корпуса;
    \item рассчитывается ориентация изделия;
    \item определяются точки, лежащие на окружности, образующей входное отверстие;
    \item рассчитывается положение изделия и поверхности наплавки.
\end{enumerate}

В результате эксперимента было обнаружено, что при контакте с изделием проволока может сильно деформироваться, что существенно сказывается на точности определения координат точек.

Неточности были скомпенсированы изменением подхода к поиску касания.
Первичное приближение находится с использованием прежних режимов работы.
После чего, происходит повторный поиск касания в суженной области поиска со значительным снижением скорости перемещения электрода.
Такой подход вкупе с применением метода наименьших квадратов, описанным в главах~\ref{sec:DetailOrientation} и~\ref{sec:DetailPosition}, позволил добиться приемлемой для рассматриваемого производственного направления точности.


\section{Наплавка}
Также были проведены опытные работы по наплавке с целью проверки работоспособности сварочного аппарата и подбора режимов работы.

На рисунке~\ref{fig:Experiment:Welding1} изображён результат первых нескольких проходов без колебаний.
Швы тонкие, пористые, присутствуют несплавления.

\begin{figure}[H]
    \centering
    \vspace{14pt}
    \includegraphics[height=10cm]{Figures/Experiment/Welding1}
    \caption{Линейные проходы}
    \label{fig:Experiment:Welding1}
\end{figure}

Швы получились неудовлетворительного качества из-а неправильно подобранных режимов работы.
Общее качества шва существенно повышалось при увеличении сварочного тока и уменьшении скорости перемещения, что в итоге позволило добиться приемлемого качества.

Вторым этапом работ стала наплавка одного валика с круговыми колебаниями.
Траектория движения горелки представлена на рисунке~\ref{fig:Experiment:Freq}.

\begin{figure}[H]
    \centering
    \vspace{14pt}
    \includegraphics[height=10cm]{Figures/Experiment/Trajectory}
    \caption{Траектория движения рабочего инструмента при наплавке одного валика}
    \label{fig:Experiment:Freq}
\end{figure}

Фотография результата наплавки приведена на рисунке~\ref{fig:Experiment:Welding2}.

\begin{figure}[H]
    \centering
    \vspace{14pt}
    \includegraphics[height=10cm]{Figures/Experiment/Welding2}
    \caption{Первый кольцевой проход с колебаниями}
    \label{fig:Experiment:Welding2}
\end{figure}

Как видно из рисунка~\ref{fig:Experiment:Welding2}, наплавленный валик имеет большое количество пор, образовавшихся из-за слишком большой амплитуды колебаний и слишком высокой скорости перемещения горелки.

С откорректированными параметрами была проведена повторная наплавка валика.
Фотография результата наплавки приведена на рисунке~\ref{fig:Experiment:Welding3}.

\begin{figure}[H]
    \centering
    \vspace{14pt}
    \includegraphics[height=10cm]{Figures/Experiment/Welding3}
    \caption{Второй кольцевой проход с колебаниями}
    \label{fig:Experiment:Welding3}
\end{figure}

На рисунке~\ref{fig:Experiment:Welding3} видно, что поверхность, наплавленная с откорректированными параметрами, не имеет пор и несплавлений.
Валик получился однородным, состоящим из <<чешуек>>, что также косвено указывает на приемлемое качество.
