%TODO: Переписать главу Опытные работы
\chapter{Опытные работы}
В заключение, для проверки возможности и целесообразности выполнения роботизированной наплавки был проведён ряд опытных работ.


\section{Ощупывание}
Одной из основных задач проведённых опытных работ было подтверждение или опровержение гипотезы о возможности достаточно точно определять положение и ориентацию изделия при помощи определения касаний электродом.
Работа проводилась по следующему сценарию:

\begin{enumerate}
    \item Оператором пропускается небольшое количество проволоки, проволока обрезается.
    \item При помощи определения касания в определённом зафиксированном месте на сварочном столе определяется длина вылета проволоки.
    \item Определяются три точки, принадлежащие поверхности фланца корпуса.
    \item Рассчитывается ориентация изделия.
    \item Определяются три точки, образующие окружность входного отверстия.
    \item Рассчитывается положение изделия и поверхности наплавки.
\end{enumerate}

В результате эксперимента были обнаружены некоторые недостатки такого подхода.
Например, при касании изделия проволока может деформироваться, что существенно снижает точность измерений.
Это может быть частично скомпенсировано уменьшением вылета проволоки и выбором меньшей скорости перемещения горелки при поиске касания.


\section{Наплавка}
Также были проведены работы по наплавке, целью которых являлись проверка работоспособности сварочного аппарата и подбор режимов работы.

На рисунке~\ref{fig:Experiment:Surf1} изображён результат первых нескольких проходов без колебаний.
Швы тонкие, пористые, присутствуют несплавления.

\begin{figure}[H]
    \centering
    \vspace{14pt}
    \includegraphics[height=10cm]{Figures/Experiment/Surf1}
    \caption{Линейные проходы}
    \label{fig:Experiment:Surf1}
\end{figure}

Швы получились неудовлетворительного качества из-а неправильно подобранных режимов работы.
Общее качества шва существенно повышалось при увеличении сварочного тока и уменьшении скорости перемещения, что в итоге позволило добиться приемлемого качества.

Вторым этапом работ стала наплавка одного валика с круговыми колебаниями.
Траектория движения горелки представлена на рисунке~\ref{fig:Experiment:Freq}.

\begin{figure}[H]
    \centering
    \vspace{14pt}
    \includegraphics[height=10cm]{Figures/Experiment/Trajectory}
    \caption{Траектория движения рабочего инструмента при наплавке одного валика}
    \label{fig:Experiment:Freq}
\end{figure}

Фотография результата наплавки приведена на рисунке~\ref{fig:Experiment:Surf2}.

\begin{figure}[H]
    \centering
    \vspace{14pt}
    \includegraphics[height=10cm]{Figures/Experiment/Surf2}
    \caption{Первый кольцевой проход с колебаниями}
    \label{fig:Experiment:Surf2}
\end{figure}

Как видно из рисунка~\ref{fig:Experiment:Surf2}, наплавленный валик имеет большое количество пор, образовавшихся из-за слишком большой амплитуды колебаний и слишком высокой скорости перемещения горелки.

С откорректированными параметрами была проведена повторная наплавка валика.
Фотография результата наплавки приведена на рисунке~\ref{fig:Experiment:Surf3}.

\begin{figure}[H]
    \centering
    \vspace{14pt}
    \includegraphics[height=10cm]{Figures/Experiment/Surf3}
    \caption{Второй кольцевой проход с колебаниями}
    \label{fig:Experiment:Surf3}
\end{figure}

На рисунке~\ref{fig:Experiment:Surf3} видно, что поверхность, наплавленная с откорректированными параметрами, не имеет пор и несплавлений.
Валик получился состоящим из однородных <<чешуек>>, что также косвено указывает на высокое качество.