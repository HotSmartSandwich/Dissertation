\documentclass[utf8x]{G7-32}  % Стиль (по умолчанию будет 14pt)

% Общие настройки
% Настройки стиля ГОСТ 7-32
% Для начала определяем, хотим мы или нет, чтобы рисунки и таблицы нумеровались в пределах раздела, или нам нужна сквозная нумерация.
\EqInChapter % формулы будут нумероваться в пределах раздела
\TableInChapter % таблицы будут нумероваться в пределах раздела
\PicInChapter % рисунки будут нумероваться в пределах раздела

\setcounter{page}{1} % нумерация страниц начиная с n-ой

\usepackage{color}

\usepackage{cite}

% Математический шрифт в том числе для стрелочек векторов
\usepackage{libertine}
\usepackage[varvw]{newtxmath}
\usepackage{amsmath}
%\usepackage{newtxmath}

% Добавляем гипертекстовое оглавление в PDF
\usepackage[
    bookmarks=true, colorlinks=true, unicode=true,
    urlcolor=black,linkcolor=black, anchorcolor=black,
    citecolor=black, menucolor=black, filecolor=black,
]{hyperref}

% Изменение начертания шрифта --- после чего выглядит таймсоподобно.
% apt-get install scalable-cyrfonts-tex

% Настройка шрифта
\usepackage{pscyr}
\renewcommand{\rmdefault}{ftm} % Times New Roman

% Чтобы было красиво
\usepackage{microtype}

% Отключение переносов и текст по ширине
\usepackage{ragged2e}
\tolerance=1000
\hyphenpenalty=10000
\emergencystretch=3em

\usepackage{float}

% non-italic в формулах
\usepackage{mathtools}

% вставка титульного листа
\usepackage{pdfpages}

\usepackage{graphicx}   % Пакет для включения рисунков
\graphicspath{{./Figures/}}

% Поля страницы
\usepackage{geometry}
\geometry{left=3cm}
\geometry{right=1.5cm}
\geometry{top=1.5cm}
\geometry{bottom=2.4cm}

% Произвольная нумерация списков и многоуровневые нумерованные списки
\usepackage{enumerate}
\renewcommand{\labelenumi}{\arabic{enumi}.}
\renewcommand{\labelenumii}{\arabic{enumi}.\arabic{enumii}.}

% ячейки в несколько строчек
\usepackage{multirow}

% itemize внутри tabular
\usepackage{paralist, array}

% Перенос на новую страницу многострочных формул
\allowdisplaybreaks

% Отображение скобок для floor - округление к меньшему
\DeclarePairedDelimiter\floor{\lfloor}{\rfloor}

% Настройки листингов
\usepackage{listings}

% Значения по умолчанию
\lstset{
    basicstyle=\footnotesize\ttfamily,
    breakatwhitespace=true,% разрыв строк только на whitespacce
    breaklines=true,       % переносить длинные строки
    inputencoding=koi8-r,
    numbers=left,          % нумерация слева
    numberstyle=\footnotesize\ttfamily,
    showspaces=false,      % показывать пробелы подчеркиваниями -- идиотизм 70-х годов
    showstringspaces=false,
    showtabs=false,        % и табы тоже
    stepnumber=1,
    tabsize=4,             % кому нужны табы по 8 символов?
    frame=single
}

% Подписи к листингам на русском языке.
\renewcommand\lstlistingname{\cyr\CYRL\cyri\cyrs\cyrt\cyri\cyrn\cyrg}
\renewcommand\lstlistlistingname{\cyr\CYRL\cyri\cyrs\cyrt\cyri\cyrn\cyrg\cyri}


% JSON style from https://gist.github.com/ed-cooper/1927af4ccac39b083440d436d018d253
\usepackage{xcolor}
\definecolor{delim}{RGB}{20,105,176}
\definecolor{numb}{RGB}{106, 109, 32}
\definecolor{string}{rgb}{0.64,0.08,0.08}

\lstdefinelanguage{json}{
    basicstyle=\footnotesize\ttfamily,
    morestring=[b]",
    stringstyle=\color{string},
    literate=
    *{0}{{{\color{numb}0}}}{1}
        {1}{{{\color{numb}1}}}{1}
        {2}{{{\color{numb}2}}}{1}
        {3}{{{\color{numb}3}}}{1}
        {4}{{{\color{numb}4}}}{1}
        {5}{{{\color{numb}5}}}{1}
        {6}{{{\color{numb}6}}}{1}
        {7}{{{\color{numb}7}}}{1}
        {8}{{{\color{numb}8}}}{1}
        {9}{{{\color{numb}9}}}{1}
        {\{}{{{\color{delim}{\{}}}}{1}
        {\}}{{{\color{delim}{\}}}}}{1}
        {[}{{{\color{delim}{[}}}}{1}
        {]}{{{\color{delim}{]}}}}{1},
}


\begin{document}
    \includepdf{TitlePage}  % Титульный лист

    \frontmatter  % Отключение нумерации
    \renewcommand\thefigure{\arabic{figure}}
    \renewcommand\thetable{\arabic{table}}

    \tableofcontents  % Содержание
    \chapter{Аннотация}
В работе описана информационно-управляющая система для роботизированного комплекса.
Назначение комплекса -- автоматизация процессов наплавки износостойкими материалами на уплотнительные поверхности судовой арматуры.

Для работы в автоматическом режиме и сведения влияния человеческого фактора к минимуму, были разработаны методы определения положения и ориентации установленной детали.

Также был разработан метод формирования сварочных траекторий посредством универсального объединения траектории движения с колебаниями любого типа.

Было разработано пользовательское приложение, осуществляющее коммуникацию с контроллером робота и предоставляющее оператору комплекса понятный и функциональный графический интерфейс для параметрирования процесса.
  % Аннотация
    \chapter{Введение}
Износ рабочих поверхностей подвижных деталей происходит почти во всех областях промышленности~\cite{Oo_2018}.

На рисунке~\ref{fig:Problem:StopValve} изображён запорный клапан в разрезе, предназначенный для перекрытия потока рабочей среды.

\begin{figure}[H]
    \centering
    \vspace{14pt}
    \includegraphics[width=\linewidth]{Figures/Problem/StopValve}
    \caption{Запорный клапан в разрезе}
    \label{fig:Problem:StopValve}
\end{figure}

Выделенные на рисунке~\ref{fig:Problem:StopValve} области называются уплотнительными поверхностями и, как правило, подвержены повышенному износу по причине возникновения трения <<металл по металлу>>.

На арматуростроительных предприятиях, таких как ПАО <<Аскольд>> (г. Арсеньев), АО <<Армалит>> (г. Санкт-Петербург), ОАО <<Завод <<Буревестник>>>> (г. Гатчина), а также других профильных предприятиях, для увеличения срока службы трубопроводной арматуры выполняются технологические процессы наплавки уплотнительных поверхностей износостойкими материалами.

Под наплавкой подразумеваются операции нанесения уплотнительных металлических поверхностей в установленных областях изготавливаемых деталей, осуществляемых посредством сварочного оборудования.

На ПАО <<Аскольд>> наплавка таких поверхностей осуществляется по технологии ручной дуговой наплавки неплавящимся вольфрамовым электродом в среде защитного инертного газа.
Наплавка осуществляется в несколько слоёв.
Каждый слой, в свою очередь, состоит из нескольких концентрических окружностей, называемых валиками.
Отдельные валики наплавляются последовательно, с перерывами.
Это необходимо для того, чтобы температура изделия не превышала температуру, допустимую технологическим процессом.
После наплавки каждого валика проводится зачистка поверхности от шлака и брызг металла, а также осуществляется визуальный осмотр качества наплавки.
На рисунке~\ref{fig:Problem:WeldingDetail} приведён пример изделия в процессе выполнения наплавки.

\begin{figure}[H]
    \centering
    \vspace{14pt}
    \includegraphics[width=\linewidth]{Figures/Problem/WeldingDetail}
    \caption{Изделие во время выполнения наплавки}
    \label{fig:Problem:WeldingDetail}
\end{figure}

На рисунке~\ref{fig:Problem:WeldingDetail} слева изображена деталь сразу после наплавки второго валика до зачистки поверхности.
Видно, что наплавленный слой подвержен включениям шлака тёмного цвета.
Справа изображена эта же деталь после наплавки третьего валика и зачистки поверхности металлической щёткой: поверхность чистая, имеет металлический блеск.
В таком виде детали направляются на дальнейшую механическую обработку.

Для рассматриваемого производственного направления характерен высокий процент неизбежно-технологического брака, закладываемого в стоимость продукции.
Основной причиной брака является человеческий фактор: операции наплавки крайне требовательны к строгому соблюдению технологического процесса.

Операции наплавки производятся согласно технологическому процессу, в котором описываются как все необходимые для изделия подготовительные процедуры (требования к поверхности, температурный режим), так и основные параметры работы непосредственно при наплавке.
Значения этих параметров приведены в таблице~\ref{tab:WeldingParameters}.

\begin{table}[H]
    \footnotesize
    \caption{Режимы работы для операций наплавки}
    \label{tab:WeldingParameters}
    \begin{tabular}{|p{0.09\linewidth}|p{0.1\linewidth}|p{0.11\linewidth}|p{0.1\linewidth}|p{0.13\linewidth}|p{0.09\linewidth}|p{0.07\linewidth}|p{0.1\linewidth}|}
        \hline
        Условный диаметр арматуры, мм & Сварочный ток, А & Напряжение на дуге, В & Амплитуда колебаний, мм & Ширина наплавляемых валиков, мм & Скорость наплавки, м/ч & Расход аргона, л/мин & Частота колебаний, кол/мин \\ \hline
        50--65                        & 140--160         & 26--28                & 5--10                   & 8--14                           & 10--12                 & 18--25               & 30--50                     \\ \hline
        80                            & 140--160         & 26--28                & 6--12                   & 10--16                          & 10--12                 & 18--25               & 20--40                     \\ \hline
        100--150                      & 150--170         & 27--29                & 8--14                   & 12--18                          & 9--11                  & 18--25               & 20--40                     \\ \hline
        200+                          & 160--180         & 28--30                & 10--16                  & 14--20                          & 8--10                  & 18--25               & 20--30                     \\ \hline
    \end{tabular}
\end{table}

Как видно из таблицы~\ref{tab:WeldingParameters}, в технологическом процессе указываются не конкретные значения параметров, а их достаточно грубые диапазоны.
Конкретные значения из этих диапазонов выбираются непосредственно сварщиком.
И если такие параметры, как сварочные ток и напряжение, расход защитного газа можно выставить и поддерживать достаточно точно, то такие параметры, как скорость перемещения горелки, частота и амплитуда колебаний при выполнении наплавки вручную возможно соблюдать лишь приблизительно.
Работа каждого сварщика индивидуальна и может отличаться от изделия к изделию.

Кроме того, к наплавке износостойкими материалами допускаются только высококвалифицированные электрогазосварщики 5-го и 6-го разрядов, что приводит к потребности в узкоспециализированных и высокооплачиваемых кадрах.

Цель работы заключается в разработке информационно-управляющей системы для роботизированного комплекса наплавки износостойких поверхностей изделий судовой арматуры.

Роботизированный комплекс с информационно-управляющей системой должен удовлетворять следующим требованиям:

\begin{itemize}
	\item обеспечивать выполнение технологических процессов наплавки в полностью автоматическом режиме;
	\item участие оператора должно сводиться к предварительной подготовке роботизированного комплекса к работе, установке и креплению изделия на рабочей поверхности, выбор технологического процесса при помощи панели оператора и запуск наплавки в автоматическом режиме с панели оператора;
	\item обеспечивать заданные конструкторской и нормативно-технической документацией физико-механические и химические свойства наплавляемых поверхностей, а также их качество, отсутствие дефектов или превышений ими допустимых норм;
	\item использовать наплавку аргонодуговым методом в качестве основной технологии;
	\item автоматически определять положение установленного изделия в пространстве, тем самым компенсируя неточности в его установке оператором;
	\item отображать основные показатели текущего состояния системы и процесса;
	\item хранить основные технологические характеристики процесса наплавки как отдельные программы работы с возможностью их редактирования и повторного использования.
\end{itemize}

В соответствии с поставленной целью, были сформулированы следующие задачи:

\begin{itemize}
	\item определить состав аппаратной части роботизированного комплекса;
	\item разработать метод определения положения и ориентации установленного изделия в пространстве;
	\item определить способ генерации программ работы для робота;
	\item разработать методы коммуникации информационно-управляющей системы с контроллером робота;
	\item разработать пользовательский интерфейс для оператора роботизированного комплекса.
\end{itemize}
  % Введение

    \mainmatter  % Начало нумерации
    \renewcommand\thefigure{\thechapter.\arabic{figure}}
    \renewcommand\thetable{\thechapter.\arabic{table}}

    \chapter{Обзор существующих решений}
Ручные сварка и наплавка -- это достаточно сложные операции, требующие от сварщика высокой квалификации и опыта~\cite{Hattori_1965, Seregina_2018}.
Это косвено подтверждается большим количество научных работ, так или иначе связанных с облегчением труда сварщиков и усовершенствованием технологического процесса.


\section{Способы корректирования ручной работы}
В работе~\cite{Aiteanu_2003} описывается разработанная сварочная маска, предназначенная для повышения удобства ручных операций сварки и наплавки за счёт улучшения обзора рабочей области и контроля качества в режиме реального времени при помощи технологий дополненной реальности.
Несмотря на технологичность такого решения, оно малоприменимо в условиях реального производства: опытный сварщик способен без труда поддерживать правильное положение сварочной горелки.
Потому такое решение оказывается совершенно неоправданным из-за своей дороговизны и малой практической пользы.

Более применимым на практике видится подход, описанный в работе~\cite{Muller_2018}.
В ней технологический процесс аргонодуговой сварки неплавящимся электродом представлен в виде объекта управления.
Цель работы заключается в корректировке операций сварки, выполняемых вручную, при помощи механизированной регулировки длины дуги.

Похожий подход к улучшению качества процесса за счёт корректировки его параметров был описан в работах~\cite{Yang_2020, Dai_2011, Xu_2008}.
Значимым отличием является то, что регулирование в этих работах осуществляется не за счёт перемещения сварочной горелки и изменения длины дуги, а за счёт изменения параметров работы сварочного аппарата.

Как описанная в работе~\cite{Muller_2018} стабилизирующая система, так и описанные в работах~\cite{Yang_2020, Dai_2011, Xu_2008} методы регулировки параметров, действительно позволяют добиться улучшения качества процесса, но все эти разработки не исключают из сварочных операций влияние человеческого фактора, который и является основной причиной брака.
Для этого необходимо полностью автоматизировать процесс.


\section{Автоматизация процессов наплавки}

\subsection{Установки для наплавки} \label{subsec:WeldingMachines}
Одним из наиболее популярных способов автоматизации процессов наплавки является использование специализированных наплавочных установок.
Эксплуатация таких установок описана в работах~\cite{Baskoro_2016, Deyong_You_2014, Jafari_2010, Su_2010, Yi_Jinggang_2010}.
Как правило, установки для наплавки состоят из закреплённой на трёхосевой каретке сварочной горелки и двухосевого позиционера.
В работе~\cite{Qi_2019} рассматривается сварочная машина для сварки швов большого диаметра с аналогичным способом крепления рабочего инструмента.

Каретка регулирует расстояние рабочего инструмента до поверхности наплавки и осуществляет его перемещение в горизонтальной плоскости.
Позиционер отвечает за вращение изделия, за счёт чего формируется наплавляемый слой.
Пример описываемой установки приведён на рисунке~\ref{fig:Overview:WeldingMachine}.

\begin{figure}[H]
    \centering
    \vspace{14pt}
    \includegraphics[height=15cm]{Figures/SolutionsOverview/WeldingMachine}
    \caption{Установка для наплавки}
    \label{fig:Overview:WeldingMachine}
\end{figure}

Главным преимуществом таких установок является сравнительно невысокая стоимость при высокой жёсткости конструкции, которая, в свою очередь, позволяет достичь достаточную для рассматриваемого производственного направления повторяемость процесса.
Однако, такие установки имеют ряд значительных недостатков.

На рисунке~\ref{fig:Overview:ProgramEditor} изображено окно редактирования программы работы для одной из таких установок.
Несмотря на кажущуюся простоту, разработка программы работы для каждого нового изделия является крайне сложным и продолжительным процессом, заключающемся в эмпирическом подборе скорости наплавки и смещения горелки для каждого из проходов.

\begin{figure}[H]
    \centering
    \vspace{14pt}
    \includegraphics[width=\linewidth]{Figures/SolutionsOverview/ProgramEditor}
    \caption{Интерфейс редактирования программы работы}
    \label{fig:Overview:ProgramEditor}
\end{figure}

Отладочные работы должны производиться оператором-сварщиком высокой квалификации, который во время пробных наплавок должен непрерывно производить визуальный контроль сварочной ванны, подбирать положение и высоту горелки, корректировать значения тока и напряжения.
Чтобы разработать программу работы для нового изделия даже опытному оператору-сварщику необходимо произвести не менее десяти пробных наплавок.

Другим важным ограничением является проприетарность программного обеспечения.
Система управления, разрабатываемая поставщиком оборудования, как правило, выполняется на базе программируемого логического контроллера и не подразумевает изменения или дальнейшей доработки.
На рисунке~\ref{fig:Overview:ProgramInterface} представлен основной экран пользовательского интерфейса.

\begin{figure}[H]
    \centering
    \vspace{14pt}
    \includegraphics[width=\linewidth]{Figures/SolutionsOverview/ProgramInterface}
    \caption{Основное окно интерфейса}
    \label{fig:Overview:ProgramInterface}
\end{figure}

Главным недостатком изображённого на рисунке~\ref{fig:Overview:ProgramInterface} пользовательского интерфейса является малая информативность, которую, как было сказано ранее, невозможно доработать или, например, скорректировать под изменившиеся задачи.

Сами же установки имеют ряд ограничений эксплуатации.
Например, наплавка труднодоступных поверхностей внутри корпусов может обеспечиваться только при помощи наклона позиционера, что ведёт к нарушению технологического процесса, по которому наплавка должно производиться в нижнем, то есть горизонтальном, положении.
К тому же, многие установки не реализует функцию поперечных колебаний горелки, что необходимо для качественного формирования наплавочного слоя.

\subsection{Роботизированные комплексы}
Универсальным решением для автоматизации самых разных производственных процессов является организация роботизированного комплекса на основе шестиосевого робота-манипулятора.
Многозвенная конструкция не имеет такой жёсткости, как рассмотренные установки, но для выполнения наплавочных операций она не требуется.
Зато ставшая стандартной среди промышленных роботов-манипуляторов кинематическая схема <<Puma>> обеспечивает высокую досягаемость рабочего инструмента робота с сохранением возможности гибко изменять его ориентацию в пространстве, что было продемонстрировано в работах~\cite{Fan_Tien_Cheng_1997, Gupta_1990, Mei_2019}.
Для выполнения наплавки на изделия из широкого номенклатурного ряда, большая часть которых имеет труднодоступные области и индивидуальный технологический процесс, использование шестиосевого робота-манипулятора видится оптимальным с точки зрения удобства эксплуатации решением.

Единственным серьёзным недостатком такого решения является высокая стоимость современной роботизированной ячейки.
В неё входит не только сам робот, но и целый ряд сопутствующего оборудования, такого, как позиционер, сварочный аппарат, устройство подачи присадочного материала, дополнительные системы визуального и температурного контроля, а также программное обеспечение.


\section{Методы определение положения и ориентации изделия}
Одной из целей автоматизации производственных процессов является сведение к минимуму влияния человеческого фактора.
Однако, полностью нивелировать его практически невозможно.
Неточности возникают с момента монтажа, когда может быть допущено недостаточное выравнивание оборудования относительно уровня пола или могут присутствовать неровности самого фундамента.
Также не всегда возможно разработать такую оснастку, в которой изделие при закреплении каждый раз будет позиционироваться одинаково.

\subsection{Системы технического зрения}
Одним из способов определения положения и ориентации изделия является использование систем технического зрения.

Как правило, системы технического зрения для роботов -- это готовые интегрированные решения, предлагаемые производителями в расширенной комплектации оборудования.
С одной стороны, это гарантирует полную совместимость системы технического зрения с оборудованием комплекса.
С другой, это накладывает ограничения на применение системы технического зрения для решения других задач, что только усугубляется необходимостью в использовании проприетарного программного обеспечения.

Использование систем технического зрения также связано с необходимостью соблюдения некоторых специфичных условий для окружающей среды, таких, как правильное позиционирование камеры относительно объекта, хорошая освещённость.
При том, что стоимость систем технического зрения, способных обеспечить высокую точность определения положения и ориентации изделия, достаточно высока.

\subsection{Методы на основе касания}
При производстве свариваемых заготовок, как правило, большого размера, допускаются различные отклонения и допуски.
Для того, чтобы компенсировать низкую точность обработки изделий перед сваркой, разрабатывают алгоритмы определения положения изделия на основании касаний его рабочим инструментом.

Подобный метод использовался в работе~\cite{Fei_Gao_2015} и был направлен на определение смещений свариваемых конструкций.
Такой способ выгодно отличается от систем технического зрения тем, что он не требует дополнительного дорогостоящего оборудования при обеспечении высокой точности.

Как правило, суть таких алгоритмов заключается в медленном перемещении рабочей точки инструмента внутри области, в которой положение изделия является допустимым.
При обнаружении касания, координаты рабочей точки фиксируются и используются для дальнейшего пересчёта и определения положения изделия.
При выходе рабочей точки за пределы допустимой области, условия поиска изменяются, либо операция поиска прекращается из-за необходимости вмешательства оператора.

Именно такой подход был выбран для реализации алгоритмов определения положения и ориентации изделия в абсолютной системе координат робота.


\section{Выводы}
Одним из главных недостатков любого ручного труда является его индивидуальность, что приводит к непостоянству качества выпускаемой продукции.
Поэтому, для исключения из технологически сложного процесса наплавки человеческого фактора, снижения процента брака, уменьшения себестоимости изделий и увеличения производительности, на ПАО <<Аскольд>> было решено автоматизировать процессы наплавки.

Практика работ~\cite{Al_Sarraf_2016, Dai_2011, Gwan_Hyung_Kim, Liang_2011, Lin_2018, Xu_2008} показала, что малое отклонение сварочных параметров позволяет определить их влияние на качество сварного соединения.
Так как сварка и наплавка -- это родственные процессы, то подобным образом возможно определить степень влияния тех или иных параметров наплавки на появление брака готовой продукции.

Поэтому, основным требованием к аппаратной части комплекса для наплавки износостойких поверхностей является обеспечение достаточно высокой точности и повторяемости процесса.
Это необходимо для достижения идентичности качества при выполнении наплавки на одних и тех же режимах работы и для обеспечения возможности определения влияния малых отклонений тех или иных параметров работы на качество готовой продукции.

Как показала практика эксплуатации установок для наплавки, описанных в подразделе~\ref{subsec:WeldingMachines}, автоматизация наплавочных процессов даже на специализированном оборудовании является сложной и трудоёмкой задачей, потому как все основные параметры наплавки задаются технологическим процессом в виде диапазонов, что было продемонстрировано в таблице~\ref{tab:Introduction:WeldingParameters}.

Несмотря на то, что специализированные установки значительно дешевле, они не способны покрыть весь спектр задач наплавки износостойкими материалами.
Они хорошо подходят для типовых задач, например, для сварки труб.
Но применение их в таком технологически сложном процессе, как наплавка, когда различные изделия из широкого списка номенклатуры поступают преимущественно малыми партиями и зачастую имеют индивидуальный технологический процесс, видится малоперспективным из-за закрытого программного обеспечения и сложного процесса формирования программы работы для каждого нового изделия.

Роботизированный комплекс для наплавки на базе шестиосевого робота-манипулятора позволяет охватить весь спектр задач, связанных с наплавкой.
Возможно как совершенствование комплекса посредством внедрения дополнительного оборудования (различные измерительные системы, дополнительные системы технического зрения и т.д.), так и дальнейшее применение роботов для других задач, не связанных с технологическим процессом наплавки или сварки.

В качестве аппаратной составляющей комплекса, в работе используется предоставленный предприятием шестиосевой робот манипулятор FANUC M-16iB/20 с контроллером FANUC R-J3iB и сварочный полуавтомат PANASONIC YD-500KR.
  % Обзор существующих решений
    \chapter{Информационно-управляющая система}
В ходе выполнения работы, были рассмотрены различные варианты управления промышленным роботом-манипулятором.
В связи с проприетарностью встроенного программного обеспечения, при разработке информационно-управляющей системы оставалось только использовать ограниченный функционал, предлагаемый производителем оборудования.
Задача сводилась к разработке наиболее удобного способа формирования траектории движения рабочего инструмента.


\section{Программа работы робота}

\subsection{Основные способы создания программ}
Базовым и наиболее распространённым в промышленности способом является программирование напрямую через пульт управления (Teach Pendant).
Оператор вручную задаёт координаты каждой точки и ориентацию рабочего инструмента робота, либо сохраняет их в режиме обучения.
Главным преимуществом такого способа является простота и относительная быстрота написания простых программ работы.
Однако, он категорически не подходит для задания более сложных траекторий, какой и является траектория наплавки.

Для написания программ, описывающих сложные траектории движения существуют различные CAD/CAM системы, единственным значимым недостатком которых является высокая стоимость и строго ограниченный инструментарий.
Сами же сгенерированные траектории движения представляют из себя массивы из множества точек.

Изучив возможность самостоятельного написания и загрузки программ работы робота в контроллер, было принято решение спроектировать и разработать собственное приложение, включающее в себя пользовательский графический интерфейс для создания и редактирования траекторий наплавки и постпроцессор -- программное обеспечение, позволяющее преобразовать заданную оператором траекторию в программу работы для контроллера робота.
Похожий подход был применён в работе~\cite{Nagata_2017}, но в ней рассматривается исключительно интерполяция уже готового набора данных дугами.

\subsection{Структура программ работы} \label{subsec:ProgramStructure}
Основные расширения файлов программ, с которыми работают контроллеры FANUC -- это <<.LS>> и <<.TP>>.

Файл с расширением <<.LS>> -- это листинг программы, текстовый файл с ASCII кодировкой, для открытия и редактирования которого может быть использован любой текстовый редактор.
Представляет из себя два основных блока: блок движений <</MN>>, описывающий все действия робота в виде инструкций, и блок позиций <</POS>>, в котором находится информация о координатах всех точек, используемых в программе.
Пример оформления и синтаксиса приведён в листинге~\ref{lst:LS-Example}.

\begin{lstlisting}[caption={Пример оформления .LS файлов}, label={lst:LS-Example}]
	/PROG EXAMPLE.LS

	/ATTR

	/MN
	:J P[1] 10% FINE;
	:  WAIT 1.5(sec);
	:L P[2] 250mm/sec CNT50;
	:L P[3] 4sec CNT100;
	:J P[1] 5% FINE;

	/POS
	P[1]{
		GP1:
		UF: 1, UT: 1,	CONFIG: 'N U T, 0, 0, 0',
		X = 729.8 mm,	Y = 134.9 mm,	Z = 181.1 mm,
		W = 191.2 deg,	P = 13.3 deg,	R = 14.2 deg
	};

	P[2]{
		GP1:
		UF: 1, UT: 1,	CONFIG: 'N U T, 0, 0, 0',
		X = 693.6 mm,	Y = 695.6 mm,	Z = 0.0 mm,
		W = 120.0 deg,	P = -9.9 deg,	R = -46.8 deg
	};

	P[3]{
		GP1:
		UF: 1, UT: 1,	CONFIG: 'N U T, 0, 0, 0',
		X = 512.9 mm,	Y = 129.8 mm,	Z = -31.5 mm,
		W = 191.5 deg,	P = 43.1 deg,	R = 45.1 deg
	};

	/END
\end{lstlisting}

Файл с расширением <<.TP>> -- это скомпилированная программа в виде байт-кода.
Старые контроллеры FANUC не имеют пакета <<ASCII Upload>>, позволяющего компилировать файлы с расширением <<.LS>> при их загрузке в контроллер.
Поэтому они должны быть скомпилированы предварительно.
Компиляция выполняется при помощи пакета программ FANUC WinOLPC .

\subsection{Конфигурация робота}
Для роботов FANUC целевая точка может задаваться двумя способами.

Первый способ -- через задание углов поворота каждого сочленения робота ($J_1$, $J_2$, $J_3$, $J_4$, $J_5$, $J_6$).
Пример задания целевой точки этим способом приведён в листинге~\ref{lst:J16-TargetPoint}.

\begin{lstlisting}[caption={Задание целевой точки с помощью углов J1-J6}, label={lst:J16-TargetPoint}]
	P[1]{
		UF : 1, UT : 1,
		J1 =   49.399 deg,    J2 =   10.072 deg,    J3 =  -24.105 deg,
		J4 =    -.000 deg,    J5 =  -65.895 deg,    J6 =   40.601 deg
	};
\end{lstlisting}

Второй способ -- через шесть координат, три из которых определяют положение целевой точки (X, Y, Z), а другие три -- ориентацию рабочего инструмента в абсолютной системе координат, связанной с основанием робота (W, P, R).
Пример задания целевой точки приведён в листинге~\ref{lst:XYZWPR-TargetPoint}.

\begin{lstlisting}[caption={Задание целевой точки с помощью координат XYZWPR}, label={lst:XYZWPR-TargetPoint}]
	P[1]{
		UF : 1, UT : 1,        CONFIG : 'N U T, 0, 0, 0',
		X =   750.000  mm,    Y =   800.000  mm,    Z =   225.000  mm,
		W =   180.000 deg,    P =     0.000 deg,    R =    90.000 deg
	};
\end{lstlisting}

Этот способ задания является наиболее удобным и информативным для оператора, так как ссылается на понятные человеку пространственные координаты.
Однако, он не является однозначным: в одной и той же точке, с одной и той же ориентацией рабочего инструмента, робот может находиться несколькими способами, пример чего приведён на рисунке~\ref{fig:RobotConfiguration}.

\begin{figure}[H]
    \centering
    \vspace{14pt}
    \includegraphics[height=10cm]{Figures/Software/RobotConfiguration}
    \caption{Разная конфигурация робота в одной и той же целевой точке}
    \label{fig:RobotConfiguration}
\end{figure}

Конкретная конфигурация робота задаётся для каждой точки отдельно при помощи параметра <<CONFIG>> (листинг~\ref{lst:XYZWPR-TargetPoint}).

Параметр <<CONFIG>> для роботов FANUC с кинематической схемой <<Puma>> состоит из трёх буквенных и трёх численных обозначений.
Буквенные обозначения позволяют однозначно определить положение звеньев робота.
Расшифровка буквенных обозначений приведена в таблице~\ref{tab:RobotConfig}.

\begin{longtable}[H]{|p{0.25\linewidth}|p{0.25\linewidth}|p{0.4\linewidth}|}
    \caption{Расшифровка буквенных обозначений}
    \label{tab:RobotConfig} \\
    \hline
    Обозначение & Расшифровка   & Значение                \\ \hline
    N           & wrist No-flip & запястье не перевёрнуто \\ \hline
    F           & wrist Flip    & запястье перевёрнуто    \\ \hline
    U           & elbow Up      & локоть вверх            \\ \hline
    D           & elbow Down    & локоть вниз             \\ \hline
    T           & config Top    & подход спереди          \\ \hline
    B           & config Bottom & подход сзади            \\ \hline
\end{longtable}

Большая часть приведённых конфигураций является экзотической.
Как правило, в работе используется конфигурация <<NUT>> и в ходе выполнения программы она не изменяется.

Численные обозначения указывают на количество полных оборотов сверх необходимого угла поворота для многооборотных сочленений.
Как правило, количество полных оборотов необходимо учитывать при задании таких траекторий, когда происходит перекручивание запястья робота.

При работе со сварочной горелкой, кабель которой закреплён не в полом запястье робота, а снаружи, такие перекручивания недопустимы, потому в дальнейшей работе конфигурация <<N U T, 0, 0, 0>> принимается постоянной и неизменяемой.


\section{Траектория движения}

\subsection{Параметры траектории движения} \label{subsec:TrajectoryParameters}
В технологических процессах, связанных с операциями сварки и наплавки, как правило, определяются лишь основные сварочные параметры: вид сварки, присадочный материал, режимы работы сварочного аппарата.
В случае наплавки также указывается частота и амплитуда колебаний.
Но почти никогда не даётся рекомендаций касательно выбора типа сварочных колебаний.
Выбор конкретного паттерна колебаний чаще всего остаётся непосредственно за сварщиком, что приводит к ещё большему увеличению в разбросе качества конечной продукции.

Рассматриваемые в главе~\ref{ch:SolutionsOverview} установки для наплавки, как правило, реализуют только зигзагообразные колебания.
Решения для роботизированной сварки/наплавки, как, например, FANUC ArcTool и KUKA.ArcTech предлагают готовый инструментарий для задания сварочной траектории с колебаниями лишь из узкого набора предустановленных паттернов с ограниченной возможностью их параметрического редактирования.

Для того чтобы проследить влияние выбранного типа колебаний и их конкретных параметров на качество конечной продукции, в рамках этой работы был разработан программный модуль, позволяющий универсальным способом задавать траекторию движения сварочной горелки с независимым наложением на неё колебаний разных типов.

Траектория движения сварочной горели строится на основании двух составляющих: путь и накладываемые на него колебания.

Было реализовано два типа путей:

\begin{itemize}
    \item по окружности, с указанием диаметра окружности, начального и конечного углов;
    \item по прямой, с указанием начальной и конечной точек.
\end{itemize}

Также были разработаны различные типы колебаний:

\begin{itemize}
    \item круговые;
    \item зигзагообразные;
    \item синусоидальные;
    \item колебания восьмёркой.
\end{itemize}

\subsection{Создание траектории движения}
При создания траектории движения, сначала выбирается тип пути.
После, задаётся тип колебаний и их основные параметры, прописанные в технологическом процессе для конкретного изделия.

Так как технологический процесс разработан для наплавки вручную, то все эти параметры указываются в виде допустимого диапазона с учётом возможных неточностей в работе сварщика.
В случае роботизированной наплавки, конкретные значения параметров выбираются из диапазона оператором исходя из внешнего вида предполагаемой траектории.

Как было описано в разделе~\ref{subsec:ProgramStructure}, на контроллер робота программа работы передаётся в виде набора инструкций и используемых в программе положений.
Таким образом, траектория движения робота представляет собой набор точек.

Количество точек напрямую зависит от параметров конкретной траектории и должно подбираться оператором исходя из её внешнего вида.
Для более простых траекторий с небольшим итоговым числом колебаний, количество точек может быть значительно меньше, чем для более сложных траекторий.
Недостаточное количество точек может привести к значительному снижению точности движения по траектории.
Чрезмерно завышенное число точек, в свою очередь, существенно сказывается на времени компиляции и загрузки программы работы.
К тому же, более старые версии контроллеров не всегда справляются с обработкой программ работы большого объёма.
Пример внешнего вида одной и той же траектории при разном количестве точек приведён на рисунке~\ref{fig:LowPointTrajectoryExample}.

\begin{figure}[H]
    \centering
    \vspace{14pt}
    \includegraphics[width=\linewidth]{Figures/Software/LowPointTrajectoryExample}
    \caption{Траектория при разном количестве точек}
    \label{fig:LowPointTrajectoryExample}
\end{figure}

Для того, чтобы иметь возможность менять количество точек в траектории, сама траектория должна описываться в параметрическом виде, где координаты каждой точки зависят от некоторого параметра $t$.
Величина шага изменения этого параметра и будет влиять на количество точек.

Как было указано в разделе~\ref{subsec:TrajectoryParameters}, траектория движения состоит из основного пути движения и наложенных на него колебаний.

И основной путь, и колебания представляют собой набор точек.
Наложение колебаний осуществляется за счёт сложения координат точек основного пути движения с координатами соответствующих точек колебаний.

Для примера, рассмотрим траекторию, приведённую на рисунке~\ref{fig:LowPointTrajectoryExample}.
Это траектория движения по окружности с круговые колебания.

Сперва, вычисляется массив точек, из которых состоит основная траектория:

\begin{gather*}
    \begin{bmatrix}
        x_p \\
        y_p
    \end{bmatrix}
    = \dfrac{D}{2} \cdot
    \begin{bmatrix}
        \cos{t} \\
        \sin{t}
    \end{bmatrix}, \\
    0 \leq t \leq 2 \pi,
\end{gather*} \\
где $D$ -- диаметр окружности.

Для корректной генерации колебаний, для начала необходимо вычислить их количество для выбранного пути и соответствующих параметров скорости перемещения и частоты:

\begin{equation*}
	F_N = \dfrac{2 \pi \cdot D}{S} \cdot F,
\end{equation*} \\
где $F_N$ -- количество колебаний, $S$ -- скорость движения по траектории, $F$ -- частота колебаний.



\begin{gather*}
    \begin{bmatrix}
        x_o \\
        y_o
    \end{bmatrix}
    = \dfrac{D}{2} \cdot
    \begin{bmatrix}
        \cos{t} \\
        \sin{t}
    \end{bmatrix}, \\
    0 \leq t \leq 2 \pi,
\end{gather*} \\



\begin{gather*}
    \begin{bmatrix}
        x \\
        y
    \end{bmatrix} = radius \cdot
    \begin{bmatrix}
        \cos(t) \\
        \sin(t)
    \end{bmatrix}
    +
    \begin{bmatrix}
        \cos(t) & -\sin(t) \\
        \sin(t) & \cos(t)
    \end{bmatrix}
    \times
    \begin{bmatrix}
        AmplitudeHeight \cdot \sin(n \cdot t) \\
        AmplitudeWidth \cdot \cos(n \cdot t)
    \end{bmatrix}, \\
    n = \pi \cdot diameter \cdot \dfrac{WeldingFrequency}{WeldingSpeed},
\end{gather*} \\
где $radius$  и $diameter$ -- радиус и диаметр окружности, \\
$AmplitudeHeight$ и $AmplitudeWidth$ -- амплитуды колебаний, \\
$n$ -- количество колебаний, \\
$WeldingFrequency$ -- частота колебаний, \\
$WeldingSpeed$ -- скорость движения рабочего инструмента.


%Отличительной особенностью разработанного редактора явлется простота и универсальность разработки типов колебаний и их независимое от внешнего вида траектория движения редактирование.


\section{Программное обеспечение}
На рисунке~\ref{fig:SystemSchema} изображена функциональная схема разработанной информационно-управляющей системы.

\begin{figure}[H]
    \centering
    \vspace{14pt}
    \includegraphics[width=\linewidth]{Figures/Software/SystemSchema}
    \caption{Функциональная схема разработанной информационно-управляющей системы}
    \label{fig:SystemSchema}
\end{figure}

Как видно из рисунка~\ref{fig:SystemSchema}, система разделена на подсистемы, каждая из которых отвечает за выполнение конкретной задачи.

\subsection{Подсистема коммуникации}
Подсистема коммуникации отвечает за установку соединения с контроллером робота.
Она предоставляет интерфейсы для других подсистем, позволяющие получать данные о состоянии робота и его текущем положении, считывать и записывать значения внутренних переменных контроллера робота.

Также подсистема коммуникации отвечает за загрузку скомпилированных программ работы робота по протоколу FTP. Во время передачи контроллер робота выступает в роли сервера, а подсистема коммуникации в роли клиента.

\subsection{Подсистема управления}
Для функционирования разработанной системы, на контроллере робота резервируются внутренние переменные.
При помощи считывания и записи значений этих переменных, осуществляется коммуникация между контроллером робота и системой.
Отдельные переменные отводятся под разрешение на выполнение работы, разрешение на зажигание сварочной дуги, записи текущего состояния робота и так далее.

За считывание и запись этих переменных отвечает подсистема управления, использующая соответствующий интерфейс обмена данными, предоставляемый подсистемой коммуникации.

Также подсистема управления инициирует начало процесса генерации управляющих программ и их загрузку на контроллер робота.

\subsection{Подсистема мониторинга}
Подсистема мониторинга в режиме реального времени отслеживает изменения положения робота и его состояний при помощи соответствующего интерфейса, предоставляемого подсистемой коммуникации.

Все изменения передаются в подсистему отображения для информирования оператора.

\subsection{Подсистема отображения}
Подсистема отображения отвечает за предоставление оператору полнофункционального пользовательского интерфейса.
В интерфейсе отображаются состояние робота и его текущее положение, параметры работы.

При помощи пользовательского интерфейса оператор имеет возможность редактировать конфигурации программ работы робота и траектории движения сварочной горелки.
Любые внесённые оператором изменения передаются в подсистему конфигурации, где значения параметров проходят проверку на корректность и сохраняются.
При изменении параметров траектории движения, её вид обновляется в пользовательском интерфейсе в режиме реального времени.

\subsection{Подсистема конфигурации}
Подсистема конфигурации отвечает за сохранение новых конфигураций и управление существующими конфигурациями программ работы робота.

Конфигурация представляет собой набор значений всех параметров, полностью описывающий рабочую программу робота.

Все конфигурации хранятся в базе конфигураций и загружаются по запросу от оператора, для чего подсистема конфигурации предоставляет интерфейс для подсистемы отображения.

\subsection{Постпроцессор}

  % Программное обеспечение
    % TODO Описать функцинирование
    % Загружается main, в котором ожидается выбор программы от управляющего устройства.
    % На УУ оператор выбирает программу.
    % УУ формирует программу работы, компилирует, загружает на контроллер
    % УУ изменяет переменную на номер программы и ожидает изменение статуса
    % Контроллер запускает выбранную программу
    % Если запуск программы прошёл успешно, переменная СТАТУС переводится в значение BUSY
    % УУ ожидает завершения выполнения программы
    % Если запуск не успешен, переменная СТАТУС переводится в значение ERROR
    % Если программа была успешно закончена, СТАТУС в DONE
    % Если выполнялась программа с определением касания, то УУ считывает с контроллера определённые регистры, в которых записаны найденные координаты
    %
    % TODO карта переменных
    % R1
    %   Разрешение движения
    % R2
    %   Разрешение на сварку
    % R3
    %   Выбор запускаемой программы

    \chapter{Программы работы робота}
В ходе выполнения работы, были рассмотрены различные варианты управления промышленным роботом-манипулятором.
В связи с проприетарностью встроенного программного обеспечения, при разработке информационно-управляющей системы было возможно использовать только ограниченный функционал, предлагаемый производителем оборудования.


\section{Основные способы создания программ}
Базовым и наиболее распространённым в промышленности способом является программирование напрямую через пульт управления (Teach Pendant).
Оператор вручную задаёт координаты каждой точки и ориентацию рабочего инструмента робота, либо сохраняет их в режиме обучения.
Главным преимуществом такого подхода является простота и относительно высокая скорость написания простых программ работы.
Однако, он категорически не подходит для задания более сложных траекторий, какой и является траектория наплавки.

Для написания программ, описывающих сложные траектории движения существуют различные CAD/CAM системы, единственным значимым недостатком которых является высокая стоимость и строго ограниченный инструментарий.
Сами же сгенерированные траектории движения представляют из себя массивы из множества точек.

Изучив возможность самостоятельного написания и загрузки программ работы робота в контроллер, было принято решение спроектировать и разработать собственное приложение, включающее в себя пользовательский графический интерфейс для создания и редактирования траекторий наплавки и постпроцессор -- программное обеспечение, позволяющее преобразовать заданную оператором траекторию в программу работы для контроллера робота.
Похожий подход был применён в работе~\cite{Nagata_2017}, но в ней рассматривается исключительно интерполяция уже готового набора данных дугами.


\section{Структура программ работы} \label{sec:ProgramStructure}
Основные расширения файлов программ, с которыми работают контроллеры FANUC -- это <<.LS>> и <<.TP>>.

Файл с расширением <<.LS>> -- это листинг программы, текстовый файл с ASCII кодировкой, для открытия и редактирования которого может быть использован любой текстовый редактор.
Представляет из себя два основных блока: блок движений <</MN>>, описывающий все действия робота в виде инструкций, и блок позиций <</POS>>, в котором находится информация о координатах всех точек, используемых в программе.
Пример оформления и синтаксиса приведён в листинге~\ref{lst:LS-Example}.

\begin{lstlisting}[caption={Пример оформления .LS файлов}, label={lst:LS-Example}]
	/PROG EXAMPLE.LS

	/ATTR

	/MN
	:J P[1] 10% FINE;
	:  WAIT 1.5(sec);
	:L P[2] 250mm/sec CNT50;
	:L P[3] 4sec CNT100;
	:J P[1] 5% FINE;

	/POS
	P[1]{
		GP1:
		UF: 1, UT: 1,	CONFIG: 'N U T, 0, 0, 0',
		X = 729.8 mm,	Y = 134.9 mm,	Z = 181.1 mm,
		W = 191.2 deg,	P = 13.3 deg,	R = 14.2 deg
	};

	P[2]{
		GP1:
		UF: 1, UT: 1,	CONFIG: 'N U T, 0, 0, 0',
		X = 693.6 mm,	Y = 695.6 mm,	Z = 0.0 mm,
		W = 120.0 deg,	P = -9.9 deg,	R = -46.8 deg
	};

	P[3]{
		GP1:
		UF: 1, UT: 1,	CONFIG: 'N U T, 0, 0, 0',
		X = 512.9 mm,	Y = 129.8 mm,	Z = -31.5 mm,
		W = 191.5 deg,	P = 43.1 deg,	R = 45.1 deg
	};

	/END
\end{lstlisting}

Файл с расширением <<.TP>> -- это скомпилированная программа в виде байт-кода.
Старые контроллеры FANUC не имеют пакета расширения <<ASCII Upload>>, позволяющего компилировать файлы с расширением <<.LS>> при их загрузке в контроллер.
Поэтому они должны быть скомпилированы предварительно.
Компиляция выполняется при помощи пакета программ FANUC WinOLPC .


\section{Конфигурация робота} \label{sec:RobotConfiguration}
Для роботов FANUC целевая точка может задаваться двумя способами.

Первый способ -- через задание углов поворота каждого сочленения робота ($J_1$, $J_2$, $J_3$, $J_4$, $J_5$, $J_6$).
Пример задания целевой точки этим способом приведён в листинге~\ref{lst:J16-TargetPoint}.

\begin{lstlisting}[caption={Задание целевой точки с помощью углов J1-J6}, label={lst:J16-TargetPoint}]
	P[1]{
		UF : 1, UT : 1,
		J1 =   49.399 deg,    J2 =   10.072 deg,    J3 =  -24.105 deg,
		J4 =    -.000 deg,    J5 =  -65.895 deg,    J6 =   40.601 deg
	};
\end{lstlisting}

Второй способ -- через шесть координат, три из которых определяют положение целевой точки (X, Y, Z), а другие три -- ориентацию рабочего инструмента в абсолютной системе координат, связанной с основанием робота (W, P, R).
Пример задания целевой точки приведён в листинге~\ref{lst:XYZWPR-TargetPoint}.

\begin{lstlisting}[caption={Задание целевой точки с помощью координат XYZWPR}, label={lst:XYZWPR-TargetPoint}]
	P[1]{
		UF : 1, UT : 1,        CONFIG : 'N U T, 0, 0, 0',
		X =   750.000  mm,    Y =   800.000  mm,    Z =   225.000  mm,
		W =   180.000 deg,    P =     0.000 deg,    R =    90.000 deg
	};
\end{lstlisting}

Этот способ задания является наиболее удобным и информативным для оператора, так как ссылается на понятные человеку пространственные координаты.
Однако, он не является однозначным: в одной и той же точке, с одной и той же ориентацией рабочего инструмента, робот может находиться несколькими способами, пример чего приведён на рисунке~\ref{fig:RobotConfiguration}.

\begin{figure}[H]
    \centering
    \vspace{14pt}
    \includegraphics[width=\linewidth]{Figures/Software/RobotConfiguration}
    \caption{Разная конфигурация робота в одной и той же целевой точке}
    \label{fig:RobotConfiguration}
\end{figure}

Конкретная конфигурация робота задаётся для каждой точки отдельно при помощи параметра <<CONFIG>> (листинг~\ref{lst:XYZWPR-TargetPoint}).

Параметр <<CONFIG>> для роботов FANUC с кинематической схемой <<Puma>> состоит из трёх буквенных и трёх численных обозначений.
Буквенные обозначения позволяют однозначно определить положение звеньев робота.
Расшифровка буквенных обозначений приведена в таблице~\ref{tab:RobotConfig}.

\begin{table}[H]
    \caption{Расшифровка буквенных обозначений}
    \label{tab:RobotConfig}
    \begin{tabular}{|p{0.26\linewidth}|p{0.26\linewidth}|p{0.4\linewidth}|}
        \hline
        Обозначение & Расшифровка   & Значение                \\ \hline
        N           & wrist No-flip & запястье не перевёрнуто \\ \hline
        F           & wrist Flip    & запястье перевёрнуто    \\ \hline
        U           & elbow Up      & локоть вверх            \\ \hline
        D           & elbow Down    & локоть вниз             \\ \hline
        T           & config Top    & подход спереди          \\ \hline
        B           & config Bottom & подход сзади            \\ \hline
    \end{tabular}
\end{table}

Большая часть приведённых конфигураций является экзотической.
Как правило, в работе используется конфигурация <<NUT>> и в ходе выполнения программы она не изменяется.

Численные обозначения указывают на количество полных оборотов сверх необходимого угла поворота для многооборотных сочленений.
Как правило, количество полных оборотов необходимо учитывать при задании таких траекторий, когда происходит перекручивание запястья робота.

При работе со сварочной горелкой, кабель которой закреплён не в полом запястье робота, а снаружи, такие перекручивания недопустимы, потому в дальнейшей работе конфигурация <<N U T, 0, 0, 0>> принимается постоянной и неизменяемой.

    \chapter{Определение положения и ориентации детали}
Одной из целей автоматизации производственных процессов является сведение к минимуму влияния человеческого фактора.
Однако, полностью нивелировать его практически невозможно.
Неточности возникают с момента монтажа, когда может быть допущено недостаточное выравнивание оборудования относительно уровня пола или могут присутствовать неровности самого фундамента.
Также не всегда возможно разработать такую оснастку, в которой изделие при закреплении каждый раз будет позиционироваться одинаково.

Чтобы скомпенсировать неточности в позиционировании изделия, необходимо определить его положение и ориентацию в абсолютной системе координат, связанной с основанием робота.


\section{Существующие методы определение положения и ориентации}

\subsection{Системы технического зрения}
Одним из способов определения положения и ориентации изделия является использование систем технического зрения.

Как правило, системы технического зрения для роботов -- это готовые интегрированные решения, предлагаемые производителями в расширенной комплектации оборудования.
С одной стороны, это гарантирует полную совместимость системы технического зрения с оборудованием комплекса.
С другой, это накладывает ограничения на применение системы технического зрения для решения других задач, что только усугубляется необходимостью в использовании проприетарного программного обеспечения.

Использование систем технического зрения также связано с необходимостью соблюдения некоторых специфичных условий для окружающей среды, таких, как правильное позиционирование камеры относительно объекта, хорошая освещённость.
При том, что стоимость систем технического зрения, способных обеспечить высокую точность определения положения и ориентации изделия, достаточно высока.

\subsection{Методы на основе касания}
При производстве свариваемых заготовок, как правило, большого размера, допускаются различные отклонения и допуски.
Для того, чтобы компенсировать низкую точность обработки изделий перед сваркой, разрабатывают алгоритмы определения положения изделия на основании касаний его рабочим инструментом.

Подобный метод использовался в работе~\cite{Fei_Gao_2015} и был направлен на определение смещений свариваемых конструкций.
Такой способ выгодно отличается от систем технического зрения тем, что он не требует дополнительного дорогостоящего оборудования при обеспечении высокой точности.
Как правило, суть таких алгоритмов заключается в медленном перемещении рабочей точки инструмента внутри области, в которой положение изделия является допустимым.
При обнаружении касания, координаты рабочей точки фиксируются и используются для дальнейшего пересчёта и определения положения изделия.
При выходе рабочей точки за пределы допустимой области, условия поиска изменяются, либо операция поиска прекращается из-за необходимости вмешательства оператора.


\section{Определение касания}
Для проведения опытных работ был выделен сварочный аппарат Panasonic YD-500KR, представленный на рисунке~\ref{fig:Find Touch:Panasonic YD-500KR}.

\begin{figure}[H]
    \centering
    \vspace{14pt}
    \includegraphics[height=10cm]{Figures/FindDetail/Panasonic_YD-500KR}
    \caption{Сварочный аппарат Panasonic YD-500KR}
    \label{fig:Find Touch:Panasonic YD-500KR}
\end{figure}

Сварочный аппарат используется совместно с устройством подачи проволоки, к которому подключается сварочная горелка.
Как видно на рисунке~\ref{fig:Find Touch:Panasonic YD-500KR}, сварочная горелка имеет эргономичную форму и не подразумевает возможности крепления к роботу.
Потому, для её крепления к фланцу робота был разработан и изготовлен специальный крепёж.
Фотография закреплённой на роботе горелки приведена на рисунке~\ref{fig:Find Touch:Weld Holder}.

\begin{figure}[H]
    \centering
    \vspace{14pt}
    \includegraphics[height=10cm]{Figures/FindDetail/WeldHolder}
    \caption{Разработанный крепёж для сварочной горелки}
    \label{fig:Find Touch:Weld Holder}
\end{figure}

Используемый сварочный аппарат предназначен для полуавтоматической сварки.
То есть, присадочный материал в виде проволоки играет роль электрода и непрерывно подаётся с катушки в сварочную ванну.
В процессе сварки, сварочная дуга образуется между присадочной проволокой и изделием, которое подключено к выходу сварочного аппарата с полярностью, обратной полярности проволоки.
При прикосновении изделия электродом, цепь сварочного источника замыкается, соответственно, напряжение в цепи падает.
Зафиксировав падение напряжения, можно определить наличие касания.

Как правило, современные сварочные аппараты, даже не предназначенные для роботизированной сварки, имеют возможность снятия показаний тока и напряжения, но сварочный аппарат Panasonic YD-500KR не современный и такой опции не имеет.
Для определения касания была разработана и реализована схема подключения, изображённая на рисунке~\ref{fig:Find Touch:Electrical Circuit}.

\begin{figure}[H]
    \centering
    \vspace{14pt}
    \includegraphics[height=10cm]{Figures/FindDetail/ElectricalCircuit}
    \caption{Электрическая схема для определения касания}
    \label{fig:Find Touch:Electrical Circuit}
\end{figure}

На схеме E11 и E12 -- источники постоянного напряжения сварочного аппарата для напряжения сварки и для питания внутренней логики соответственно.
E2 -- источник напряжения логики контроллера робота.
Контакт Kw условно обозначает наличие на электроде сварочного напряжения, а контакт Kt -- наличие контакта между электродом и изделием.
В начальном состоянии, когда оба контакта разомкнуты, светодиод оптопары U излучает свет под действием тока, протекающего от источника E12 через резистор R14.
Благодаря этому, фототранзистор оптопары находится в открытом положении, а напряжение между его контактами остаётся пренебрежительно малым.

При контакте электрода и изделия, то есть при замыкании контакта Kt, ток источника E12 протекает через этот контакт, минуя цепи с резисторами R13 и R14.
Светодиод оптопары перестаёт светить, фототранзистор запирается, а между его контактов фиксируется напряжение +24 В, что сигнализирует о наличии касания между электродом и изделием.


\section{Расчёт ориентации детали}
Наплавка выполняется на уплотнительную поверхность, находящуюся внутри корпуса.
Для удобства дальнейшей работы, в центре поверхности наплавки расположим связанную с ней систему координат.
Пример изделия с выделенной поверхностью наплавки приведён на рисунке~\ref{fig:Find Touch:Valve Example}.

\begin{figure}[H]
    \centering
    \vspace{14pt}
    \includegraphics[height=10cm]{Figures/FindDetail/DetailExample}
    \caption{Модель корпуса с выделенной поверхностью наплавки}
    \label{fig:Find Touch:Valve Example}
\end{figure}

Поверхность наплавки расположена параллельно плоскости верхнего фланца.
Поэтому, для определения ориентации связанной СК достаточно определить ориентацию плоскости фланца.

Для нахождения ориентации плоскости фланца, необходимо определить три точки на его поверхности.
Однако, для компенсации возможных неточностей в определении координат точек касания, предлагается использовать большее количество опорных точек.
Поиск точек на поверхности осуществляется при помощи движения рабочего инструмента робота вертикально сверху-вниз до фиксирования касания, либо до выхода из заранее определённой области поиска.
Найденные точки удобнее всего аппроксимировать плоскостью при помощи метода наименьших квадратов.

Рассмотрим уравнение плоскости:

\begin{equation}
    \label{eq:Plane}
    z = A x + B y + C,
\end{equation} \\
где $x$, $y$, $z$ -- координаты точек, принадлежащих плоскости, $A$ и $B$ -- коэффициенты наклона плоскости относительно осей $X$ и $Y$ соответственно, $C$ -- вертикальное смещение плоскости по оси $Z$.

Тогда функция для нахождения ошибки $i$-го измерения будет иметь следующий вид:

\begin{equation}
    \label{eq:Plane_MNK_e}
    e(x_i, y_i, z_i) = A x_i + B y_i + C - z_i.
\end{equation} \\

Задачей метода наименьших квадратов является минимизация суммы квадратов ошибок, то есть следующей функции:

\begin{equation}
    \label{eq:Plane_MNK_S}
    S(A, B, C) = \sum_{i=1}^{n} \left[ e(x_i, y_i, z_i) \right]^2.
\end{equation} \\

Функции~\ref{eq:Plane_MNK_S} достигает своего минимального значение в такой точке, в которой градиент этой функции равен нулю.

\begin{gather*}
    \nabla S(A, B, C) = 0, \\
    \begin{cases}
        \begin{aligned}
            \dfrac{\partial S(A, B, C)}{\partial A} = 2 \sum_{i=1}^{n} \left[
            \dfrac{\partial e(x_i, y_i, z_i)}{\partial A} \cdot e(x_i, y_i, z_i) \right] = 0 \\
            \dfrac{\partial S(A, B, C)}{\partial B} = 2 \sum_{i=1}^{n} \left[
            \dfrac{\partial e(x_i, y_i, z_i)}{\partial B} \cdot e(x_i, y_i, z_i) \right] = 0 \\
            \dfrac{\partial S(A, B, C)}{\partial C} = 2 \sum_{i=1}^{n} \left[
            \dfrac{\partial e(x_i, y_i, z_i)}{\partial C} \cdot e(x_i, y_i, z_i) \right] = 0
        \end{aligned}
    \end{cases}
\end{gather*}

\begin{align}
    \nonumber
    &\begin{cases}
         \begin{aligned}
             &\sum_{i=1}^{n} \left[ x_i \left( A x_i + B y_i - C - z_i \right) \right] = 0 \\
             &\sum_{i=1}^{n} \left[ y_i \left( A x_i + B y_i - C - z_i \right) \right] = 0 \\
             &\sum_{i=1}^{n} \left[ - \left( A x_i + B y_i - C - z_i \right) \right] = 0
         \end{aligned}
    \end{cases} \\ \nonumber
    &\begin{cases}
         \begin{aligned}
             &\sum_{i=1}^{n} \left[ A x_i^2 + B x_i y_i - C x_i - x_i z_i \right] = 0 \\
             &\sum_{i=1}^{n} \left[ A x_i y_i + B y_i^2 - C y_i - y_i z_i \right] = 0 \\
             &\sum_{i=1}^{n} \left[ A x_i + B y_i - C - z_i \right] = 0
         \end{aligned}
    \end{cases} \\ \nonumber
    &\begin{cases}
         \begin{aligned}
             &A \sum_{i=1}^{n} x_i^2
             + B \sum_{i=1}^{n} x_i y_i
             - C \sum_{i=1}^{n} x_i
             - \sum_{i=1}^{n} x_i z_i = 0 \\
             &A \sum_{i=1}^{n} x_i y_i
             + B \sum_{i=1}^{n} y_i^2
             - C \sum_{i=1}^{n} y_i
             - \sum_{i=1}^{n} y_i z_i = 0 \\
             &A \sum_{i=1}^{n} x_i
             + B \sum_{i=1}^{n} y_i
             - C \cdot n - \sum_{i=1}^{n} z_i = 0
         \end{aligned}
    \end{cases} \\
    &\begin{cases}
         \label{eq:Plane_MNK_Final_System}
         \begin{aligned}
             &A \sum_{i=1}^{n} x_i^2
             + B \sum_{i=1}^{n} x_i y_i
             - C \sum_{i=1}^{n} x_i
             = \sum_{i=1}^{n} x_i z_i \\
             &A \sum_{i=1}^{n} x_i y_i
             + B \sum_{i=1}^{n} y_i^2
             - C \sum_{i=1}^{n} y_i
             = \sum_{i=1}^{n} y_i z_i \\
             &A \sum_{i=1}^{n} x_i
             + B \sum_{i=1}^{n} y_i
             - C \cdot n = \sum_{i=1}^{n} z_i
         \end{aligned}
    \end{cases}
\end{align} \\

Система~\ref{eq:Plane_MNK_Final_System} имеет следующий вид:

\begin{equation}
    \label{eq:Plane_Cramer_System}
    \begin{cases}
        \begin{aligned}
            A a_{11} + B a_{12} + C a_{13} = b_1 \\
            A a_{21} + B a_{22} + C a_{23} = b_2 \\
            A a_{31} + B a_{32} + C a_{33} = b_3
        \end{aligned}
    \end{cases},
\end{equation} \\
где $A$, $B$ и $C$ -- искомые коэффициенты уравнения плоскости~\ref{eq:Plane},

\begin{alignat*}{3}
    a_{11} &= \sum_{i=1}^{n} x_i^2, & \qquad
    a_{12} &= \sum_{i=1}^{n} x_i y_i, & \qquad
    a_{13} &= - \sum_{i=1}^{n} x_i, \\
    a_{21} &= \sum_{i=1}^{n} x_i y_i, &
    a_{22} &= \sum_{i=1}^{n} y_i^2, &
    a_{23} &= - \sum_{i=1}^{n} y_i, \\
    a_{31} &= \sum_{i=1}^{n} x_i, &
    a_{32} &= \sum_{i=1}^{n} y_i, &
    a_{33} &= - n.
\end{alignat*} \\

Решим систему~\ref{eq:Plane_Cramer_System} методом Крамера:

\begin{alignat*}{3}
    A &= \dfrac{\Delta_A}{\Delta}, & \qquad
    B &= \dfrac{\Delta_B}{\Delta}, & \qquad
    C &= \dfrac{\Delta_C}{\Delta},
\end{alignat*} \\
где

\begin{alignat*}{2}
    \Delta &= \begin{vmatrix}
                  a_{11} & a_{12} & a_{13} \\
                  a_{21} & a_{22} & a_{23} \\
                  a_{31} & a_{32} & a_{33}
    \end{vmatrix}, & \qquad
    \Delta_A &= \begin{vmatrix}
                    b_1 & a_{12} & a_{13} \\
                    b_2 & a_{22} & a_{23} \\
                    b_3 & a_{32} & a_{33}
    \end{vmatrix}, \\
    \Delta_B &= \begin{vmatrix}
                    a_{11} & b_1 & a_{13} \\
                    a_{21} & b_2 & a_{23} \\
                    a_{31} & b_3 & a_{33}
    \end{vmatrix}, & \qquad
    \Delta_C &= \begin{vmatrix}
                    a_{11} & a_{12} & b_1 \\
                    a_{21} & a_{22} & b_2 \\
                    a_{31} & a_{32} & b_3
    \end{vmatrix}.
\end{alignat*} \\

Найдя коэффициенты уравнения плоскости~\ref{eq:Plane}, рассчитаем углы наклона плоскости фланца и, соответственно, углы наклона связанной с поверхностью наплавки системой координат относительно осей $X$ и $Y$ абсолютной системы координат:

\begin{align*}
    \alpha &= \arctan(A), \\
    \beta &= \arctan(B).
\end{align*}


\section{Расчёт положения детали}
Расчёт положения детали основывается на определении точек, лежащих на внутренней грани входного отверстия корпуса.
Поиск точек на грани входного отверстия происходит в повёрнутой системе координат, параметры которой были определены при поиске ориентации корпуса.
Таким образом, все найденные точки будут лежать в плоскости $XY$.

Так как входное отверстие имеет форму окружности, то для расчёта координат его центра достаточно трёх точек.
Но, как и при определении ориентации детали, для компенсации возможных неточностей в определении координат точек касания, предлагается использовать большее количество опорных точек.
Поиск точек осуществляется движением от произвольной точки внутри отверстия до соприкосновения с изделием или до выхода из области поиска.
Найденное произвольное количество точек аппроксимируется методом наименьших квадратов.

Уравнение окружности имеет следующий вид:

\begin{equation*}
    \left( x - x_0 \right)^2 + \left( y - y_0 \right)^2 = r^2,
\end{equation*} \\
где $x$, $y$ -- координаты точек, лежащих на окружности, $x_0$ и $y_0$ -- координаты центра окружности, $r$ - радиус окружности.

Функция ошибки $i$-го измерения будет иметь следующий вид:

\begin{equation}
    \begin{aligned}
        \label{eq:Circle_MNK_e}
        e(x_i, y_i) &= \left( x_i - x_0 \right)^2 + \left( y_i - y_0 \right)^2 - r^2 = \\
        &= x_i^2 + y_i^2 - 2 (x_i x_0 + y_i y_0) + x_0^2 + y_0^2 - r^2.
    \end{aligned}
\end{equation} \\

Тогда функция для минимизации квадрата ошибок будет иметь следующий вид:

\begin{equation*}
    S(x_0, y_0, r) = \sum_{i=1}^{n} \left[ e(x_i, y_i) \right]^2.
\end{equation*} \\

Задача сводится к поиску таких значений $x_0$ и $y_0$, при которой функция $S(x_0, y_0, r)$ достигает своего минимума, то есть её частные производные по $x_0$ и $y_0$ равны нулю:

\begin{align}
    \nonumber
    &\begin{cases}
         \begin{aligned}
             \dfrac{\partial S(x_0, y_0, r)}{\partial x_0} = 2 \sum_{i=1}^{n} \left[
             \dfrac{\partial e(x_i, y_i)}{\partial x_0} \cdot e(x_i, y_i) \right] = 0 \\
             \dfrac{\partial S(x_0, y_0, r)}{\partial y_0} = 2 \sum_{i=1}^{n} \left[
             \dfrac{\partial e(x_i, y_i)}{\partial y_0} \cdot e(x_i, y_i)\right] = 0 \\
             \dfrac{\partial S(x_0, y_0, r)}{\partial r} = 2 \sum_{i=1}^{n} \left[
             \dfrac{\partial e(x_i, y_i)}{\partial y_0} \cdot e(x_i, y_i)\right] = 0
         \end{aligned}
    \end{cases} \\ \nonumber
    &\begin{cases}
         \begin{aligned}
             &2 \sum_{i=1}^{n} \left[ (2 x_0 - 2 x_i) \cdot e(x_i, y_i) \right] = 0 \\
             &2 \sum_{i=1}^{n} \left[ (2 y_0 - 2 y_i) \cdot e(x_i, y_i) \right] = 0 \\
             &2 \sum_{i=1}^{n} \left[ 2 r \cdot e(x_i, y_i) \right] = 0
         \end{aligned}
    \end{cases} \\ \nonumber
    &\begin{cases}
         \begin{aligned}
             &4 \sum_{i=1}^{n} \left[ x_0 \cdot e(x_i, y_i) - x_i \cdot e(x_i, y_i) \right] = 0 \\
             &4 \sum_{i=1}^{n} \left[ y_0 \cdot e(x_i, y_i) - x_i \cdot e(x_i, y_i) \right] = 0 \\
             &4 n \cdot r \sum_{i=1}^{n} e(x_i, y_i) = 0
         \end{aligned}
    \end{cases} \\
    &\begin{cases}
         \label{eq:Circle_MNK_system}
         \begin{aligned}
             &n \cdot x_0 \sum_{i=1}^{n} e(x_i, y_i)
             - \sum_{i=1}^{n} \left[ x_i \cdot e(x_i, y_i) \right] = 0 \\
             &n \cdot y_0 \sum_{i=1}^{n} e(x_i, y_i)
             - \sum_{i=1}^{n} \left[ y_i \cdot e(x_i, y_i) \right] = 0 \\
             &\sum_{i=1}^{n} e(x_i, y_i) = 0
         \end{aligned}
    \end{cases}
\end{align} \\

Так как минимальным значением функции $S(x_0, y_0, r)$ является ноль, то:

\begin{equation}
    \label{eq:Circle_MNK_e_is_zero}
    \sum_{i=1}^{n} e(x_i, y_i) = 0.
\end{equation} \\

Подставив~\ref{eq:Circle_MNK_e_is_zero} в~\ref{eq:Circle_MNK_system}, получим:

\begin{equation*}
    \begin{cases}
        \begin{aligned}
            &\sum_{i=1}^{n} \left[ x_i \cdot e(x_i, y_i) \right] = 0 \\
            &\sum_{i=1}^{n} \left[ y_i \cdot e(x_i, y_i) \right] = 0 \\
            &\sum_{i=1}^{n} e(x_i, y_i) = 0
        \end{aligned}
    \end{cases}
\end{equation*} \\

Раскрыв $e(x_i, y_i)$ в соответствии с~\ref{eq:Circle_MNK_e} и преобразовав систему уравнений, получим:

\begin{align}
    \nonumber
    &\begin{cases}
         \begin{aligned}
             &\sum_{i=1}^{n} \left[ x_i \left(
             x_i^2 + y_i^2 - 2 (x_i x_0 + y_i y_0) + x_0^2 + y_0^2 - r^2 \right) \right] = 0 \\
             &\sum_{i=1}^{n} \left[ y_i \left(
             x_i^2 + y_i^2 - 2 (x_i x_0 + y_i y_0) + x_0^2 + y_0^2 - r^2 \right) \right] = 0 \\
             &\sum_{i=1}^{n} \left[
             x_i^2 + y_i^2 - 2 (x_i x_0 + y_i y_0) + x_0^2 + y_0^2 - r^2 \right] = 0
         \end{aligned}
    \end{cases} \\ \nonumber
    &\begin{cases}
         \begin{aligned}
             &\sum_{i=1}^{n} \left[
             x_i^3 + x_i y_i^2 - 2 x_i (x_i x_0 + y_i y_0)
             + x_i \left( x_0^2 + y_0^2 - r^2 \right) \right] = 0 \\
             &\sum_{i=1}^{n} \left[
             x_i^2 y_i + y_i^3 - 2 y_i (x_i x_0 + y_i y_0)
             + y_i \left( x_0^2 + y_0^2 - r^2 \right) \right] = 0 \\
             &\sum_{i=1}^{n} x_i^2 + \sum_{i=1}^{n} y_i^2
             - 2 \left( x_0 \sum_{i=1}^{n} x_i + y_0 \sum_{i=1}^{n} y_i \right)
             + n \left( x_0^2 + y_0^2 - r^2 \right) = 0
         \end{aligned}
    \end{cases} \\
    &\begin{cases}
         \label{eq:Circle_MNK_full_system}
         \begin{aligned}
             &\sum_{i=1}^{n} x_i^3 + \sum_{i=1}^{n} x_i y_i^2
             - 2 \sum_{i=1}^{n} x_i \left( x_0 \sum_{i=1}^{n} x_i + y_0 \sum_{i=1}^{n} y_i \right) + \\
             &\qquad + \sum_{i=1}^{n} x_i \left( x_0^2 + y_0^2 - r^2 \right) = 0 \\
             \\
             &\sum_{i=1}^{n} x_i^2 y_i + \sum_{i=1}^{n} y_i^3
             - 2 \sum_{i=1}^{n} y_i \left( x_0 \sum_{i=1}^{n} x_i + y_0 \sum_{i=1}^{n} y_i \right) + \\
             &\qquad + \sum_{i=1}^{n} y_i \left( x_0^2 + y_0^2 - r^2 \right) = 0 \\
             \\
             &x_0^2 + y_0^2 - r^2 = - \dfrac{1}{n} \left(
             \sum_{i=1}^{n} x_i^2 + \sum_{i=1}^{n} y_i^2
             - 2 \left( x_0 \sum_{i=1}^{n} x_i
             + y_0 \sum_{i=1}^{n} y_i \right)\right)
         \end{aligned}
    \end{cases}
\end{align} \\

Подставим значение выражения $x_0^2 + y_0^2 - r^2$ из третьего уравнения системы~\ref{eq:Circle_MNK_full_system} в её первые два уравнения:

\begin{align}
    \nonumber
    &\begin{cases}
         \begin{aligned}
             &\sum_{i=1}^{n} x_i^3 + \sum_{i=1}^{n} x_i y_i^2 - 2 \left(
             x_0 \sum_{i=1}^{n} x_i^2 + y_0 \sum_{i=1}^{n} x_i y_i \right) - \\
             &\qquad - \dfrac{1}{n} \sum_{i=1}^{n} x_i \left(
             \sum_{i=1}^{n} x_i^2 + \sum_{i=1}^{n} y_i^2
             - 2 \left( x_0 \sum_{i=1}^{n} x_i + y_0 \sum_{i=1}^{n} y_i \right)\right) = 0 \\
             \\
             &\sum_{i=1}^{n} x_i^2 y_i + \sum_{i=1}^{n} y_i^3 - 2 \left(
             x_0 \sum_{i=1}^{n} x_i y_i + y_0 \sum_{i=1}^{n} y_i^2 \right) - \\
             &\qquad - \dfrac{1}{n} \sum_{i=1}^{n} y_i \left(
             \sum_{i=1}^{n} x_i^2 + \sum_{i=1}^{n} y_i^2
             - 2 \left( x_0 \sum_{i=1}^{n} x_i + y_0 \sum_{i=1}^{n} y_i \right)\right) = 0
         \end{aligned}
    \end{cases} \\ \nonumber
    &\begin{cases}
         \begin{aligned}
             &2 x_0 \left( \dfrac{1}{n} \sum_{i=1}^{n} x_i \sum_{i=1}^{n} x_i - \sum_{i=1}^{n} x_i^2 \right)
             + 2 y_0 \left( \dfrac{1}{n} \sum_{i=1}^{n} x_i \sum_{i=1}^{n} y_i
             - \sum_{i=1}^{n} x_i y_i \right) + \\
             &\qquad + \sum_{i=1}^{n} x_i^3 + \sum_{i=1}^{n} x_i y_i^2
             - \dfrac{1}{n} \sum_{i=1}^{n} x_i
             \left( \sum_{i=1}^{n} x_i^2 + \sum_{i=1}^{n} y_i^2 \right) = 0 \\
             \\
             &2 x_0 \left( \dfrac{1}{n} \sum_{i=1}^{n} x_i \sum_{i=1}^{n} y_i
             - \sum_{i=1}^{n} x_i y_i \right)
             + 2 y_0 \left( \dfrac{1}{n} \sum_{i=1}^{n} x_i \sum_{i=1}^{n} x_i - \sum_{i=1}^{n} y_i^2 \right) + \\
             &\qquad + \sum_{i=1}^{n} x_i^2 y_i + \sum_{i=1}^{n} y_i^3
             - \dfrac{1}{n} \sum_{i=1}^{n} y_i
             \left(\sum_{i=1}^{n} x_i^2 + \sum_{i=1}^{n} y_i^2 \right) = 0
         \end{aligned}
    \end{cases} \\
    &\begin{cases}
         \label{eq:Circle_MNK_final_system}
         \begin{aligned}
             &2 x_0 \left( \sum_{i=1}^{n} x_i^2 - \dfrac{1}{n} \sum_{i=1}^{n} x_i \sum_{i=1}^{n} x_i \right)
             + 2 y_0 \left( \sum_{i=1}^{n} x_i y_i
             - \dfrac{1}{n} \sum_{i=1}^{n} x_i \sum_{i=1}^{n} y_i \right) = \\
             &\qquad = \sum_{i=1}^{n} x_i^3 + \sum_{i=1}^{n} x_i y_i^2
             - \dfrac{1}{n} \sum_{i=1}^{n} x_i \left( \sum_{i=1}^{n} x_i^2 + \sum_{i=1}^{n} y_i^2 \right) \\
             \\
             &2 x_0 \left( \sum_{i=1}^{n} x_i y_i
             - \dfrac{1}{n} \sum_{i=1}^{n} x_i \sum_{i=1}^{n} y_i \right)
             + 2 y_0 \left( \sum_{i=1}^{n} y_i^2 - \dfrac{1}{n} \sum_{i=1}^{n} x_i \sum_{i=1}^{n} x_i \right) = \\
             &\qquad = \sum_{i=1}^{n} x_i^2 y_i + \sum_{i=1}^{n} y_i^3
             - \dfrac{1}{n} \sum_{i=1}^{n} y_i \left( \sum_{i=1}^{n} x_i^2 + \sum_{i=1}^{n} y_i^2 \right)
         \end{aligned}
    \end{cases}
\end{align} \\

Нетрудно заметить, что система~\ref{eq:Circle_MNK_final_system} имеет следующий вид:

\begin{equation}
    \label{eq:SLAU_system}
    \begin{cases}
        \begin{aligned}
            x_0 a_{11} + y_0 a_{12} = b_1
            \\
            x_0 a_{21} + y_0 a_{22} = b_2
        \end{aligned}
    \end{cases},
\end{equation} \\
где

\begin{alignat*}{2}
    a_{11} &= 2 \left( \sum_{i=1}^{n} x_i^2 - \dfrac{1}{n} \sum_{i=1}^{n} x_i \sum_{i=1}^{n} x_i \right), & \qquad
    a_{12} &= 2 \left( \sum_{i=1}^{n} x_i y_i - \dfrac{1}{n} \sum_{i=1}^{n} x_i \sum_{i=1}^{n} y_i \right), \\
    a_{21} &= 2 \left( \sum_{i=1}^{n} x_i y_i - \dfrac{1}{n} \sum_{i=1}^{n} x_i \sum_{i=1}^{n} y_i \right), &
    a_{22} &= 2 \left( \sum_{i=1}^{n} y_i^2 - \dfrac{1}{n} \sum_{i=1}^{n} x_i \sum_{i=1}^{n} x_i \right),
\end{alignat*} \\

\begin{align*}
    b_1 &= \sum_{i=1}^{n} x_i^3 + \sum_{i=1}^{n} x_i y_i^2
    - \dfrac{1}{n} \sum_{i=1}^{n} x_i \left(
    \sum_{i=1}^{n} x_i^2 + \sum_{i=1}^{n} y_i^2 \right), \\
    b_2 &= \sum_{i=1}^{n} x_i^2 y_i + \sum_{i=1}^{n} y_i^3
    - \dfrac{1}{n} \sum_{i=1}^{n} y_i \left(
    \sum_{i=1}^{n} x_i^2 + \sum_{i=1}^{n} y_i^2 \right).
\end{align*} \\

Система~\ref{eq:SLAU_system} является нормальной формой системы двух линейных алгебраических уравнений с двумя неизвестными и имеет единственное решение.
Найдём его метод Крамера:

\begin{equation*}
    x_0 = \dfrac{\Delta_{x_0}}{\Delta}, \qquad
    y_0 = \dfrac{\Delta_{y_0}}{\Delta},
\end{equation*} \\
где

\begin{align*}
    \Delta = \begin{vmatrix}
                 a_{11} & a_{12} \\
                 a_{21} & a_{22}
    \end{vmatrix} &= a_{11} a_{22} - a_{12} a_{21}, \\
    \Delta_{x_0} = \begin{vmatrix}
                       b_1 & a_{12} \\
                       b_2 & a_{22}
    \end{vmatrix} &= a_{22} b_1 - a_{12} b_2, \\
    \Delta_{y_0} = \begin{vmatrix}
                       a_{11} & b_1 \\
                       a_{21} & b_2
    \end{vmatrix} &= a_{11} b_2 - a_{21} b_1.
\end{align*} \\

Зная положение входного отверстия изделия, его ориентацию и высоту от фланца до поверхности наплавки, можно точно восстановить положение и ориентацию связанной с поверхностью наплавки системой координат.


%\section{Расчёт ориентации горелки внутри корпуса}
%Наиболее сложными для наплавки являются корпуса, где поверхность наплавки шире, чем диаметр входного отверстия.
%На рисунке~\ref{fig:Find Touch:Best Example Valve} приведён эскиз такого корпуса.
%
%\begin{figure}[H]
%    \centering
%    \vspace{14pt}
%    \includegraphics[width=\linewidth]{Figures/Find Touch/Best Example Valve}
%    \caption{Пример корпуса с поверхностью наплавки шире входного отверстия}
%    \label{fig:Find Touch:Best Example Valve}
%\end{figure}
%
%При наплавке поверхностей внутри подобных корпусов, необходимо ориентировать сварочную горелку таким образом, чтобы она в любой точке рабочей траектории не соприкасалась с входным отверстием детали.
  % Определние положения и ориентации изделия
    \section{Генерация траекторий}

\subsection{Параметры траектории движения} \label{subsec:TrajectoryParameters}
В технологических процессах, связанных с операциями сварки и наплавки, как правило, определяются лишь основные сварочные параметры: вид сварки, присадочный материал, режимы работы сварочного аппарата.
В случае наплавки также указывается частота и амплитуда колебаний.
Но почти никогда не даётся рекомендаций касательно выбора типа сварочных колебаний.
Выбор конкретного паттерна колебаний чаще всего остаётся непосредственно за сварщиком, что приводит к ещё большему увеличению в разбросе качества конечной продукции.

Рассматриваемые в подразделе~\ref{subsec:WeldingMachines} установки для наплавки, как правило, реализуют только зигзагообразные колебания.
Решения для роботизированной сварки/наплавки, как, например, FANUC ArcTool и KUKA.ArcTech предлагают готовый инструментарий для задания сварочной траектории с колебаниями лишь из узкого набора предустановленных паттернов с ограниченной возможностью их параметрического редактирования.

Для того чтобы проследить влияние выбранного типа колебаний и их конкретных параметров на качество конечной продукции, в рамках этой работы был разработан программный модуль, позволяющий универсальным способом задавать траекторию движения сварочной горелки с независимым наложением на неё колебаний разных типов.

\subsection{Создание траектории движения}
Траектория движения сварочной горелки строится на основании двух составляющих: путь и накладываемые на него колебания.
При создании траектории движения, сначала выбирается тип пути.
После, задаётся тип колебаний и их основные параметры, прописанные в технологическом процессе для конкретного изделия.

Так как технологический процесс разработан для наплавки вручную, то все эти параметры указываются в виде допустимого диапазона с учётом возможных неточностей в работе сварщика.
В случае роботизированной наплавки, конкретные значения параметров выбираются из диапазона оператором исходя из внешнего вида предполагаемой траектории.

Как было описано в подразделе~\ref{subsec:ProgramStructure}, на контроллер робота программа движения передаётся в виде набора инструкций и используемых в программе положений.
Таким образом, траектория движения робота представляется в виде набора точек.
Количество точек напрямую зависит от параметров конкретной траектории и должно подбираться оператором исходя из её внешнего вида.

Для более простых траекторий с небольшим итоговым числом колебаний, количество точек может быть значительно меньше, чем для более сложных траекторий.
Недостаточное количество точек может привести к значительному снижению точности движения по траектории.
Чрезмерно завышенное число точек, в свою очередь, существенно сказывается на времени компиляции и загрузки программы работы.
К тому же, более старые версии контроллеров не всегда справляются с обработкой программ работы большого объёма.
Пример внешнего вида одной и той же траектории при разном количестве точек приведён на рисунке~\ref{fig:Trajectory:LowPointTrajectoryExample}.

\begin{figure}[H]
    \centering
    \vspace{14pt}
    \includegraphics[width=\linewidth]{Figures/Trajectory/LowPointTrajectoryExample}
    \caption{Траектория при разном количестве точек}
    \label{fig:Trajectory:LowPointTrajectoryExample}
\end{figure}

Для того, чтобы иметь возможность менять количество точек в траектории, сама траектория должна описываться в параметрическом виде, где координаты каждой точки зависят от некоторого параметра $t$.
Величина шага изменения этого параметра и будет влиять на количество точек.

Как было указано в подразделе~\ref{subsec:TrajectoryParameters}, траектория движения состоит из основного пути движения и наложенных на него колебаний.
И основной путь, и колебания представляют собой набор точек.

Для примера, рассмотрим траекторию, приведённую на рисунке~\ref{fig:Trajectory:LowPointTrajectoryExample}.
Это траектория движения по окружности с круговыми колебаниями, изображёнными на рисунке~\ref{fig:Trajectory:CircleOscillation}.

\begin{figure}[H]
    \centering
    \vspace{14pt}
    \includegraphics[width=\linewidth]{Figures/Trajectory/CircleOscillation}
    \caption{Пример круговых колебаний}
    \label{fig:Trajectory:CircleOscillation}
\end{figure}

Сперва, вычисляется массив из $n$ точек, из которых состоит основная траектория:

\begin{gather}
    \label{eq:Trajectory:CirclePath}
    \begin{cases}
        \begin{aligned}
            x_p = \dfrac{D}{2} \cdot \cos t \\
            \\
            y_p = \dfrac{D}{2} \cdot \sin t
        \end{aligned}
    \end{cases} \\
    \nonumber
    0 \leq t \leq 2 \pi,
\end{gather} \\
где $D$ -- диаметр окружности основной траектории.

Для корректной генерации колебаний, необходимо вычислить общее количество колебаний, которое зависит от частоты колебаний, протяжённости выбранного пути и скорости перемещения рабочего инструмента:

\begin{equation*}
    F_N = \dfrac{2 \pi \cdot D}{\upsilon} \cdot F,
\end{equation*} \\
где $\upsilon$ -- скорость движения по траектории, $F$ -- частота колебаний.

Вычисление точек колебаний осуществляется следующим образом:

\begin{gather}
    \begin{cases}
        \label{eq:Trajectory:CircleOscillation}
        \begin{aligned}
            x_o = \dfrac{A_w}{2} \cdot \sin t \\
            \\
            y_o = \dfrac{A_h}{2} \cdot \cos t
        \end{aligned}
    \end{cases} \\
    \nonumber
    0 \leq t \leq 2 \pi \cdot F_N,
\end{gather} \\
где $A_w$ -- амплитуда колебаний по ширине, $A_h$ -- амплитуда колебаний по высоте, а общее количество точек соотвествует количеству точек основной траектории $n$.

Таким образом, точки колебаний, полученные по формуле~\ref{eq:Trajectory:CircleOscillation}, описывают одинаковые окружности, которые, при наложении на основную траекторию, образуют колебания, изображённые на рисунке~\ref{fig:Trajectory:CircleOscillation}

Для каждой точки основной траектории рассчитывается угол направления колебания:

\begin{equation}
    \label{eq:Trajectory:Angle}
    a_i = \arctan \left( \dfrac{{y_p}_{i+1} - {y_p}_i}{{x_p}_{i+1} - {x_p}_{i}} \right),
\end{equation} \\
где $i \in [1, 2, ..., n - 1]$ -- номер точки траектории.

Для каждой из $n$ точек основного пути~\ref{eq:Trajectory:CirclePath} происходит сложение вектор-столбца, составленного из координат этой точки с повёрнутым на соответствующий номеру этой точки угол~\ref{eq:Trajectory:Angle} вектор-столбцом, образованным координатами соответствующей точки колебаний~\ref{eq:Trajectory:CircleOscillation}:

\begin{gather*}
    \begin{bmatrix}
        {x_t}_i \\
        {y_t}_i
    \end{bmatrix} =
    \begin{bmatrix}
        {x_p}_i \\
        {y_p}_i
    \end{bmatrix} +
    \begin{bmatrix}
        \cos a_i & - \sin a_i \\
        \sin a_i & \cos a_i
    \end{bmatrix}
    \begin{bmatrix}
        {x_o}_i \\
        {y_o}_i
    \end{bmatrix}.
\end{gather*} \\

Таким образом происходит наложение колебаний на основной путь движения, что позволяет получить итоговую траекторию движения.
Пример наложения круговых колебаний на траектории по прямой и по окружности изображён на рисунке~\ref{fig:Trajectory:CircleOscillationOnPath}

\begin{figure}[H]
    \centering
    \vspace{14pt}
    \includegraphics[width=\linewidth]{Figures/Trajectory/CircleOscillationOnPath}
    \caption{Пример наложения круговых колебаний на траектории движения по прямой и по окружности}
    \label{fig:Trajectory:CircleOscillationOnPath}
\end{figure}

Важной особенностью разработанного алгоритма является его универсальность для разных типов колебаний.
При добавлении колебаний нового типа, достаточно описать их в параметрической форме, подобной уравнению~\ref{eq:Trajectory:CircleOscillation}.

Например, параметрическая форма для колебаний восьмёркой, представленных на рисунке~\ref{fig:Trajectory:EightOscillation}, имеет следующий вид:

\begin{gather*}
    \begin{cases}
        \begin{aligned}
            &x_o = \dfrac{\dfrac{A_w}{2} \cdot \sin t \cos t}{1 + \sin^2 t} \\
            \\
            &y_o = \dfrac{\dfrac{A_h}{2} \cdot \cos t}{1 + \sin^2 t}
        \end{aligned}
    \end{cases} \\
    0 \leq t \leq 2 \pi \cdot F_N.
\end{gather*}

\begin{figure}[H]
    \centering
    \vspace{14pt}
    \includegraphics[width=\linewidth]{Figures/Trajectory/EightOscillation}
    \caption{Пример колебаний восьмёркой}
    \label{fig:Trajectory:EightOscillation}
\end{figure}

Наложение колебаний восьмёркой на траекторию движения по прямой и по окружности изображено на рисунке~\ref{fig:Trajectory:EightOscillationOnPath}

\begin{figure}[H]
    \centering
    \vspace{14pt}
    \includegraphics[width=\linewidth]{Figures/Trajectory/EightOscillationOnPath}
    \caption{Пример наложения колебаний восьмёркой на траектории движения по прямой и по окружности}
    \label{fig:Trajectory:EightOscillationOnPath}
\end{figure}

Одним из простейших видов колебаний являются колебания зигзагом, представленные на рисунках~\ref{fig:Trajectory:ZigzagOscillation} и~\ref{fig:Trajectory:ZigzagOscillationOnPath}.
Их параметрическая форма имеет следующий вид:

\begin{gather*}
    \begin{cases}
        \begin{aligned}
            &x_o = 0 \\
            &y_o = A_h \cdot \left( 0.5 - \left| t - 2 \floor*{\dfrac{t}{2}} - 1 \right| \right)
        \end{aligned}
    \end{cases} \\
    0 \leq t \leq 2 \cdot F_N.
\end{gather*}

\begin{figure}[H]
    \centering
    \vspace{14pt}
    \includegraphics[width=\linewidth]{Figures/Trajectory/ZigzagOscillation}
    \caption{Пример колебаний зигзагом}
    \label{fig:Trajectory:ZigzagOscillation}
\end{figure}

\begin{figure}[H]
    \centering
    \vspace{14pt}
    \includegraphics[width=\linewidth]{Figures/Trajectory/ZigzagOscillationOnPath}
    \caption{Пример наложения колебаний зигзагом на траектории движения по прямой и по окружности}
    \label{fig:Trajectory:ZigzagOscillationOnPath}
\end{figure}

    \chapter{Опытные работы}
Для проверки возможности и целесообразности выполнения роботизированной наплавки, был проведён ряд опытных работ.


\section{Ощупывание}
Одной из задач проведённых опытных работ было подтверждение или опровержение гипотезы о возможности достаточно точного определения положения и ориентации изделия при помощи нахождения точек на поверхности детали касанием электрода.
Работа проводилась по следующему сценарию:

\begin{enumerate}
    \item оператором пропускается небольшое количество сварочной проволоки;
    \item проволока обрезается таким образом, чтобы вылет проволоки составлял ~2-3 сантиметра;
    \item при помощи определения касания в определённом зафиксированном месте на сварочном столе, определяется длина вылета проволоки;
    \item определяются точки, лежащие на поверхности фланца корпуса;
    \item рассчитывается ориентация изделия;
    \item определяются точки, лежащие на окружности, образующей входное отверстие;
    \item рассчитывается положение изделия и поверхности наплавки.
\end{enumerate}

В результате эксперимента было обнаружено, что при контакте с изделием проволока может сильно деформироваться, что существенно сказывается на точности определения координат точек.

Неточности были скомпенсированы изменением подхода к поиску касания.
Первичное приближение находится с использованием прежних режимов работы.
После чего, происходит повторный поиск касания в суженной области поиска со значительным снижением скорости перемещения электрода.
Такой подход вместе с применением метода наименьших квадратов, описанного в подразделах~\ref{subsec:DetailOrientation} и~\ref{subsec:DetailPosition}, позволил добиться приемлемой для рассматриваемого производственного направления точности.


\section{Наплавка}
Также были проведены опытные работы по наплавке с целью проверки работоспособности сварочного аппарата и подбора режимов работы.

На рисунке~\ref{fig:Experiment:Welding1} изображён результат первых нескольких проходов без колебаний.
Швы тонкие, пористые, присутствуют несплавления.

\begin{figure}[H]
    \centering
    \vspace{14pt}
    \includegraphics[height=10cm]{Figures/Experiment/Welding1}
    \caption{Линейные проходы}
    \label{fig:Experiment:Welding1}
\end{figure}

Швы получились неудовлетворительного качества из-а неправильно подобранных режимов работы.
Общее качества шва существенно повышалось при увеличении сварочного тока и уменьшении скорости перемещения, что в итоге позволило добиться приемлемого качества.

Вторым этапом работ стала наплавка одного валика с круговыми колебаниями.
Траектория движения горелки представлена на рисунке~\ref{fig:Experiment:Freq}.

\begin{figure}[H]
    \centering
    \vspace{14pt}
    \includegraphics[height=10cm]{Figures/Experiment/Trajectory}
    \caption{Траектория движения рабочего инструмента при наплавке одного валика}
    \label{fig:Experiment:Freq}
\end{figure}

Фотография результата наплавки приведена на рисунке~\ref{fig:Experiment:Welding2}.

\begin{figure}[H]
    \centering
    \vspace{14pt}
    \includegraphics[height=10cm]{Figures/Experiment/Welding2}
    \caption{Первый кольцевой проход с колебаниями}
    \label{fig:Experiment:Welding2}
\end{figure}

Как видно из рисунка~\ref{fig:Experiment:Welding2}, наплавленный валик имеет большое количество пор, образовавшихся из-за слишком большой амплитуды колебаний и слишком высокой скорости перемещения горелки.

С откорректированными параметрами была проведена повторная наплавка валика.
Фотография результата наплавки приведена на рисунке~\ref{fig:Experiment:Welding3}.

\begin{figure}[H]
    \centering
    \vspace{14pt}
    \includegraphics[height=10cm]{Figures/Experiment/Welding3}
    \caption{Второй кольцевой проход с колебаниями}
    \label{fig:Experiment:Welding3}
\end{figure}

На рисунке~\ref{fig:Experiment:Welding3} видно, что поверхность, наплавленная с откорректированными параметрами, не имеет пор и несплавлений.
Валик получился однородным, состоящим из <<чешуек>>, что также косвено указывает на приемлемое качество.
  % Опытные работы

    \backmatter  % Конец нумерации

    \chapter{Заключение}
В процессе выполнения работы были решены различные задачи, необходимые для разработки информационно-управляющей системы роботизированного комплекса, назначением которого является наплавка износостойких поверхностей на изделия судовой арматуры.

Были рассмотренны различные подходы и способы автоматизации процесса наплавки, приведены их недостатки, обосновывающие использование в рассматриваемом производственном направлении шестиосевого робота-манипулятора.

Для взаимодействия оператора с роботизированным комплексом, было разработано приложение, предоставляющее весь необходимый для работы с комплексом функционал в виде графического пользовательского интерфейса.
Разработанное приложение предназначено не только для вывода графического интерфейса, но также для коммуникации с контроллером робота, считывания его состояний, создания, компиляции и загрузки программ работы.

Для минимизации влияния человеческого фактора и выполнения наплавочных работ в полностью автоматическом режиме, был разработан метод определения положения и ориентации детали.
Разработанный метод основан на определении точек на поверхности изделия при помощи касания сварочным электродом, а потому совмещает в себе отсутствие необходимости в приобретении дополнительного дорогостоящего оборудования и высокую точность.

В результате, были проведены опытные работы по определению положения и ориентации изделия, а также опытные наплавочные работы, которые подтвердили возможность и показали высокий потенциал автоматизации такого сложного производственного направления, как наплавка износостойких поверхностей.
  % Заключение
    \bibliographystyle{gost780u}
\bibliography{jabref}

%\addcontentsline{toc}{chapter}{Список литературы}
%\renewcommand\bibname{Список литературы}
%\makeatletter
%\bibliographystyle{gost780u}
%\renewcommand{\@biblabel}[1]{#1.}
%\makeatother
%\bibliography{jabref}     %% имя библиографической базы (bib-файла)
  % Список использованных источников
\end{document}
