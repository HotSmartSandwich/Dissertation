\section{Разработка программ движения робота}
В ходе выполнения работы, были рассмотрены различные варианты управления промышленным роботом-манипулятором.
В связи с проприетарностью встроенного программного обеспечения, при разработке информационно-управляющей системы было возможно использовать только ограниченный функционал, предлагаемый производителем оборудования.

\subsection{Способы создания программ}
Базовым и наиболее распространённым в промышленности способом является программирование напрямую через пульт управления (Teach Pendant).
Оператор вручную задаёт координаты каждой точки и ориентацию рабочего инструмента робота, либо сохраняет их в режиме обучения.
Главным преимуществом такого подхода является простота и относительно высокая скорость написания простых программ работы.
Однако, он категорически не подходит для задания более сложных траекторий, какой и является траектория наплавки.

Для написания программ, описывающих сложные траектории движения существуют различные CAD/CAM системы, единственным значимым недостатком которых является высокая стоимость и строго ограниченный инструментарий.
Сами же сгенерированные траектории движения представляют из себя массивы из множества точек.

Изучив возможность самостоятельного написания и загрузки программ работы робота в контроллер, было принято решение спроектировать и разработать собственное приложение, включающее в себя пользовательский графический интерфейс для создания и редактирования траекторий наплавки и постпроцессор -- программное обеспечение, позволяющее преобразовать заданную оператором траекторию в программу движения для контроллера робота.
Похожий подход был применён в работе~\cite{Nagata_2017}, но в ней рассматривается исключительно интерполяция уже готового набора данных дугами.

\subsection{Структура программ работы} \label{subsec:ProgramStructure}
Основные расширения файлов программ, с которыми работают контроллеры FANUC -- это <<.LS>> и <<.TP>>.

Файл с расширением <<.LS>> -- это листинг программы, текстовый файл с ASCII кодировкой, для открытия и редактирования которого может быть использован любой текстовый редактор.
Состоит из двух основных блоков: блок движений <</MN>>, описывающий все действия робота в виде инструкций, и блок позиций <</POS>>, в котором находится информация о координатах всех точек, используемых в программе.
Пример оформления и синтаксиса приведён в листинге~\ref{lst:RobotProgram:LS_Example}.

\begin{lstlisting}[caption={Пример оформления .LS файлов}, label={lst:RobotProgram:LS_Example}]
/PROG EXAMPLE.LS

/MN
:J P[1] 10% FINE;
:  WAIT 1.5(sec);
:L P[2] 250mm/sec CNT50;
:L P[3] 4sec CNT100;
:J P[1] 5% FINE;

/POS
P[1]{
    GP1:
    UF: 1, UT: 1,     CONFIG: 'N U T, 0, 0, 0',
    X = 729.8  mm,    Y = 134.9  mm,    Z = 181.1  mm,
    W = 191.2 deg,    P =  13.3 deg,    R =  14.2 deg
};
P[2]{
    GP1:
    UF: 1, UT: 1,     CONFIG: 'N U T, 0, 0, 0',
    X = 693.6  mm,    Y = 695.6  mm,    Z =   0.0  mm,
    W = 120.0 deg,    P =  -9.9 deg,    R = -46.8 deg
};
P[3]{
    GP1:
    UF: 1, UT: 1,     CONFIG: 'N U T, 0, 0, 0',
    X = 512.9  mm,    Y = 129.8  mm,    Z = -31.5  mm,
    W = 191.5 deg,    P =  43.1 deg,    R =  45.1 deg
};

/END
\end{lstlisting}

Файл с расширением <<.TP>> -- это скомпилированная программа в виде байт-кода.
Старые контроллеры FANUC не имеют пакета расширения <<ASCII Upload>>, позволяющего компилировать файлы с расширением <<.LS>> при их загрузке в контроллер.
Поэтому они должны быть скомпилированы предварительно.
Компиляция выполняется при помощи пакета программ FANUC WinOLPC.

\subsection{Конфигурация робота}
Для роботов FANUC целевая точка может задаваться двумя способами.

Первый способ -- через задание углов поворота каждого сочленения робота ($J_1$, $J_2$, $J_3$, $J_4$, $J_5$, $J_6$).
Пример задания целевой точки этим способом приведён в листинге~\ref{lst:RobotProgram:J16_TargetPoint}.

\begin{lstlisting}[caption={Задание целевой точки с помощью углов J1-J6}, label={lst:RobotProgram:J16_TargetPoint}]
P[1]{
    UF : 1, UT : 1,
    J1 = 49.399 deg,    J2 =  10.072 deg,    J3 = -24.105 deg,
    J4 =  0.000 deg,    J5 = -65.895 deg,    J6 =  40.601 deg
};
\end{lstlisting}

Второй способ -- через шесть координат, три из которых определяют положение целевой точки (X, Y, Z), а другие три -- ориентацию рабочего инструмента в абсолютной системе координат, связанной с основанием робота (W, P, R).
Пример задания целевой точки приведён в листинге~\ref{lst:RobotProgram:XYZWPR_TargetPoint}.

\begin{lstlisting}[caption={Задание целевой точки с помощью координат XYZWPR}, label={lst:RobotProgram:XYZWPR_TargetPoint}]
P[1]{
    UF : 1, UT : 1,     CONFIG : 'N U T, 0, 0, 0',
    X = 750.000  mm,    Y = 800.000  mm,    Z = 225.000  mm,
    W = 180.000 deg,    P =   0.000 deg,    R =  90.000 deg
};
\end{lstlisting}

Этот способ задания является наиболее удобным и информативным для оператора, так как ссылается на понятные человеку пространственные координаты.
Однако, он не является однозначным: в одной и той же точке, с одной и той же ориентацией рабочего инструмента, робот может иметь различные конфигурации, пример чего приведён на рисунке~\ref{fig:RobotProgram:RobotConfiguration}.

\begin{figure}[H]
    \centering
    \vspace{14pt}
    \includegraphics[width=\linewidth]{Figures/RobotProgram/RobotConfiguration}
    \caption{Разная конфигурация робота в одной и той же целевой точке}
    \label{fig:RobotProgram:RobotConfiguration}
\end{figure}

Конкретная конфигурация робота задаётся для каждой точки отдельно при помощи параметра <<CONFIG>> (листинг~\ref{lst:RobotProgram:XYZWPR_TargetPoint}).

Параметр <<CONFIG>> для роботов FANUC с кинематической схемой <<Puma>> состоит из трёх буквенных и трёх численных обозначений.
Буквенные обозначения позволяют однозначно определить положение звеньев робота.
Расшифровка буквенных обозначений приведена в таблице~\ref{tab:RobotProgram:RobotConfig}.

\begin{table}[H]
    \caption{Расшифровка буквенных обозначений}
    \label{tab:RobotProgram:RobotConfig}
    \begin{tabular}{|p{0.26\linewidth}|p{0.26\linewidth}|p{0.4\linewidth}|}
        \hline
        Обозначение & Расшифровка   & Значение                \\ \hline
        N           & wrist No-flip & запястье не перевёрнуто \\ \hline
        F           & wrist Flip    & запястье перевёрнуто    \\ \hline
        U           & elbow Up      & локоть вверх            \\ \hline
        D           & elbow Down    & локоть вниз             \\ \hline
        T           & config Top    & подход спереди          \\ \hline
        B           & config Bottom & подход сзади            \\ \hline
    \end{tabular}
\end{table}

Большая часть приведённых конфигураций является экзотической.
Как правило, в работе используется конфигурация <<NUT>> и в ходе выполнения программы она не изменяется.

Численные обозначения указывают на количество полных оборотов сверх необходимого угла поворота для многооборотных сочленений.
Как правило, количество полных оборотов необходимо учитывать при задании таких траекторий, когда происходит перекручивание запястья робота.

При аналитическом решении обратной задачи кинематики в моменты подобных перекручиваний уставка угла вместо того, чтобы, например, измениться от +160° до +190°, резко изменится в противоположном направлении до -170°, что продемонстрировано на рисунке~\ref{fig:RobotProgram:AngleChange}.

\begin{figure}[H]
    \centering
    \vspace{14pt}
    \includegraphics[height=9cm]{Figures/RobotProgram/AngleChange}
    \caption{Изменение угла}
    \label{fig:RobotProgram:AngleChange}
\end{figure}

Для того, чтобы исключить описанную проблему, можно либо отслеживать выход углов поворота сочленений за пределы интервала (-180°, +180°), в соответствии с чем, корректировать конфигурацию, либо изначально задавать целевые точки через углы поворота сочленений ($J_1$, $J_2$, $J_3$, $J_4$, $J_5$, $J_6$).
Оба этих способа являются практически равноценными с точки зрения реализации, так как оба подразумевают решение обратной задачи кинематики.

\subsection{Решение обратной задачи кинематики}
Обратная задача кинематики предполагает нахождение углов поворота звеньев робота при заданных положении и ориентации его рабочей точки.

Так как траектория движения рабочего инструмента робота состоит из множества точек, решение обратной задачи кинематики для каждой точки при помощи методов численной оптимизации может вызвать чрезмерные временные затраты.
Поэтому, для решения обратной задачи кинематики был выбран геометрический метод.
Начальными условиями для обратной задачи кинематики являются:

\begin{itemize}
    \item координаты целевой точки ($x_t$, $y_t$, $z_t$);
    \item ориентация рабочего инструмента в целевой точке ($w$, $p$, $r$);
    \item габариты установленного рабочего инструмента ($x_i$, $y_i$, $z_i$);
    \item кинематическая схема робота, представленная на рисунке~\ref{fig:RobotProgram:IKP:KinematicsSchema};
\end{itemize}

\begin{figure}[H]
    \centering
    \vspace{14pt}
    \includegraphics[height=10cm]{Figures/RobotProgram/IKP/KinematicsSchema}
    \caption{Кинематическая схема шестиосевых роботов FANUC}
    \label{fig:RobotProgram:IKP:KinematicsSchema}
\end{figure}

Положительные направления вращения звеньев указаны направлением стрелок.

Рассмотрим обратную задачу кинематики, как задачу поиска координат точек, соответствующих сочленениям робота и нахождения углов между этими точками.
Модель робота и искомые точка представлены на рисунке~\ref{fig:RobotProgram:IKP:dots_0}.

\begin{figure}[H]
    \centering
    \vspace{14pt}
    \includegraphics[height=10cm]{Figures/RobotProgram/IKP/dots_0}
    \caption{Модель робота с отмеченными искомыми точками}
    \label{fig:RobotProgram:IKP:dots_0}
\end{figure}

\subsubsection*{Шаг 1}
Целевую точку представим в виде вектор-столбца $\overrightarrow{P}$:

\begin{equation*}
    \overrightarrow{P} =
    \begin{bmatrix}
        x_t \\
        y_t \\
        z_t
    \end{bmatrix}.
\end{equation*} \\

Габариты рабочего инструмента так же представим в виде вектор-столбца $\overrightarrow{I}$:

\begin{equation*}
    \overrightarrow{I} =
    \begin{bmatrix}
        x_i \\
        y_i \\
        z_i
    \end{bmatrix}.
\end{equation*} \\

Ориентацию рабочего инструмента робота выразим при помощи матрицы сложного поворота $O^{3\times3}$:

\begin{gather*}
    \begin{split}
        O &=
        \begin{bmatrix}
            \cos(r) & -\sin(r) & 0 \\
            \sin(r) & \cos(r)  & 0 \\
            0       & 0        & 1
        \end{bmatrix} \times
        \begin{bmatrix}
            \cos(p)  & 0 & \sin(p) \\
            0        & 1 & 0       \\
            -\sin(p) & 0 & \cos(p)
        \end{bmatrix} \times
        \begin{bmatrix}
            1 & 0       & 0        \\
            0 & \cos(w) & -\sin(w) \\
            0 & \sin(w) & \cos(w)  \\
        \end{bmatrix} = \\
        &=
        \begin{bmatrix}
            \text{C}(p) \text{C}(r) & \text{S}(w) \text{S}(p) \text{C}(r) - \text{C}(w) \text{S}(r) & \text{C}(w) \text{S}(p) \text{C}(r) + \text{S}(w) \text{S}(r) \\
            \text{C}(p)             & \text{S}(w) \text{S}(p) \text{S}(r) + \text{C}(w) \text{C}(r) & \text{C}(w) \text{S}(p) \text{S}(r) - \text{S}(w) \text{C}(r) \\
            - \text{S}(p)           & \text{S}(w) \text{C}(p)                                       & \text{C}(w) \text{C}(p)
        \end{bmatrix},
    \end{split}
\end{gather*} \\
где $\text{S}(\alpha) = \sin(\alpha)$, $\text{С}(\alpha) = \cos(\alpha)$.

Найдём векторы $\overrightarrow{p_5}$ и $\overrightarrow{p_4}$, указывающие на конец пятого и четвёртого сочленений робота соответственно:

\begin{align*}
    \overrightarrow{p_5} &= \overrightarrow{P} - O \times \overrightarrow{I}, \\
    \overrightarrow{p_4} &= \overrightarrow{p_5} - O \times
    \begin{bmatrix}
        0 \\
        0 \\
        l_5
    \end{bmatrix}.
\end{align*} \\

Проекция вектора $\overrightarrow{p_4}$ на плоскость $XY$ позволяет найти угол $j_1$, так как это единственный угол, отвечающий за вращение в этой плоскости:

\begin{equation*}
    j_1 = \arctan \left( \dfrac{\overrightarrow{p_4}_Y}{\overrightarrow{p_4}_X} \right),
\end{equation*} \\
где $\overrightarrow{p_4}_X$ и $\overrightarrow{p_4}_Y$ -- проекции вектора $\overrightarrow{p_4}$ на соответствующие оси.

\subsubsection*{Шаг 2}
Для нахождения углов $j_2$ и $j_3$ рассмотрим робота в плоскости $XZ$.
Для этого удобнее прибегнуть не к проекциям, а к развороту всех сочленений и звеньев робота на угол $-j_1$ вокруг оси $Z$, проходящей вертикально из основания робота.
Модель робота после поворота представлена на рисунке~\ref{fig:RobotProgram:IKP:dots_1}.

\begin{figure}[H]
    \centering
    \vspace{14pt}
    \includegraphics[height=10cm]{Figures/RobotProgram/IKP/dots_1}
    \caption{Модель робота после первого поворота}
    \label{fig:RobotProgram:IKP:dots_1}
\end{figure}

Рассчитаем новые координаты известных векторов:

\begin{gather*}
    \overrightarrow{p_2}_{j_1} =
    \begin{bmatrix}
        l_1 \\
        0   \\
        0
    \end{bmatrix}, \\
    \overrightarrow{p_4}_{j_1} =
    \begin{bmatrix}
        \cos(-j_1) & -\sin(-j_1) & 0 \\
        \sin(-j_1) & \cos(-j_1)  & 0 \\
        0          & 0           & 1
    \end{bmatrix} \times \overrightarrow{p_4}, \\
    \overrightarrow{p_5}_{j_1} =
    \begin{bmatrix}
        \cos(-j_1) & -\sin(-j_1) & 0 \\
        \sin(-j_1) & \cos(-j_1)  & 0 \\
        0          & 0           & 1
    \end{bmatrix} \times \overrightarrow{p_5}, \\
    \overrightarrow{P}_{j_1} =
    \begin{bmatrix}
        \cos(-j_1) & -\sin(-j_1) & 0 \\
        \sin(-j_1) & \cos(-j_1)  & 0 \\
        0          & 0           & 1
    \end{bmatrix} \times \overrightarrow{P}.
\end{gather*} \\

\subsubsection*{Шаг 3}
Несложно убедиться, что точки $p_1$, $p_2$, $p_3$, $p_4$ лежат в плоскости $XZ$.

Задачу нахождения координат точки $p_3$ рассмотрим как задачу нахождения пересечения двух окружностей с центрами в точках $p_2$ и $p_4$ и радиусами $r_1$ и $r_2$ соответственно.
Эскиз представлен на рисунке~\ref{fig:RobotProgram:IKP:p3}.

\begin{figure}[H]
    \centering
    \vspace{14pt}
    \includegraphics[width=\linewidth]{Figures/RobotProgram/IKP/p3}
    \caption{Нахождение координат точки $p_3$}
    \label{fig:RobotProgram:IKP:p3}
\end{figure}

Радиусы окружностей $r_1$ и $r_2$ выразим через длины звеньев робота, а расстояние между центрами окружностей $d$ через расстояние между точками $p_2$ и $p_4$:

\begin{gather*}
    r_1 = l_2, \\
    r_2 = \sqrt{l_3^2 + l_4^2}, \\
    d = \left| \overrightarrow{p_4} - \overrightarrow{p_2} \right|.
\end{gather*} \\

Найдём перпендикуляр $h$ через сторону $a$ прямоугольного треугольника, образованного точками $p_2$, $M$ и $p_3$:

\begin{align*}
    \begin{cases}
        a^2 + h^2 = r_1^2 \\
        b^2 + h^2 = r_2^2 \\
        a + b = d
    \end{cases} & \Rightarrow
    \begin{cases}
        a^2 - b^2 = r_1^2 - r_2^2 \\
        b = d - a
    \end{cases}, \\
    a^2 - \left( d - a \right)^2 &= r_1^2 - r_2^2, \\
    a^2 - \left( d - 2da + a^2 \right) &= r_1^2 - r_2^2, \\
    2da - d^2 &= r_1^2 - r_2^2, \\
    2da &= r_1^2 - r_2^2 + d^2, \\
    a &= \dfrac{r_1^2 - r_2^2 + d^2}{2d}, \\
    h &= \sqrt{r_1^2 - a^2}.
\end{align*} \\

Вычислим координаты точки $M$ в виде вектора и сформируем вектор $\overrightarrow{p_3}_{j_1}$:

\begin{gather*}
    \overrightarrow{M} = \overrightarrow{p_2}_{j_1} + \dfrac{a}{d} \left( \overrightarrow{p_4}_{j_1} - \overrightarrow{p_2}_{j_1} \right), \\
    \overrightarrow{p_3}_{j_1} =
    \begin{bmatrix}
        \overrightarrow{M}_X - \dfrac{h}{d} \left( \overrightarrow{p_4}_{j_1} - \overrightarrow{p_2}_{j_1} \right)_Z \\
        0                                                                                                            \\
        \overrightarrow{M}_Z + \dfrac{h}{d} \left( \overrightarrow{p_4}_{j_1} - \overrightarrow{p_2}_{j_1} \right)_X
    \end{bmatrix},
\end{gather*} \\
где индексы $X$ и $Z$ обозначают проекцию вектора на соответствующую ось.

Зная координаты вектора $\overrightarrow{p_3}_{j_1}$, найдём углы $j_2$ и $j_3$:

\begin{gather*}
    j_2 = \dfrac{\pi}{2} - \arctan \left( \dfrac
    {\left( \overrightarrow{p_3}_{j_1} - \overrightarrow{p_2}_{j_1} \right)_Z}
    {\left( \overrightarrow{p_3}_{j_1} - \overrightarrow{p_2}_{j_1} \right)_X} \right), \\
    j_3 = \arctan \left( \dfrac
    {\left( \overrightarrow{p_4}_{j_1} - \overrightarrow{p_3}_{j_1} \right)_Z}
    {\left( \overrightarrow{p_4}_{j_1} - \overrightarrow{p_3}_{j_1} \right)_X} \right) - \arctan\left( \dfrac{l_3}{l_4} \right).
\end{gather*} \\

\subsubsection*{Шаг 4}
Повернём звенья робота вокруг третьего сочленения на угол $j_3$, тем самым расположив четвёртое звено робота параллельно плоскости $XY$.
Это необходимо для того, чтобы определить угол $j_4$ через проекции точек $p_4$ и $p_5$ на плоскость $YZ$.
Поворот осуществляется на угол $j_3$, а не $-j_3$, так как направление положительного вращения сочленения $p_3$ обратно направлению положительного вращения вокруг оси $Y$, что продемонстрировано на рисунке~\ref{fig:RobotProgram:IKP:KinematicsSchema}.
Модель робота после поворота представлена на рисунке~\ref{fig:RobotProgram:IKP:dots_2}.

\begin{figure}[H]
    \centering
    \vspace{14pt}
    \includegraphics[height=10cm]{Figures/RobotProgram/IKP/dots_2}
    \caption{Модель робота после второго поворота}
    \label{fig:RobotProgram:IKP:dots_2}
\end{figure}

Рассчитаем новые координаты векторов:

\begin{gather*}
    \overrightarrow{p_4}_{j_3} =
    \begin{bmatrix}
        \cos(j_3)  & 0 & \sin(j_3) \\
        0          & 1 & 0         \\
        -\sin(j_3) & 0 & \cos(j_3)
    \end{bmatrix} \times \left( \overrightarrow{p_4}_{j_1} - \overrightarrow{p_3}_{j_1} \right) + \overrightarrow{p_3}_{j_1}, \\
    \overrightarrow{p_5}_{j_3} =
    \begin{bmatrix}
        \cos(j_3)  & 0 & \sin(j_3) \\
        0          & 1 & 0         \\
        -\sin(j_3) & 0 & \cos(j_3)
    \end{bmatrix} \times \left( \overrightarrow{p_5}_{j_1} - \overrightarrow{p_3}_{j_1} \right) + \overrightarrow{p_3}_{j_1}, \\
    \overrightarrow{P}_{j_3} =
    \begin{bmatrix}
        \cos(j_3)  & 0 & \sin(j_3) \\
        0          & 1 & 0         \\
        -\sin(j_3) & 0 & \cos(j_3)
    \end{bmatrix} \times \left( \overrightarrow{P}_{j_1} - \overrightarrow{p_3}_{j_1} \right) + \overrightarrow{p_3}_{j_1}.
\end{gather*} \\

Для нахождения угла $j_4$ рассмотрим точки $p_4$ и $p_5$ в плоскости $YZ$ (см. рисунок~\ref{fig:RobotProgram:IKP:j4}).

\begin{figure}[H]
    \centering
    \vspace{14pt}
    \includegraphics[height=10cm]{Figures/RobotProgram/IKP/j4}
    \caption{Угол $j_4$ в плоскости $YZ$}
    \label{fig:RobotProgram:IKP:j4}
\end{figure}

Из рисунка~\ref{fig:RobotProgram:IKP:j4} найдём угол $j_4$:

\begin{gather*}
    j_4 = -\dfrac{\pi}{2} - \arctan \left( \dfrac
    {\left( \overrightarrow{p_5}_{j_3} - \overrightarrow{p_4}_{j_3} \right)_Z}
    {\left( \overrightarrow{p_5}_{j_3} - \overrightarrow{p_4}_{j_3} \right)_Y} \right),
\end{gather*} \\
где индексы $Y$ и $Z$ обозначают проекцию вектора на соответствующую ось.

\subsubsection*{Шаг 5}
Произведём поворот четвёртого звена робота на угол $j_4$, расположив тем самым точки $p_4$ и $p_5$ в плоскости $XZ$ (см. рисунок~\ref{fig:RobotProgram:IKP:j5}).
Рассчитаем новые координаты векторов:

\begin{gather*}
    \overrightarrow{p_5}_{j_4} =
    \begin{bmatrix}
        1 & 0         & 0          \\
        0 & \cos(j_4) & -\sin(j_4) \\
        0 & \sin(j_4) & \cos(j_4)  \\
    \end{bmatrix} \times \left( \overrightarrow{p_5}_{j_3} - \overrightarrow{p_4}_{j_3} \right) + \overrightarrow{p_4}_{j_3}, \\
    \overrightarrow{P}_{j_4} =
    \begin{bmatrix}
        1 & 0         & 0          \\
        0 & \cos(j_4) & -\sin(j_4) \\
        0 & \sin(j_4) & \cos(j_4)  \\
    \end{bmatrix} \times \left( \overrightarrow{P}_{j_3} - \overrightarrow{p_4}_{j_3} \right) + \overrightarrow{p_4}_{j_3}.
\end{gather*} \\


\begin{figure}[H]
    \centering
    \vspace{14pt}
    \includegraphics[height=10cm]{Figures/RobotProgram/IKP/j5}
    \caption{Угол $j_5$ в плоскости $XZ$}
    \label{fig:RobotProgram:IKP:j5}
\end{figure}

Из рисунка~\ref{fig:RobotProgram:IKP:j5} найдём угол $j_5$:

\begin{gather*}
    j_5 = \arctan \left( \dfrac
    {\left( \overrightarrow{p_5}_{j_4} - \overrightarrow{p_4}_{j_4} \right)_Z}
    {\left( \overrightarrow{p_5}_{j_4} - \overrightarrow{p_4}_{j_4} \right)_X} \right).
\end{gather*} \\

\subsubsection*{Шаг 6}
Произведём поворот пятого звена робота на угол $j_5$, что позволит расположить точки $p_5$ и $P$ в плоскости $YZ$.
Рассчитаем новые координаты вектора $\overrightarrow{p_5}$ и вектора $\overrightarrow{P}$, обозначающего целевую точку:

\begin{gather*}
    \overrightarrow{p_5}_{j_5} =
    \begin{bmatrix}
        \cos(j_5)  & 0 & \sin(j_5) \\
        0          & 1 & 0         \\
        -\sin(j_5) & 0 & \cos(j_5)
    \end{bmatrix} \times \left( \overrightarrow{p_5}_{j_4} - \overrightarrow{p_4}_{j_4} \right) + \overrightarrow{p_4}_{j_4}, \\
    \overrightarrow{P}_{j_5} =
    \begin{bmatrix}
        \cos(j_5)  & 0 & \sin(j_5) \\
        0          & 1 & 0         \\
        -\sin(j_5) & 0 & \cos(j_5)
    \end{bmatrix} \times \left( \overrightarrow{P}_{j_4} - \overrightarrow{p_4}_{j_4} \right) + \overrightarrow{p_4}_{j_4},
\end{gather*} \\
где индексы $X$ и $Z$ обозначают проекцию вектора на соответствующую ось.

Угол $j_6$ найдём аналогично углу $j_4$ (см. рисунок~\ref{fig:RobotProgram:IKP:j6}).

\begin{figure}[H]
    \centering
    \vspace{14pt}
    \includegraphics[height=10cm]{Figures/RobotProgram/IKP/j6}
    \caption{Угол $j_6$ в плоскости $YZ$}
    \label{fig:RobotProgram:IKP:j6}
\end{figure}

Из рисунка~\ref{fig:RobotProgram:IKP:j6} найдём угол $j_6$:

\begin{gather*}
    j_6 = -\dfrac{pi}{2} - \arctan \left( \dfrac
    {\left( \overrightarrow{P}_{j_5} - \overrightarrow{p_5}_{j_5} \right)_Z}
    {\left( \overrightarrow{P}_{j_5} - \overrightarrow{p_5}_{j_5} \right)_X} \right).
\end{gather*} \\

Описанная последовательность нахождения углов и поворотов звеньев робота позволяет из заданного положения рабочего инструмента и его ориентации, рассчитать углы поворота всех сочленений робота.

Несмотря на то, что алгоритм может выглядеть излишне громоздким, его программная реализация в составе системы управления является достаточно простой и лаконичной.
Алгоритм совмещает в себе минимальную вычислительную сложность и высокую точность за счёт того, что в ходе вычислений не используются ни численные методы оптимизации, ни неустойчивые обратные тригонометрические функции, как, например, арккосинус, имеющий низкую вычислительную точность для углов, близких к $0$ или $\pi$ \cite{bib:Fu_1987}.
Так же выгодным отличием такого подхода от решения обратной задачи кинематики с использованием методов численной оптимизации является гарантированное сохранение заданной конфигурации, в то время как оптимизационное решение может сойтись в локальный минимум или привести к неестественному позиционированию.
