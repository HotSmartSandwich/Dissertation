\chapter{Описание проблемы}
На рисунке~\ref{fig:Problem:Stop Valve} изображён запорный клапан в разрезе.

\begin{figure}[H]
    \centering
    \vspace{14pt}
    \includegraphics[width=\linewidth]{Figures/Problem/StopValve}
    \caption{Запорный клапан в разрезе}
    \label{fig:Problem:Stop Valve}
\end{figure}

Выделенные на рисунке~\ref{fig:Problem:Stop Valve} области называются уплотнительными поверхностями и, как правило, подвержены повышенному износу по причине возникновения трения <<металл по металлу>>.

Для увеличения срока службы конечных изделий, на арматуростроительных предприятиях, в том числе на ПАО <<Аскольд>>, выполняются технологические процессы наплавки уплотнительных поверхностей износостойкими материалами.

На ПАО <<Аскольд>> наплавка таких поверхностей осуществляется по технологии ручной дуговой наплавки неплавящимся вольфрамовым электродом в среде защитного инертного газа.
Наплавка осуществляется в несколько слоёв.
Каждый слой, в свою очередь, состоит из нескольких концентрических окружностей, называемых валиками.
Отдельные валики наплавляются последовательно, с перерывами.
Это необходимо для того, чтобы температура изделия не превышала температуру, допустимую технологическим процессом.
После наплавки каждого валика проводится зачистка поверхности от шлака и брызг металла, а также осуществляется визуальный осмотр качества наплавки.
На рисунке~\ref{fig:Problem:Welding Detail} приведён пример изделия в процессе выполнения наплавки.

\begin{figure}[H]
    \centering
    \vspace{14pt}
    \includegraphics[width=\linewidth]{Figures/Problem/WeldingDetail}
    \caption{Изделие во время выполнения наплавки}
    \label{fig:Problem:Welding Detail}
\end{figure}

На рисунке~\ref{fig:Problem:Welding Detail} слева изображена деталь сразу после наплавки второго валика до зачистки поверхности.
Видно, что наплавленный слой подвержен включениям шлака тёмного цвета.
Справа изображена эта же деталь после наплавки третьего валика и зачистки поверхности металлической щёткой: поверхность чистая, имеет металлический блеск.
В таком виде детали направляются на дальнейшую механическую обработку.

Для рассматриваемого производственного направления характерен высокий процент неизбежно-технологического брака, закладываемого в стоимость продукции.
Основной причиной брака является человеческий фактор: операции наплавки крайне требовательны к строгому соблюдению технологического процесса.
При выполнении наплавки вручную, такие параметры, как скорость перемещения горелки, частота и амплитуда колебаний, возможно соблюдать лишь приблизительно.
Кроме того, к наплавке износостойкими материалами допускаются только высококвалифицированные электрогазосварщики 5-го и 6-го разрядов, что приводит к потребности в узкоспециализированных и высокооплачиваемых кадрах.

